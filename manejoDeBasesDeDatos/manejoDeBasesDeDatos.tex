\documentclass[a4paper]{book}

\usepackage{import}
\input{./../../commands.tex}
\usepackage{cmands}

\newcommand{\asignatura}{Manejo de Bases de Datos}
\newcommand{\autor}{Juan Pablo Sierra Useche}
\newcommand{\semestre}{Tercer}
\newcommand{\universidad}{Universidad Del Rosario}
\newcommand{\escuela}{Escuela de Ingeniería, Ciencia y tecnología}
\newcommand{\carrera}{Matemáticas Aplicadas y Ciencias de la Computación}


\begin{document}

    \titPag
    \tableofcontents

    % ====| B E G I N     O F    W O R K    S P A C E |==== %

    \begin{chapter}{Primer Semestre}
    \label{chap:primer_semestre}

    \section{04.08.2020 $\mybar$ Sistemas Manejadores de Bases de Datos}
    \label{sec:sistemas_manejadores_de_bases_de_datos}

        Un \textbf{DBMS} puede definirce como un el resultado de la únion de
        dos componentes: una colección de datos relacionados entre sí (a la que
        normalmente se le llama \textbf{base de datos}) y un conjunto de
        programas para poder acceder a esos datos. Y cumple la funcion de
        brindar un metodo de guardado y acceso de datos, conveniente y
        eficiente.

        \subsection{Conceptos Importantes}
        \label{ssec:conceptos_importantes}

        \begin{description}
            \item[Relación] Representación de datos en dos dimensiones con uno
                o más atributos y con cero o más tuplas.
            \item[Tupla] En el \textbf{modelo relacional de datos} una tupla
                representa a un elemento dentro de una relación.
            \item[Columna] En el \textbf{modelo relacional de datos} una
                columna representa un atributo de los elementos en una
                relación.
            \item[Modelo Relacional de Datos] Es una collección de tablas,
                donde el usuario puede consultar lo que tienen adentro,
                insertar nuevas tuplas, eliminarlas, modificarlas.
            \item[Esquema] Es un diseño lógico de la \textbf{base de datos},
                que incluye tanto las relaciónes como los atributos de las
                mismal, sus tipos (opcional) y sus <<primary key>> y <<foreign
                key>>.
        \end{description}










    \end{chapter}




    % ====| E N D    O F    W O R K    S P A C E |==== %

    \printbibliography
\end{document}
