\documentclass[usenames,dvipsnames]{beamer}

\usepackage{transparent}
\usepackage{graphicx}   %Use Images
\usepackage{pdfpages}
\usepackage{xifthen}
\usepackage{xcolor}
\usepackage{import}
\usepackage{tikz}
\usepackage{tipa}
\usepackage{cgloss4e}
\usepackage{qtree}
\usepackage{enumerate}
\usepackage{longtable}
\usepackage{textgreek}
\usepackage{parskip}
\usepackage{xpatch}

\makeatletter
\patchcmd\beamer@@tmpl@frametitle{\insertframetitle}{\insertsection-\insertframetitle}{}{}
\makeatother

\usetheme{Antibes}
\usecolortheme{dove}
\setbeamertemplate{frametitle}[default][center]
\setlength{\parskip}{12pt}

\AtBeginSection[]
{
 \begin{frame}<beamer>
 \frametitle{Plan de Acción}
 \tableofcontents[currentsection]
 \end{frame}
}

\title{Monitoria \(Sesi\text{ó}n_{\color{olive}{2.0}}\)}
\subtitle{Algoritmos y Estructuras de Datos}
\author{Juan Pablo Sierra Useche}
\institute{Universidad del Rosario}
\date{15.08.2020}



% - New figure
\newcommand{\incfig}[2][1]{
    \def\svgwidth{#1\columnwidth}
    \import{./figures/}{#2.pdf_tex}
}

\begin{document}

\begin{frame}
\maketitle
\end{frame}


\section{Antes de empezar con lo bueno\ldots}
\label{sec:sec0}
\begin{frame}
\begin{alertblock}{{\color{red}TODO:}}\pause
    \begin{enumerate}[{1) }]
        \item Leer bien el problema.\pause
        \item Separar las instrucciones.\pause
        \item Desarrollarlas por separado.\pause
        \item Únificar.\pause
        \item Probar y corregir errores.\pause
        \item \textbf{Disfrutarlo}\ldots Eres lo máximo por haber solucionado el problema.
    \end{enumerate}
\end{alertblock}
\end{frame}


\section{Stacks}
\label{sec:1}
\begin{frame}
\label{fr:ej1}
\frametitle{Ejercicio 1}
\only<1>{
    \begin{block}{{\color{blue}instrucciones:}}
        Escriba un programa que lea una secuencia de caracteres y los imprima
        en orden inverso. Para esto use un \textbf{stack} de caracteres.
    \end{block}
}
\only<2>{
    \begin{block}{{\color{blue}instrucciones:}}
        {\color{pink}Escriba un programa que lea una secuencia de caracteres y
        los imprima en orden inverso}. Para esto use un \textbf{stack} de
        caracteres.
    \end{block}
}
\only<3>{
    \begin{block}{{\color{blue}instrucciones:}}
        Escriba un programa que lea una secuencia de caracteres y los imprima
        en orden inverso.{\color{olive} Para esto use un \textbf{stack} de
        caracteres.}
    \end{block}
}
\only<4>{
\begin{block}{\color{violet}PA' LOS HARDCORE}
    Hacer un programa que \textbf{lea un archivo} y utilice el programa
    anterior, para imprimir cada palabra del texto al reves \textbf{de forma
    individual}.
\end{block}
}
\end{frame}

\begin{frame}
\label{fr:ej2}
\frametitle{Ejercicio 2}
\only<1>{
\begin{block}{{\color{blue}instrucciones:}}
    escriba un programa que reciba un \textbf{stack} de numeros enteros
    arbitrarios y que retorne el mismo stack pero donde solo los elementos que
    son menores a los elementos anteriores permanezcan en el \textbf{stack}.
    \begin{center}
        {\color{orange} [}10,8,11,4,7,2,1{\color{orange} ]} \(\to\) {\color{orange} [}10,8,4,2,1{\color{orange} ]}
    \end{center}
\end{block}
}
\only<2>{
\begin{block}{{\color{blue}instrucciones:}}
    Escriba un programa que reciba un \textbf{stack} de numeros enteros
    arbitrarios y que retorne el mismo stack pero donde solo los elementos que
    son menores a los elementos anteriores permanezcan en el \textbf{stack}.
    \begin{center}
        {\color{orange} [}10,8,{\color{red}11},4,{\color{red}7},2,1{\color{orange} ]} \(\to\) {\color{orange} [}10,8,4,2,1{\color{orange} ]}
    \end{center}
\end{block}
}
\end{frame}


\section{Queue}
\label{sec:2}
\begin{frame}
\label{fr:ej3}
\frametitle{Ejercicio 3}
\only<1>{
\begin{block}{{\color{blue}instrucciones:}}
    Usando las operaciones estándar de un \textbf{queue}, escriba un programa
    que elimina el elemento medio de una cola de números de doble precisión.
\end{block}
}
\only<2>{
\begin{block}{{\color{blue}instrucciones:}}
Usando las operaciones estándar de un \textbf{queue}, escriba un programa que {\color{olive}elimina el elemento medio de una cola de números de doble precisión.}
\end{block}
}
\only<3>{
\begin{block}{{\color{blue}instrucciones:}}
    Usando las operaciones estándar de un \textbf{queue}, escriba un programa que elimina el elemento medio de una cola de números de doble precisión.
\end{block}
}
\only<4>{
\begin{block}{{\color{blue}instrucciones:}}
    {\color{teal}Usando las operaciones estándar} de un \textbf{queue}, escriba
    un programa que elimina el elemento medio de una cola de números de doble
    precisión.
\end{block}
}
\only<5>{
\begin{block}{\color{violet}PÆ' L05 QU3 T€NG4N G@Na5 D +}
    Modificar el programa para que retorne un \textbf{queue} con el elemento de
    la mitad en la primera posición.
\end{block}
}
\end{frame}
\begin{frame}
\label{fr:ej4}
\frametitle{Ejercicio 4}
\only<1>{
\begin{block}{{\color{blue}instrucciones:}}
    En el banco de Chía ubicado en el centro comercial más concurrido del
municipio tienen el problema de que la gente suele colarse muy seguido, para
solucionar esté problema licitaron con varios grupos de expertos en estructuras
de datos; de todos ellos escogieron dos: el primero debe entregar un algoritmo
que al conectarse con las cámaras clasifique con un número impar a aquellos que
se hayan colado, y el segundo grupo debe encargarse de reconocer la posición de
aquellos que se colaron (para que así el guardia los detecte) y además los debe
sacar de la estructura de datos. Como estamos en pandemia y la Universidad no
nos puede prestar las camaras para probar el primer algoritmo se nos encargo
hacer el segundo.  \end{block}
}
\end{frame}


\section{Mi quiz del semestre pasado.}
\begin{frame}
\label{fr:ej5}
\frametitle{Ejercicio 5}
\only<1>{
\begin{block}{{\color{blue}instrucciones:}}
    Escriba la función:
    \begin{center}
        {\color{orange}int}
        howMany({\color{orange}list}\(<\){\color{orange}int}\(>\)
        L,{\color{orange}int} k)
    \end{center}
    que recibe una lista de enteros \textbf{L} y un entero \textbf{k}. Mediante
    un ciclo \textbf{while}, la función debe encontrar el número de veces que
    se encuentra \textbf{k} en \textbf{L} y retornar esta cantidad. Puede
    modificar la lista si lo considera necesario.
}
\only<2>{
\begin{block}{{\color{blue}instrucciones:}}
    Escriba la función:
    \begin{center}
        {\color{orange}int}
        howMany({\color{orange}list}\(<\){\color{orange}int}\(>\)
        L,{\color{orange}int} k)
    \end{center}
    que {\color{brown}recibe una lista de enteros \textbf{L} y un entero \textbf{k}.} Mediante
    un ciclo \textbf{while}, la función debe encontrar el número de veces que
    se encuentra \textbf{k} en \textbf{L} y retornar esta cantidad. Puede
    modificar la lista si lo considera necesario.
}
\only<3>{
\begin{block}{{\color{blue}instrucciones:}}
    Escriba la función:
    \begin{center}
        {\color{orange}int}
        howMany({\color{orange}list}\(<\){\color{orange}int}\(>\)
        L,{\color{orange}int} k)
    \end{center}
    que recibe una lista de enteros \textbf{L} y un entero \textbf{k}. {\color{cyan}Mediante
    un ciclo \textbf{while}}, la función debe encontrar el número de veces que
    se encuentra \textbf{k} en \textbf{L} y retornar esta cantidad. Puede
    modificar la lista si lo considera necesario.
}
\only<4>{
\begin{block}{{\color{blue}instrucciones:}}
    Escriba la función:
    \begin{center}
        {\color{orange}int}
        howMany({\color{orange}list}\(<\){\color{orange}int}\(>\)
        L,{\color{orange}int} k)
    \end{center}
    que recibe una lista de enteros \textbf{L} y un entero \textbf{k}. Mediante
    un ciclo \textbf{while}, {\color{magenta}la función debe encontrar el
    número de veces que se encuentra \textbf{k} en \textbf{L} y retornar esta
    cantidad}. Puede modificar la lista si lo considera necesario.
}
\only<5>{
\begin{block}{{\color{blue}instrucciones:}}
    Escriba la función:
    \begin{center}
        {\color{orange}int}
        howMany({\color{orange}list}\(<\){\color{orange}int}\(>\)
        L,{\color{orange}int} k)
    \end{center}
    que recibe una lista de enteros \textbf{L} y un entero \textbf{k}. Mediante
    un ciclo \textbf{while}, la función debe encontrar el número de veces que
    se encuentra \textbf{k} en \textbf{L} y retornar esta cantidad. {\color{green}Puede
    modificar la lista si lo considera necesario.}
}
\end{block}
\end{frame}


\section{Ñapa\ldots}
\label{sec:extra}

\begin{frame}
\frametitle{\ldots}

\begin{figure}[ht]
    \centering
    \incfig{cow}
\end{figure}

\end{frame}




\end{document}
