\documentclass[dvipsnames,a4paper]{book}

% ====| P A C K A G E S |==== %
\usepackage{import}
\usepackage{nicematrix}
%\usepackage[backend=biber, style=authoryear-icomp]{biblatex}    %Bibliography stuff
%\ProvidesPackage{cmands}
% ====| P A C K A G E S |==== %
\usepackage{import}
%\usepackage[backend=biber, style=authoryear-icomp]{biblatex}    %Bibliography stuff
%\ProvidesPackage{cmands}
% ====| P A C K A G E S |==== %
\usepackage{import}
%\usepackage[backend=biber, style=authoryear-icomp]{biblatex}    %Bibliography stuff
%\ProvidesPackage{cmands}
% ====| P A C K A G E S |==== %
\usepackage{import}
%\usepackage[backend=biber, style=authoryear-icomp]{biblatex}    %Bibliography stuff
%\input{./../../commands.tex}
\usepackage[margin=1in,includefoot]{geometry}   %Margins
\usepackage[utf8]{inputenc} %Language stuff
\usepackage [latin1]{inputenc}  %Spanish symbols
\usepackage[spanish]{babel} %Sets document language to spanish
\usepackage{tcolorbox}  %Frame boxes
\usepackage{enumerate}  %Lists options
\usepackage{mfirstuc}   %I use it to capitalize words
\usepackage{graphicx}   %Use Images
\usepackage{listings}   %For displaying code
\usepackage{titlesec}   %Costume titles/sections/...
\usepackage{hyperref}   %Linking options
\usepackage{multicol}	%Use column
\usepackage{amsmath}    %Display equations options
\usepackage{amssymb}    %More symbols
\usepackage{titling}    %Use title variables in other places
\usepackage{xcolor}     %To manage colors
\usepackage{transparent}%For figures
\usepackage{pdfpages}   %For figures
% =========================== %

% ====| P A C K A G E S    S E T T I N G S |==== %
% \addbibresource{/media/jpi/Data/01_Education/04_bibiografia/bibliography.bib} %<--- Bibliography path
\setlength{\columnsep}{1cm}
\hypersetup{
    colorlinks,
    citecolor=black,
    filecolor=black,
    linkcolor=black,
    urlcolor=black
}
% ============================================== %

% ====| P E R S O N A L    C O M M A N D S    &    E N V I R O N M E N T S|==== %

% - New figure
\newcommand{\incfig}[2][1]{%
    \def\svgwidth{#1\columnwidth}
    \import{./figures/}{#2.pdf_tex}
}

\pdfsuppresswarningpagegroup=1

% - New page
\newcommand{\np}{\null\newpage}

% - vertical thic line
\newcommand\mybar{\kern1pt\rule[-\dp\strutbox]{.8pt}{\baselineskip}\kern1pt}

% - new homework
\newenvironment{tarea}[3]
    {
        \null\newpage
        \begin{tcolorbox}
            \textbf{\asignatura\ -\ \autor}
            \subsection{\capitalisewords{#1}}
            \label{ssec:#1}
            \begin{flushright}
            \textbf{Desde:}  #2 \\
            \textbf{Hasta:}  #3 \\
            \end{flushright}
        \end{tcolorbox}

    \begin{enumerate}[{Ejercicio} 1.]
    }
    {
        \end{enumerate}
        \np
    }



% ============================================================================= %


% ====| H E A T H E R S    S E T T I N G S |==== %

\titleformat{\chapter}[display]
    {\normalfont\huge\bfseries\raggedleft}{\chaptertitlename\ \thechapter}
    {20pt}{\Huge}
\titleformat{\section}[display]
    {\Large\bfseries}{}
    {0em}{}[\titlerule]


\newcommand{\titPag}{
    \begin{titlepage}
        \begin{flushright}
            \textsc{\large {\semestre\ Semestre}}\\
            \line(1,0){450} \\
            [0.635cm]
            \huge{\bfseries \asignatura} \\
            [0.2cm]
            \line(1,0){350} \\
            \LARGE{\bfseries \autor} \\
            [16.25cm]
        \end{flushright}
        \begin{flushright}
        \textsc{
            \universidad \\
            [0.1cm]
            \escuela \\
            [0.1cm]
            \carrera
        }
        \end{flushright}
    \end{titlepage}
}
% ============================================== %

% ====| C O D E    I N    F I L E S    S E T T I N G S |==== %
\definecolor{codegreen}{rgb}{0,0.6,0}
\definecolor{codegray}{rgb}{0.5,0.5,0.5}
\definecolor{codepurple}{rgb}{0.58,0,0.82}
\definecolor{backcolour}{rgb}{0.95,0.95,0.92}

\lstdefinestyle{mystyle}{
    backgroundcolor=\color{backcolour},
    commentstyle=\color{codegreen},
    keywordstyle=\color{magenta},
    numberstyle=\tiny\color{codegray},
    stringstyle=\color{codepurple},
    basicstyle=\ttfamily\footnotesize,
    breakatwhitespace=false,
    breaklines=true,
    captionpos=b,
    keepspaces=true,
    numbers=left,
    numbersep=5pt,
    showspaces=false,
    showstringspaces=false,
    showtabs=false,
    tabsize=2
}

\lstset{style=mystyle}
% ========================================================== %

\usepackage[margin=1in,includefoot]{geometry}   %Margins
\usepackage[utf8]{inputenc} %Language stuff
\usepackage [latin1]{inputenc}  %Spanish symbols
\usepackage[spanish]{babel} %Sets document language to spanish
\usepackage{tcolorbox}  %Frame boxes
\usepackage{enumerate}  %Lists options
\usepackage{mfirstuc}   %I use it to capitalize words
\usepackage{graphicx}   %Use Images
\usepackage{listings}   %For displaying code
\usepackage{titlesec}   %Costume titles/sections/...
\usepackage{hyperref}   %Linking options
\usepackage{multicol}	%Use column
\usepackage{amsmath}    %Display equations options
\usepackage{amssymb}    %More symbols
\usepackage{titling}    %Use title variables in other places
\usepackage{xcolor}     %To manage colors
\usepackage{transparent}%For figures
\usepackage{pdfpages}   %For figures
% =========================== %

% ====| P A C K A G E S    S E T T I N G S |==== %
% \addbibresource{/media/jpi/Data/01_Education/04_bibiografia/bibliography.bib} %<--- Bibliography path
\setlength{\columnsep}{1cm}
\hypersetup{
    colorlinks,
    citecolor=black,
    filecolor=black,
    linkcolor=black,
    urlcolor=black
}
% ============================================== %

% ====| P E R S O N A L    C O M M A N D S    &    E N V I R O N M E N T S|==== %

% - New figure
\newcommand{\incfig}[2][1]{%
    \def\svgwidth{#1\columnwidth}
    \import{./figures/}{#2.pdf_tex}
}

\pdfsuppresswarningpagegroup=1

% - New page
\newcommand{\np}{\null\newpage}

% - vertical thic line
\newcommand\mybar{\kern1pt\rule[-\dp\strutbox]{.8pt}{\baselineskip}\kern1pt}

% - new homework
\newenvironment{tarea}[3]
    {
        \null\newpage
        \begin{tcolorbox}
            \textbf{\asignatura\ -\ \autor}
            \subsection{\capitalisewords{#1}}
            \label{ssec:#1}
            \begin{flushright}
            \textbf{Desde:}  #2 \\
            \textbf{Hasta:}  #3 \\
            \end{flushright}
        \end{tcolorbox}

    \begin{enumerate}[{Ejercicio} 1.]
    }
    {
        \end{enumerate}
        \np
    }



% ============================================================================= %


% ====| H E A T H E R S    S E T T I N G S |==== %

\titleformat{\chapter}[display]
    {\normalfont\huge\bfseries\raggedleft}{\chaptertitlename\ \thechapter}
    {20pt}{\Huge}
\titleformat{\section}[display]
    {\Large\bfseries}{}
    {0em}{}[\titlerule]


\newcommand{\titPag}{
    \begin{titlepage}
        \begin{flushright}
            \textsc{\large {\semestre\ Semestre}}\\
            \line(1,0){450} \\
            [0.635cm]
            \huge{\bfseries \asignatura} \\
            [0.2cm]
            \line(1,0){350} \\
            \LARGE{\bfseries \autor} \\
            [16.25cm]
        \end{flushright}
        \begin{flushright}
        \textsc{
            \universidad \\
            [0.1cm]
            \escuela \\
            [0.1cm]
            \carrera
        }
        \end{flushright}
    \end{titlepage}
}
% ============================================== %

% ====| C O D E    I N    F I L E S    S E T T I N G S |==== %
\definecolor{codegreen}{rgb}{0,0.6,0}
\definecolor{codegray}{rgb}{0.5,0.5,0.5}
\definecolor{codepurple}{rgb}{0.58,0,0.82}
\definecolor{backcolour}{rgb}{0.95,0.95,0.92}

\lstdefinestyle{mystyle}{
    backgroundcolor=\color{backcolour},
    commentstyle=\color{codegreen},
    keywordstyle=\color{magenta},
    numberstyle=\tiny\color{codegray},
    stringstyle=\color{codepurple},
    basicstyle=\ttfamily\footnotesize,
    breakatwhitespace=false,
    breaklines=true,
    captionpos=b,
    keepspaces=true,
    numbers=left,
    numbersep=5pt,
    showspaces=false,
    showstringspaces=false,
    showtabs=false,
    tabsize=2
}

\lstset{style=mystyle}
% ========================================================== %

\usepackage[margin=1in,includefoot]{geometry}   %Margins
\usepackage[utf8]{inputenc} %Language stuff
\usepackage [latin1]{inputenc}  %Spanish symbols
\usepackage[spanish]{babel} %Sets document language to spanish
\usepackage{tcolorbox}  %Frame boxes
\usepackage{enumerate}  %Lists options
\usepackage{mfirstuc}   %I use it to capitalize words
\usepackage{graphicx}   %Use Images
\usepackage{listings}   %For displaying code
\usepackage{titlesec}   %Costume titles/sections/...
\usepackage{hyperref}   %Linking options
\usepackage{multicol}	%Use column
\usepackage{amsmath}    %Display equations options
\usepackage{amssymb}    %More symbols
\usepackage{titling}    %Use title variables in other places
\usepackage{xcolor}     %To manage colors
\usepackage{transparent}%For figures
\usepackage{pdfpages}   %For figures
% =========================== %

% ====| P A C K A G E S    S E T T I N G S |==== %
% \addbibresource{/media/jpi/Data/01_Education/04_bibiografia/bibliography.bib} %<--- Bibliography path
\setlength{\columnsep}{1cm}
\hypersetup{
    colorlinks,
    citecolor=black,
    filecolor=black,
    linkcolor=black,
    urlcolor=black
}
% ============================================== %

% ====| P E R S O N A L    C O M M A N D S    &    E N V I R O N M E N T S|==== %

% - New figure
\newcommand{\incfig}[2][1]{%
    \def\svgwidth{#1\columnwidth}
    \import{./figures/}{#2.pdf_tex}
}

\pdfsuppresswarningpagegroup=1

% - New page
\newcommand{\np}{\null\newpage}

% - vertical thic line
\newcommand\mybar{\kern1pt\rule[-\dp\strutbox]{.8pt}{\baselineskip}\kern1pt}

% - new homework
\newenvironment{tarea}[3]
    {
        \null\newpage
        \begin{tcolorbox}
            \textbf{\asignatura\ -\ \autor}
            \subsection{\capitalisewords{#1}}
            \label{ssec:#1}
            \begin{flushright}
            \textbf{Desde:}  #2 \\
            \textbf{Hasta:}  #3 \\
            \end{flushright}
        \end{tcolorbox}

    \begin{enumerate}[{Ejercicio} 1.]
    }
    {
        \end{enumerate}
        \np
    }



% ============================================================================= %


% ====| H E A T H E R S    S E T T I N G S |==== %

\titleformat{\chapter}[display]
    {\normalfont\huge\bfseries\raggedleft}{\chaptertitlename\ \thechapter}
    {20pt}{\Huge}
\titleformat{\section}[display]
    {\Large\bfseries}{}
    {0em}{}[\titlerule]


\newcommand{\titPag}{
    \begin{titlepage}
        \begin{flushright}
            \textsc{\large {\semestre\ Semestre}}\\
            \line(1,0){450} \\
            [0.635cm]
            \huge{\bfseries \asignatura} \\
            [0.2cm]
            \line(1,0){350} \\
            \LARGE{\bfseries \autor} \\
            [16.25cm]
        \end{flushright}
        \begin{flushright}
        \textsc{
            \universidad \\
            [0.1cm]
            \escuela \\
            [0.1cm]
            \carrera
        }
        \end{flushright}
    \end{titlepage}
}
% ============================================== %

% ====| C O D E    I N    F I L E S    S E T T I N G S |==== %
\definecolor{codegreen}{rgb}{0,0.6,0}
\definecolor{codegray}{rgb}{0.5,0.5,0.5}
\definecolor{codepurple}{rgb}{0.58,0,0.82}
\definecolor{backcolour}{rgb}{0.95,0.95,0.92}

\lstdefinestyle{mystyle}{
    backgroundcolor=\color{backcolour},
    commentstyle=\color{codegreen},
    keywordstyle=\color{magenta},
    numberstyle=\tiny\color{codegray},
    stringstyle=\color{codepurple},
    basicstyle=\ttfamily\footnotesize,
    breakatwhitespace=false,
    breaklines=true,
    captionpos=b,
    keepspaces=true,
    numbers=left,
    numbersep=5pt,
    showspaces=false,
    showstringspaces=false,
    showtabs=false,
    tabsize=2
}

\lstset{style=mystyle}
% ========================================================== %

\usepackage[margin=1in,includefoot]{geometry}   %Margins
\usepackage[utf8]{inputenc} %Language stuff
%\usepackage [latin1]{inputenc}  %Spanish symbols
\usepackage[spanish]{babel} %Sets document language to spanish
\usepackage{tcolorbox}  %Frame boxes
\usepackage{enumerate}  %Lists options
\usepackage{mfirstuc}   %I use it to capitalize words
\usepackage{graphicx}   %Use Images
\usepackage{listings}   %For displaying code
\usepackage{titlesec}   %Costume titles/sections/...
\usepackage{hyperref}   %Linking options
\usepackage{multicol}	%Use column
\usepackage{amsmath}    %Display equations options
\usepackage{amssymb}    %More symbols
\usepackage{titling}    %Use title variables in other places
\usepackage{xcolor}     %To manage colors
\usepackage{transparent}%For figures
\usepackage{pdfpages}   %For figures
% =========================== %

% ====| P A C K A G E S    S E T T I N G S |==== %
% \addbibresource{/media/jpi/Data/01_Education/04_bibiografia/bibliography.bib} %<--- Bibliography path
% ============================================== %

% ====| P E R S O N A L    C O M M A N D S    &    E N V I R O N M E N T S|==== %

% - New figure
\newcommand{\incfig}[2][1]{%
    \def\svgwidth{#1\columnwidth}
    \import{./figures/}{#2.pdf_tex}
}

\pdfsuppresswarningpagegroup=1

% - New page
\newcommand{\np}{\null\newpage}

% - new homework
\newenvironment{tarea}[3]
    {
        \null\newpage
        \begin{tcolorbox}
            \textbf{\asignatura\ -\ \autor}
            \subsection{\capitalisewords{#1}}
            \label{ssec:#1}
            \begin{flushright}
            \textbf{Desde:}  #2 \
            \textbf{Hasta:}  #3 \
            \end{flushright}
        \end{tcolorbox}

    \begin{enumerate}[{Ejercicio} 1.]
    }
    {
        \end{enumerate}
        \np
    }

% - new observation
\newenvironment{obs}
    {
        \begin{flushleft}
       \textbf{Observación}\
        \line(1,0){200} \
        \end{flushleft}
    }
    {
        \begin{flushright}
        \line(1,0){200}
        \end{flushright}
    }
% ============================================================================= %

\begin{document}
\section{16.08.2020 - Taller 1}
\label{sec:taller_1}

\begin{enumerate}[{Ej1. }]
\item En una clase el {\color{teal} 60\% de los estudiantes son genios}, al
    {\color{brown} 70\% les gusta} el chocolate y el {\color{orange} 40\%
    cae en las dos categorías}. ¿Cuál es la probabilidad de que un
    estudiante seleccionado al azar no sea genio ni le guste el chocolate?

\subsubsection{Respuesta}

\begin{figure}[ht]
    \centering
    \incfig[0.4]{t1ej1}
\end{figure}
Para resolver esté problema es necesario encontrar el valor de \(A\cup B\),
pues de esta forma solo sería necesario calcular \(1-\left(A\cup B\right)\)
para obtener la respuesta a la probabilidad de seleccionar al azar un
estudiante que no sea genio ni que le guste el chocolate.

\begin{align*}
    P\left(A\right)= 0.7&/P\left(B\right)=0.6/P\left(A\cap B\right)=0.4\\
    P\left(A\cup B\right) &= P\left(A\right)+P\left(B\right)-P\left(A\cap B\right)\\
    &=0.7+0.6-0.4\\
    &=0.9\\
    \\
    P\left(\overline{A\cup B}\right) &= 1-P\left(A\cup B\right)\\
    &= 1-0.9\\
    &=0.1
\end{align*}

Por lo tanto se puede concluir que la probabilidad de que al azar sea seleccionado un estudiante que no sea genio ni le guste el chocolate es de {\color{orange} 0.1}.

\item  Un dado de seis caras está cargado para que la probabilidad que caiga
    {\color{OrangeRed} en un número par sea dos veces la probabilidad de que
    caiga en un número impar}. {\color{olive} Construya un modelo de
    probabilidad para un lanzamiento de este dado y calcule la probabilidad de
    que el resulatado sea menor a 4}.

\subsubsection{Respuesta}

\begin{figure}[ht]
    \centering
    \incfig[0.33]{t1ej2}
\end{figure}

Como la probabilidad entre cada uno de los lados es igual para los elementos
que estan en cada uno de los subconjuntos se sabe lo siguiente:

\begin{itemize}
    \item \(P\left\{i\right\}=\frac{2}{9},i\in\left\{2,4,6\right\} \)
    \item \(P\left\{j\right\}=\frac{1}{9},j\in\left\{1,3,5\right\} \)
\end{itemize}

Como solo con los valores {\color{Mahogany} 1,2,3} se cumple la condición, se
tiene que la probabilidad que se nos pide es igual
{\color{BrickRed} \(P\left(\left\{1\right\}\right)+P\left(\left\{2\right\}
\right)+P\left(\left\{3\right\}\right)\)} lo que es igual a {\color{
BrickRed}\(\frac{2}{9}+\frac{1}{9}+\frac{1}{9}\)}, lo que nos lleva a concluir
que la probabilidad de que al lanzar el dado cargado caiga un valor menor a
cuatro es de {\color{RedViolet} \(\frac{4}{9}\)}.

\item Una moneda se lanza repetidamente {\color{Salmon} hasta que sale cara}.
    Determine el espacio muestral de este experimento.
\subsubsection{Respuesta}
Como lo que se está observando es la cantidad de lansamientos requeridos, el
espacio muestral se conforma por \(n\), tal que \(\Omega = \left\{n|n\text{ es
la cantidad de veces que se lanza la moneda}\right\}\).


\item Usted entra a un torneo de ajedez en el que {\color{Sepia} debe jugar
    contra 3} oponentes. Aunque los oponentes están fijos,
    {\color{MidnightBlue} usted puede escoger el orden en que los enfrente
    orden en que los enfrenta}. De experiencias anteriores, usted sabe cuál es
    la probabilidad de derrotar a cada uno de los
    oponentes.{\color{JungleGreen} Usted gana el torneo si logra derrotar a dos
    de forma consecutiva}. Si usted quiere maximizar la probabilidad de ganar
    el torneo, {\color{Orchid} muestre que la estrategia óptima es jugar con el
    oponente más debil en el segundo partido, y que el orden en que juegue
contra los otros dos no importa}.
\subsubsection{Respuesta}





\item Utilice los axiomas de probabilidad para demostrar, que para dos eventos
    \(A\) y  \(B\),
    \[P\left(\left(A\cap \overline{B}\right)\cup\left(\overline{A}\cap B\right)
    \right)=P\left(A\right)+P\left(B\right)-2P\left(A\cap B\right),\]
    que es la probabilidad de que exactamente uno de los eventos \(A\) o \(B\)
    ocurra.
\subsubsection{Respuesta}











\item Se lanzan dos dados justos de seis caras. Asuma que cada uno  de los
    {\color{NavyBlue} 36 posibles resultadoses igualmente probable}.

    \begin{enumerate}[{a) }]
        \item Determine la probabilidad de que salga un doble {\color{Magenta}
            (1 y 1, 2 y 2, etc)}.
        \item Dado que el resultado es menor o igual a 4, {\color{ForestGreen}
            determine la probabilidad condicional de que halla salido un
            doble}.
        \item Determine la probabilidad de que al menos uno de los dados sea 6.

        \item Dado que los dos dados caes en números diferentes, {\color{Melon}
            determine la probabilidad de que al menos uno de los dados haya
            caído en 6}.
    \end{enumerate}
\subsubsection{Respuestas}
\begin{figure}[ht]
    \centering
    \incfig[0.5]{dosdasos}
\end{figure}
\begin{enumerate}[{a) }]
    \item Ya que todos los valores tiene la misma posibilidad, significa que
        cualquiera de las 36 combinaciones de numeros tienen {\color{Tan}
        \(\frac{1}{36}\)} de posibilidad. Luego como hay seis situaciones donde
        se cumple la condición, se puede concluir afirmando que la posibilidad
        de sacar soble es de {\color{Mulberry} \(\frac{1}{6}\)}.
    \item Como solo hay seis posibles resultados donde la suma de ambos dados
        es menor o igual a seis, y solo dos de ellos son dobles, se tiene que
        la probabilidad para que esto suceda es de {\color{MidnightBlue}
        \(\frac{1}{3}\)}.
    \item Ya que entre las 36 situaciones posibles de esté lanzamiento once
        cumplen la condición, se puede concluir afirmando que la probabilidad
        de que al menos uno de los dados sea 6, es de {\color{YellowOrange}
        \(\frac{11}{36}\)}.
    \item Revisando el \textit{punto} anterior, se puede ver que la
        unica situación que encaja ahí pero no en este \textit{punto}, es en el
        caso de que ambos dados caigan en seis, por lo tanto se debe hacer el
        mismo cálculo pero con una opción menos; siendo esto igual
        {\color{Cyan} \(\frac{5}{18}\)}.
\end{enumerate}


\item Se tienen {\color{BlueViolet} 3 monedas}, {\color{RubineRed} una con dos
    caras}, {\color{Peach} una con dos sellos}, {\color{Periwinkle} y una con
    una cara y un sello}. Se selecciona una moneda al azar y el resultado es cara.
    ¿Cuál es la probabilidad de que el otro lado de la moneda sea sello?
    \subsubsection{Respuesta}
\begin{figure}[ht]
    \centering
    \incfig[0.5]{caracarasellosello}
\end{figure}

Si se analiza el problema, se








\item Un vuelo programado tiene probabilidades iguales a {\color{Plum} 0.93 de
    salir a tiempo} y {\color{BlueGreen} 0.90 de llegar a tiempo}. Además se
    sabe que la probabilidad de {\color{Bittersweet} salir y llegar a tiempo es
    de 0.85}.
    \begin{enumerate}[{a) }]
        \item ¿Cuál es la probabilidad de que llegue a tiempo dado que salió a
            tiempo?.
        \item Si el vuelo llegó a tiempo, ¿cuál es la probabilidad de que haya
            salido a tiempo?
    \end{enumerate}

    \subsubsection{Respuestas}
    \begin{enumerate}[{a) }]
        \item
    \end{enumerate}











\item Pepa tiene que pesentar los exámenes de cálculo y álgebra el mismo día. La probabilidad de pasar el examen de cálculo es de {\color{Aquamarine} \(\frac{3}{4}\)}, la de pasar el de álgebra es de {\color{Plum} \(\frac{1}{2}\)} y la de pasar ambos es de  {\color{Thistle} \(\frac{1}{4}\)}.
    \begin{enumerate}[{a) }]
        \item ¿Cuál es la probabilidad de que pase al menos uno de los exámenes?
        \item ¿Cuál es la probabilidad de que pase exactamente uno de los exámenes?
        \item  Si aprueba solo un examen, ¿cuál es la probabilidad de que sea
            el de álgebra?
    \end{enumerate}

\item Se tiene un grupo de personas conformado por {\color{Salmon} 40 hombres} y {\color{BrickRed} 60 mujeres}. El grupo presenta un examen y obtienen los siguientes resulatados:

\begin{center}
    \begin{tabular}{ |c|c|c|c| }
        \hline
        Resultado& \(H\) & \(M\) &\\ \hline
        Aprobado \(\left(A\right) \) &24&36&60\\ \hline
        Reprobado \(\left(R\right)\) &16&24&40\\ \hline
                        &40&60&\\ \hline
    \end{tabular}
\end{center}
Se selecciona una persona al azar:
\begin{enumerate}
    \item Si selecciona a un hombre ¿cuál es la probabilidad de que haya
        aprobado?
    \item Si selecciona alguien que aprobó, ¿cuál es la probabilidad de que sea
        una mujer?
\end{enumerate}


\item Se tienen dos eventos \(P\left(A\right)=0.5\), \(P\left(B\right)=0.3\) y \(P\left(A\cap B\right)=0.1\). Determine:
    \begin{itemize}
        \item \(P\left(A|B\right)\)
        \item \(P\left(B|A\right)\)
         \item \(P\left(A|A\cap B\right)\)
         \item \(P\left(A|A\cup B\right)\)
         \item \(P\left(A\cap B|A\cup B\right)\)
    \end{itemize}

    \subsubsection{Respuestas}

    \begin{itemize}
        \item \begin{align*}
            P\left(A|B\right)&= \frac{P\left(A\cap B\right)}{P\left(B\right)} \\
            &= \frac{0.1}{0.3} \\
            &= \frac{1}{3} \\\end{align*}
        \item  \begin{align*}
            P\left(B|A\right)&= \frac{P\left(A\cap B\right)}{P\left(A\right)} \\A
            &= \frac{0.1}{0.5} \\
            &= \frac{1}{5} \\
        \end{align*}
        \item \begin{align*}
                P\left(A|A\cap B\right)&= \frac{P\left(A\cap A\cap B\right)}{P\left(A\cap B\right)} \\
                &= \frac{P\left(A\cap B\right)}{P\left(A\cap B\right)} \\
                &= 1 \\
        \end{align*}
        \item \begin{align*}
                P\left(A|A\cup B\right)&= P\left(A\cap \left(A\cup B\right) \right)\\
                &= \frac{P\left(A\right)}{P\left(A\cup B\right)} \\
                &= \frac{P\left(A\right)}{P\left(A\right)+P\left(B\right)-P\left(A\cap B\right)} \\
                &= \frac{0.5}{0.5+0.3-0.1} =\frac{5}{7}\\
        \end{align*}
        \item \begin{align*}
                P\left(A\cap B|A\cup B\right)&= \frac{P\left(\left(A\cap
                B\right)\cap \left(A\cup B\right)  \right)}{P\left(A\cup
                B\right)}\\
                &= \frac{P\left(A\cap B\right)}{P\left(A\cup B\right)} \\
                &= \frac{0.1}{0.7}=\frac{1}{7} \\
        .\end{align*}
    \end{itemize}


\item Sean \(A\) y \(B\) eventos. Muestre que \(P\left(A\cap
    B|B\right)=P\left(A|B\right)\), suponiendo que \(P\left(B\right)>0\).

    \subsubsection{Respuesta}

    \begin{align*}
        P\left(A\cap B|B\right)&=\frac{P\left(\left(A\cap B\right) \cap B\right)}{P\left(B\right)}  \\
        &= \frac{P\left(A\cap B\right)}{P\left(B\right)} \\
        &= P\left(A|B\right) \\
    \end{align*}


\item Se tienen \(k\) cajas, cada una con \(m\) bolas blancas y \(n\) negras.
    Se selecciona al azar una bola de la primera caja y se transfiere a la
    segunda caja. Luego se selecciona al azar una bola de la segunda caja y se
    transfiere a la tercera. Se repite el ejercicio hasta que al final se
    selecciona una bola al azar de las \(k\)-ésima caja. Muestre que la
    probabilidad de que la última bola seablanca es la misma que de que al
    primera sea blanca,\textit{i.e, \(\frac{m}{\left(m+n\right) }\)}.









\item Dos jugadores se turnan para seleccionar una bola de una caja que
    inicialmente contiene {\color{red} \(m\) bolas rojas} y
    {\color{MidnightBlue} \(n\) bolas azules}. El primer jugador en extraer una
    {\color{red} bola roja gana}. Determine una formula recursiva (sobre
    {\color{MidnightBlue} \(n\)}) que permita calcular la probabilidad de que
    el jugador que empieza gane.

    \textbf{Pista:} Calcule primero \(P_0\) (probabilidad de que el primer jugador gane cuando no hay {\color{MidnightBlue} bolaz azules}), luego \(P_1,P_2,etc.\), usando el teorema de probabilidad total sobre los eventos:
    \begin{enumerate}[{(1) }]
        \item Seleccionar una bola roja en el primer intento.
        \item Seleccionar una {\color{MidnightBlue} bola azul} en el primer
            intento.
    \end{enumerate}
Recuerde que la probabilidad de que el primer jugador gane es uno menos la
probabilidad de que el segundo gane.


\end{enumerate}
\end{document}
