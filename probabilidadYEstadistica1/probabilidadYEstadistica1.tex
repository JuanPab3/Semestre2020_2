\documentclass[a4paper]{book}

\usepackage{import}
\input{./../../commands.tex}
\usepackage{cmands}

\newcommand{\asignatura}{Probabilidad y Estadística 1}
\newcommand{\autor}{Juan Pablo Sierra Useche}
\newcommand{\semestre}{Tercer}
\newcommand{\universidad}{Universidad Del Rosario}
\newcommand{\escuela}{Escuela de Ingeniería, Ciencia y tecnología}
\newcommand{\carrera}{Matemáticas Aplicadas y Ciencias de la Computación}


\begin{document}

    \titPag
    \tableofcontents

    % ====| B E G I N     O F    W O R K    S P A C E |==== %


    \begin{chapter}{Evetos y Conteo}
    \label{chap:evetos_y_conteo}

    \section{03.08.2020 $\mybar$ Conjuntos, Modelos Probabilísticos y Axiomas}
    \label{sec:conjuntos_modelos_probabilisticos_y_axiomas}

    \subsection{Repaso Tería de Conjuntos}
    \label{ssec:repaso_teria_de_conjuntos}

    \begin{description}
        \item[Conjunto] Un conjunto es una colección de objetos de cualquier
            tipo (ej: números, personas, colores, sabores, etc...), a estos
            objetos se les  conoce como los elementos del conjunto.\\
            \textbf{Ejemplos:}
            \begin{itemize}
                \item Los números naturales $\mathbb N$.
                \item Los alumnos del curso.
            \end{itemize}
        \item[Objetos] Siendo así, son los objetos aquellos que definen a los
            \textbf{conjuntos} en su totalidad.
    \end{description}
    \subsubsection{Notación de Conjuntos}
    \label{ssec:notacion_de_conjuntos}

        Si $\mathbb S$ es un conjunto y \textbf{x} un elemento de $\mathbb S$,
        escribimos $x\in \mathbb S$. Pero en el caso contrario, donde
        \textbf{x} no es un elemento de $\mathbb S$ escribimos $x\notin \mathbb
        S$.

    \begin{description}
        \item[Conjunto Vacío] Este conjunto se caracteriza por no tener ningún
            elemento dentro de si mismo, y se denota: $\emptyset$.
    \end{description}

    \subsubsection{Notación Conjunto Finito}
    \label{ssec:notacion_conjunto_finito}

        Si $ \mathbb S $ es un conjunto finito co elementos $ x_1,x_2,...,x_n
        $. Podemos denotar a S como:
        \begin{equation*}
        \label{eq:1.1}
            \mathbb S = \{x_1,x_2,...,x_n\}
        \end{equation*}

    \begin{description}
        \item[Ejemplos:] $ \mathbb S $ conjuto de resulatdos de un dado
            \begin{equation*}
            \label{eq:1.2}
                \mathbb S=\{1,2,3,...,6\}
            \end{equation*}
    \end{description}

    \subsubsection{Notación Conjunto Infinito}
    \label{ssec:notacion_conjunto_infinito}

    Si $ \mathbb S$ es un conjunto infinito enumerable con elementos $
    x_1,x_2,...$ se puede escribir a $ \mathbb S $ como:

    \begin{equation*}
    \label{eq:1.3}
        \mathbb S=\{x_1,x_2,...\}
    \end{equation*}

    \subsection{Tipos de Notación}
    \label{ssec:tipos_de_notacion}

    \begin{description}
        \item[Extención] Como los casos presentados anteriormente, esté tipo de
            notación implica en enumerar elementos demostrando el patron que
            describe el comportamiento del conjunto.

            \begin{equation*}
            \label{eq:1.4}
                \mathbb P=\{2,4,6,...,40\}
            \end{equation*}

            \begin{equation*}
            \label{eq:1.5}
                \mathbb W=\{amarillo,azul,rojo\}
            \end{equation*}

        \item[Comprención] En este caso tipo de notación no se mencionan los
            elemento, sino que se mencionan las caracteristicas que tiene cada
            elemento perteneciete a el dicho conjunto.

            \begin{equation*}
            \label{eq:1.6}
                 \mathbb W = \{x|x\ es\ un\ color\ primario\}
            \end{equation*}

            \begin{equation*}
            \label{eq:1.7}
                \mathbb P=\{x|x\ es\ un\ número\ par\ entre\ el\ 2\ y\ el\ 40\}
            \end{equation*}
    \end{description}

    \subsection{Relaciones Entre Conjuntos}
    \label{ssec:relaciones_entre_conjuntos}

    Se dice que $ \mathbb S $ es un subconjunto de $ \mathbb T $ ($ \mathbb
    S,\mathbb T $ conjuntos) es decir que $ \mathbb S $ está contenido en $
    \mathbb T $ si todo elemento de $ \mathbb S $ es también un elemento de $
    \mathbb T $.

    \subsubsection{Notación Sub Conjuntos}
    \label{ssec:notacion_sub_conjuntos}

    En el caso donde el \textbf{sub conjunto} puede ser el mismo
    \textbf{conjunto} se utiliza la siguiente notación:

    \begin{equation*}
    \label{eq:1.8}
        \mathbb S\subseteq \mathbb T
    \end{equation*}

    Pero si el \textbf{sub conjunto} no puede ser el mismo \textbf{conjunto} se
    utiliza la siguiente notación:

    \begin{equation*}
    \label{eq:1.9}
        \mathbb S\subset \mathbb T
    \end{equation*}

    En el caso de que un \textbf{conjunto} no esté contenido en otro, se
    utiliza la siguiente notación:

    \begin{equation*}
    \label{eq:1.10}
        \mathbb S\not\subseteq \mathbb T
    \end{equation*}

    \subsection{Conjunto Universal}
    \label{ssec:conjunto_universal}

    Denotamos con $ \Omega $ el \textbf{Conjunto Universal}; un conjunto
    especial que como caracteristica principal tiene a todos los elementos de
    interés en un determinado contexto.

    \begin{description}
        \item[Ejemplo] $\Omega=\mathbb C$ si estudiamos raíces de polinomios
            con coeficientes reales. (\textit{Teorema Findamental del
            Álgebra}).
    \end{description}

    \subsection{Álgebra de Conjuntos}
    \label{ssec:algebra_de_conjuntos}

    \begin{description}
        \item[Complemento] La notación pala el complemento es $\mathbb S^{c}$
            dondo nos referimos al \textbf{complemento} de $\mathbb S$. Y esté
            se puede definir de la siguiente forma:

            \begin{equation*}
            \label{eq:1.11}
                \mathbb S^{c} = \{x|x\not\in \mathbb S\ (x\in \Omega)\}
            \end{equation*}

        \item[Unión] Donde $\mathbb S$ y $\mathbb J$ son conjuntos, la notación
            para la unión entre dos conjuntos es $\mathbb S\cup \mathbb J$ e
            implica:

            \begin{equation*}
            \label{eq:1.12}
                \mathbb S\cup \mathbb J = \{x|x\in \mathbb S\ ó\ x\in \mathbb J\}
            \end{equation*}

        \item[Intersección] Sean $\mathbb S$ y $\mathbb T$ conjuntos, su
            intersección se escribe: $\mathbb S\cap \mathbb T$, y se define de
            la siguiente forma:

            \begin{equation*}
            \label{eq:1.13}
                \mathbb S \cap \mathbb T = \{x|x\in \mathbb S\ y\ x\in \mathbb T\}
            \end{equation*}

        \item[Unión entre varios (o infinitos) conjunto]
            \begin{equation*}
            \label{eq:1.14}
                \bigcup_{i=0}^{n}\mathbb S_{i} = \{x|x\in \mathbb S_{i}\ (0>i>n)\}
            \end{equation*}

        \item[Intersección entre varios (o infinitos) conjunto]
            \begin{equation*}
            \label{eq:1.14}
                \bigcap_{i=0}^{n}\mathbb S_{i} = \{x|x\in \mathbb S_{i}\ (0>i>n)\}
            \end{equation*}

        \item[Conjuntos Disyuntos] Dos conjuntos $\mathbb S\ y\ \mathbb T$ se
            dicen disyuntos o disjuntos si $\mathbb S\cap \mathbb T=\emptyset$
            lo que se generaliza al decir que $\bigcap_{i=0}^{n}\mathbb
            S_{i}=\emptyset$.

        \item[Disyunción 2 a 2] Varios conjuntos $\mathbb S_i$ se dicen
            conjuntos disyuntos 2 a 2 si $\mathbb S_i \cap \mathbb
            S_j=\emptyset$.
    \end{description}

    \begin{obs}
        El par ordenado de dos objetos $x,y$ se denota por $(x,y)$ donde
        $(x,y)\ne(y,x)$. Lo que se diferencia de conjuntos donde
        $\{x,y\}=\{y,x\}$.
    \end{obs}

    \begin{obs}
        Los diagramas de venn (representaciones graficas de conjuntos) resultan
        útiles al realizar problemas que involucran conjuntos.
    \end{obs}

    \begin{tarea}{Tarea Demostración de Lemas}{03.08.2020}{05.08.2020}
        \item Demostrar los siguientes lemas:

            \begin{enumerate}

                \item $\mathbb S\cup \mathbb T= \mathbb T\cup \mathbb S$
                    (conmutatividad)

                    \textbf{Demostración}

                    Sean $\mathbb S,\mathbb T$ conjuntos, para demostrar que
                    $\mathbb S\cup \mathbb T= \mathbb T\cup \mathbb S$ es
                    necesario demostrar las dos siguientes condiciones:

                    \begin{itemize}
                        \item Si $x\in(\mathbb S\cup \mathbb T)$ es decir $x\in
                            \mathbb S$ o $x\in \mathbb S$ o $x\in \mathbb T$,
                            luego se pude decir que $x\in(\mathbb T\cup \mathbb
                            S)$.

                        \item Si $x\in(\mathbb T\cup \mathbb S)$ es decir $x\in
                            \mathbb T$ o $x\in \mathbb S$ o $x\in \mathbb S$,
                            luego se pude decir que $x\in(\mathbb S\cup \mathbb
                            T)$.
                    \end{itemize}

                    Como ambas situaciones son verdaderas, se puede concluir
                    que la proposición es verdadera. $\blacksquare$


                \item $\mathbb S\cap(\mathbb T\cup \mathbb U)=(\mathbb S\cap
                    \mathbb T)\cup(\mathbb S\cap \mathbb U)$ (distributividad)

                    \textbf{Demostración}

                    Sean $\mathbb S, \mathbb T\ y\ \mathbb U$ conjuntos, para
                    demostrar que la proposición es verdadera hay que demostrar
                    las dos siguientes condiciones:

                    \begin{itemize}
                        \item Si $x\in\left( \mathbb S\cap\left( \mathbb
                            T\cup\mathbb U \right) \right)$ significa que
                            $x\in\mathbb S$ y también que $x\in\left( \mathbb
                            T\cup\mathbb U \right)$ es decir, que
                            incondicionalmente $x\in\mathbb S$ pero también
                            $x\in\mathbb T$ ó $x\in\mathbb U$. Por lo tanto se
                            puede decir que $x\in\left( \left( \mathbb
                            S\cap\mathbb T \right)\cup\left( \mathbb
                            S\cap\mathbb U\right) \right)$.
                        \item Por otro lado hay que asumir que $x\in\left(
                            \left( \mathbb S\cap\mathbb T\right)\cup\left(
                            \mathbb S\cap\mathbb U\right) \right)$, lo que
                            significa que ó $x\in\left( \mathbb S\cap\mathbb T
                            \right)$ ó $x\in\left( \mathbb S\cap\mathbb U
                            \right)$. A partir de lo anterior se puede asegurar
                            que $x\in\mathbb S$ y que $x\in\mathbb T$ ó
                            $x\in\mathbb U$, que es lo mismo que decir
                            $x\in\left( \mathbb S\cap\left( \mathbb
                            T\cup\mathbb U \right) \right)$
                    \end{itemize}

                    Ya que se cumplen ambas condiciones, se puede concluir que
                    la proposición es verdadera. $\blacksquare$

                \item $(\mathbb S^{c})^{c} = \mathbb S$

                \textbf{Demostración}

                Sea $\mathbb S$ un conjunto, es necesario demostrar dos
                situaciones para demostrar verdadera a la proposición:

                \begin{itemize}
                    \item El hecho de que $x\in(S^{c})^{c}$ quiere decir que
                        $x\notin\mathbb S^{c}$ y por la definición de
                        complemento, se puede asegurar que $x\in\mathbb S$.
                    \item Asumiendo que $x\in\mathbb S$, por definición se
                        puede decir que $x\notin\mathbb S^{c}$, implicando que
                        x pertenece al complemento de $\mathbb S$.
                \end{itemize}

                Ya que ambas situaciones son verdaderas se a demostrado
                verdadera a la proposición. $\blacksquare$

                \item $\mathbb S\cup \Omega = \Omega$

                \textbf{Demostración}

                Sea $\mathbb S$ un \textbf{sub conjunto} de $\Omega$, se tiene:

                \begin{itemize}
                    \item Al decir $x\in\left( \mathbb S\cup\Omega \right)$,
                        por definición de \textbf{sub conjunto} se puede
                        asegurar que es lo mismo que decir que $x\in\Omega$ ya
                        que todo x que esté en $\mathbb S$ va a estar en
                        $\Omega$.
                    \item Por otro lado al decir que $x\in\Omega$ se asegura
                        que x pertenece a la unión entre $\Omega$ y cualquiera
                        de sus \textbf{sub conjuntos}. Por lo tanto se puede
                        igualar con $\mathbb S\cup\Omega$.
                \end{itemize}

            Teniendo en cuenta de que ambas condiciones se cumplen, se ha
            demostrando que la proposición es verdadera. $\blacksquare$
            \np
                \item $\mathbb S\cup(\mathbb T\cup  \mathbb U) = (\mathbb S\cup
                    \mathbb T)\cup \mathbb U$ (asociatividad)

                \textbf{Demostración}

                Sean $\mathbb S,\mathbb T,\mathbb U$ conjuntos luego:
                \begin{equation*}
                \label{eq:1.15}
                \begin{split}
                    x\in\left( \mathbb S\cup\left( \mathbb T\cup\mathbb U
                    \right) \right) & \iff x\in\mathbb S\lor x\in\left( \mathbb
                    T\cup\mathbb U\right)\\
                    & \iff x\in\mathbb S\lor \left( x\in\mathbb T\lor x
                    \in\mathbb U \right)\\
                    & \iff x\in\mathbb S\lor x\in\mathbb T\lor x \in\mathbb U\\
                    & \iff x\in\left( \left( \mathbb S\cup \mathbb T
                    \right)\cup\mathbb U \right)\\
                \end{split}
                \end{equation*}

                Así demostrando que la proposición es verdadera. $\blacksquare$

                \item $\mathbb S\cup(\mathbb T\cap \mathbb U)=(\mathbb S\cup
                    \mathbb T)\cap(\mathbb S\cup \mathbb U)$

                \textbf{Demostración}

                Sean $\mathbb S\cup\mathbb T\cup\mathbb U$ conjuntos, entonces:
                \begin{equation*}
                \label{eq:1.16}
                \begin{split}
                    x\in\left(\mathbb S\cup\left(\mathbb T\cap \mathbb
                    U\right)\right)&\iff x\in\mathbb S\lor\left(\mathbb
                    T\cap\mathbb U\right)\\
                    &\iff x\in\mathbb S\lor\left(x\in\mathbb T\land x\in\mathbb
                    U\right)\\
                    &\iff \left(x\in\mathbb S\lor x\in\mathbb
                    T\right)\land\left(x\in\mathbb S\lor x\in\mathbb U\right)\\
                    &\iff x \in\left(\left(\mathbb S\cup\mathbb
                    T\right)\cap\left(\mathbb S\cup\mathbb U\right)\right)
                \end{split}
                \end{equation*}

                De está forma se acaba de comprobar que la proposición es
                verdadera. $\blacksquare$

                \item $\mathbb S\cap \mathbb S^{c}=\emptyset$

                \textbf{Demostración}

                Sea $\mathbb S$ un conjunto, y por contradicción suponiendo que
                $\mathbb S\cap\mathbb S^{c}\ne\emptyset$, por lo tanto $\exists
                x$ tal que $x\in\left(\mathbb S\cap\mathbb S^{c}\right)$ pero
                por definición del complemento de $\mathbb S$ ($\mathbb S^{c}$
                es todo lo que no está dentro de $\mathbb S$) significa que
                $\nexists x\ \left(\to\gets\right)$. Asi demostrando que la
                proposición es verdadera. $\blacksquare$

                \item $\mathbb S\cap \Omega=\mathbb S$

                \textbf{Demostración}

                Sea $\mathbb S$ un conjunto, entonces es necesario evealuar las
                siguientes situaciones:
                \begin{itemize}
                    \item Suponiendo que $x\in\left(\mathbb S\cap\Omega\right)$
                        como $\mathbb S\subseteq\Omega$ entonces de cualquier
                        forma $\forall x\in\mathbb S$ tambien $x\in\Omega$.
                        Pero $\nexist x$ talque $x\in\mathbb S^{c}$.

                    \item Suponiendo que $x\in\mathbb S$, y como $\mathbb
                        S\subseteq\Omega$ entonces $x\in\mathbb S\land
                        x\in\Omeg$ es decir $x\in\left(\mathbb
                        S\cap\Omega\right)$.
                \end{itemize}

                Ya que en ambos casos son verdaderos se ha demostrado que la
                proposición es verdadera. $\blacksquare$
            \end{enumerate}

        \item Demostrar las \textbf{Leyes de De Morgan}:
            \begin{enumerate}
                \item \[\Big(\bigcup_{i=1}^{n}\mathbb S_{i}\Big)^{c} = \bigcap_{i=1}^{n}\mathbb S_{i}^{c}\]

                \textbf{Demostración}

                Sea la union de varios \textbf{conjuntos} tal que
                    $\bigcup_{i=1}^{n}\mathbb S_{i}=\mathbb S_{1}\cup\mathbb
                    S_{2}\cup\mathbb S_{3}\cup...\cup\mathbb S_{n}$ ahora por
                    induccion matematica se van a revisar los siguientes casos:

                    \begin{description}
                        \item[Caso base (n=1)]\\
                        \begin{equation*}
                        \label{eq:1.17}
                        \Big(\bigcup_{i=1}^{1}\mathbb S_{i}\Big)^{c} =
                            \Big(\mathbb S_{i}\Big)^{c} = \mathbb
                            S_{1}^{c}=\bigcap_{i=1}^{1}\mathbb S_{1}^{c}
                        \end{equation*}
                        \np
                        \item[Caso inductivo]\\
                            Suponiendo que $\bigcup_{i=1}^{n}\mathbb S_{i}$ la propiedad tal que:
                            \begin{equation*}
                            \label{eq:1.18}
                            \begin{split}
                            &\Big(\bigcup_{i=1}^{n}\mathbb S_{i}\Big)^{c} =
                                \mathbb S_{1}^{c}=\bigcap_{i=1}^{n}\mathbb
                                S_{i}^{c}\\
                            &Ahora:\\
                            &\left(\mathbb S_{1}\cup\mathbb
                                S_{2}\cup...\cup\mathbb
                                S_{n}\right)^{c}\cap\mathbb S_{n+1}^{c}
                                =\left(\mathbb S_{1}^{c}\cap\mathbb
                                S_{2}^{c}\cap...\cap\mathbb
                                S_{n}^{c}\right)\cap\mathbb S_{n+1}^{c}\\
                            &\left(\mathbb S_{1}\cup\mathbb
                                S_{2}\cup...\cup\mathbb S_{n}\cup\mathbb
                                S_{n+1}\right)^{c}=\left(\mathbb
                                S_{1}^{c}\cap\mathbb
                                S_{2}^{c}\cap...\cap\mathbb
                                S_{n}^{c}\cap\mathbb S_{n+1}^{c}\right)\\
                            &\Big(\bigcup_{i=1}^{n+1}\mathbb
                                S_{i}\Big)^{c}=\bigcap_{i=1}^{n+1}\mathbb
                                S_{i}^{c}\\
                            \end{split}
                            \end{equation*}
                            Así demostrando por inducción matemática que la
                            proposición es verdadera para todo $n>1$.
                            $\blacksquare$

                    \end{description}

                \item \[\Big(\bigcap_{i=0}^{n}\mathbb S_i\Big)^{c} =
                    \bigcup_{i=0}^{n}\mathbb S_{i}^{c}\]
            \end{enumerate}
    \end{tarea}

    \section{04.08.2020 $\mybar$ Modelos Probabilisticos}
    \label{sec:modelos_probabilisticos}

    Un modelo probabilístico consiste en la traducción situaciones inciertas a
    un lenguaje matemático.

    Suponiendo que existe un experimento incierto al que se le va a asociar un
    modelo probabilistico, el cual se construye de los siguientes tres
    elementos:

    \begin{enumerate}[{1. }]
        \item El espacio muestral denotado por $\Omega$. Este espacio es el
            conjunto con todos los posibles resultados del experimento.

        \item Ley de probabilidad denotada por $P$.

        \item Un espacio de eventos denotados por $F$, donde $F\subseteq
            \varphi\left(\Omega\right)$.
    \end{enumerate}

    Para cada elemento de $F$ se le va a asociar la probabilidad con la
    notación: $F\left(A\right)$ donde $A\subseteq F$.

    Todos los dichos elementos pertencecientes a $F$ tienen siertas
    caracteristicas:

    \begin{obs}
        Los subconjuntos medibles de $\Omega$ se llaman eventos.
    \end{obs}

    \begin{itemize}
        \item Los elementos deben ser \textbf{mutuamente excluyentes}, es decir
            que si un elemento del espacio muestral es solución al
            \textbf{modelo} entonces ningun otro elemento puede ser solución
            para el modelo.
        \item Los elementos del conjunto deben ser colectivamente exhaustivo,
            lo que quiere decir que la suma de todos los elementos en el
            conjuntos representan todas las posibles soluciones al experimento.
    \end{itemize}

    \begin{obs}
        \begin{itemize}
            \item El grado de detalle al definir $\Omega$ depende del interés en el
            problema (se debe incluir todo lo nnecesario y excluir todo lo no
            necesario).
             \item $\Omega$ puede ser \textit{finito} o \textit{infinito}.
        \end{itemize}
    \end{obs}

    \subsubsection{Modelos Secuenciales}
    \label{ssec:modelos_secuenciales}

    Si el experimento es de carácter secuencial, es útil describir $\Omega$
    cuadriculas o \textbf{arboles}.


    \subsection{Axiomas de Probabilidad}
    \label{ssec:axiomas_de_probabilidad}

    Suponiendo que se conoce $\Omega$ y $F$. $P$ debe reflejar qué tan posible
    son los eventos; a un evento $A\in F$ asociamos $P\left(A\right)$.

    \begin{description}
        \item[No negatividad]  $P\left(A\right)>0$, $\forall A\in F$

        \item[Aditividad] Si $A\cap B =\emptyset$, entonces $P\left(A\cup B\right)=P\left(A\right)+P\left(B\right)$

        \item[Normalización] $P\left(\Omega\right)=1$
    \end{description}

    \subsection{Lemas}
    \label{ssec:lemas}

    \begin{enumerate}[{1. }]
        \item $P\left(\emptyset\right)=0$

        \textbf{Demostración}

        Por normalización $1-P\left(\Omega\right)=P
            \left(\Omega\cup\emptyset\right)=P
            \left(\Omega\right)+P\left(\emptyset\right)=1+P
            \left(\emptyset\right)\to P\left(\emptyset\right)=0$

        \item $P\left(A\cup B\right) = P\left(A\right)+P\left(B\right) -
            P\left(A\cap B\right)$

        \textbf{Demostración}

        \begin{equation*}
        \begin{split}
            A & = \left(A \cap B\right)\cup\left(A \cap B^{c}\right)\\
            B & = \left(A \cap B\right)\cup\left(A^{c} \cap B\right)\\
            P\left(A\right) + P\left(B\right) & = \left(A \cap B\right)+\left(A \cap B^{c}\right) + \left(A \cap B\right) + \left(A^{c} \cap B\right)\\
            P\left(A\right) + P\left(B\right) - \left(A \cap B\right) & = \left(A \cap B^{c}\right) + \left(A \cap B\right) + \left(A^{c} \cap B\right)\\
            P\left(A\right) + P\left(B\right) - \left(A \cap B\right) & = P\left(A\cup B\right)\\
        \end{split}
        \end{equation*}

        \item $P\left(A\cup B\right)\leq P\left(A\right) + P\left(B\right)$

        \textbf{Demostración}

        Tarea

        \item $P\left(A\cup B\cup C\right) =  P\left(A\right) + P\left(A^{c}\cap B\right) + P\left(A^{c}\cap B^{c}\cap C\right)$

        \textbf{Demostración}

        Tarea
    \end{enumerate}

    \subsection{Modelos Pobabilísticos Discretos y Continuos}
    \label{ssec:modelos_pobabilisticos_discretos_y_continuos}

    \begin{description}
        \item[Modelos Discretos] Cuando $\Omega$ es finito o enumerable.
            \begin{obs}
            Si $\Omega$ es finito o enumerable, la función de probabilídad
                queda totalmente definida por los \textit{singletons}.
            \end{obs}
           \subsection{Ley de probabilidad discreta uniforme}
           \label{ssec:ley_de_probabilidad_discreta_uniforme}

            Si $\Omega$ tiene $n$ elementos, todos igualmente posibles (todos
            los singletons tiene la misma probabilidad). Entonces $\forall A\in
            F,A\subseteq\Omega$, $P\left(A\right)=\frac{|A|}{|\Omega|}$

        \item[Modelos Continuos] Cuando $\Omega$ no es enumerable
            $Ej:\left(\Omega=\mathbb R\right)$.
            \begin{obs}
            En este caso los singletons normalmente no son suficientes para
                definir toda la ley de probabilidad.
            \end{obs}
    \end{description}

    \section{05.08.2020 $\mybar$ Probabilidad Condicional}
    \label{sec:probabilidad_condicional}

    Utilizamos probabilidad condicional cuando queremos sobre posibles
    resultados del experimento, teniendo alguna información parcial.

    \begin{ej}
    \begin{itemize}
    \item Tiro un dado y se que salió par. Queremos calcular la prob. de tener 4.

    \item Una persona sale negativaen un examen medico y queremos sabere la
        prob. de que si haya enfermedad (falso negativo).
    \end{itemize}
    \end{ej}

    Suponiendo que hay un experimento incierto. Sabemos que ocurrió un evento
    $B$. Queremos determinar la prob. de que ocurrio otro evento $A$.

    \subsubsection{Objetivo}
    \label{ssec:objetivo}

    Construir una nueva función de prob. que incorpore información que se tiene (acurrio $B$).

    \begin{obs}
        \begin{enumerate}[{1. }]

            \item La nueva función de prob. debe satisfacer las axiomas.

            \item La nueva función de prob. debe heredar la estructura de la
                función de prob. original.

        \end{enumerate}
    \end{obs}

    \subsection{Definición Formal}
    \label{ssec:definicion_formal}

    Sea $b$ un evento tal que $P\left(B\right)>0$. Se define la prob. condicional de un evento $A$ (condicional al evento $B$) como:
    \begin{equation*}
        P\left(A|B\right)=\frac{P\left(A\cap B\right)}{P\left(B\right)}
    \end{equation*}

    \begin{obs}
        Si $P\left(B\right)=0$, la probabilidad condicional a $B$ no está definida.
    \end{obs}

    \subsubsection{Teorema}
    \label{ssec:teorema}

    Sea $P$ una función de prob. sea $B$ evento tal que $P\left(B\right)>0$. La
    probabilidad condicinal en $B$ es una ley de probabilidad. (Es decir qu
    cumple los axiomas de probabilidad).

    \subsubsection{Demostración}

    \begin{itemize}
        \item $P\left(\Omega|B\right) =\frac{P\left(\Omega\cap
            B\right)}{P\left(B\right)} =\frac{P\left(B\right)}{P\left(B\right)}
            = 1$

        \item $A_1,A_2$ eventos disyuntos luego:
            \begin{equation*}
            \begin{split}
                P\left(A_1\cup A_2|B\right) & = \frac{P\left(\left(A_1\cup
                A_2\right)\cap B\right)}{P\left(B\right)}\\
                \frac{P\left(A_1\cap B\right)\cup\left(A_2\cap B\right)}{P\left(B\right)} & = \frac{P\left(A_1\cap B\right)}{P\left(B\right)} + \frac{P\left(A_2\cap B\right)}{P\left(B\right)}\\
                & = P\left(A_1|B\right)+P\left(A_2|B\right)
            \end{split}
            \end{equation*}
        \item Sea $A$ un evento.
            \begin{equation*}
               P\left(A|B\right) = \frac{P\left(A\cap B\right)}{P\left(B\right)}\leq 0
            \end{equation*}
    \end{itemize}

    Es decir $P\left(°|B\right)$ es una ley de probabilidad. $\blacksquare$


    \subsubsection{Corolario}
    \label{ssec:corolario}

    Las propiedades de las funciones de probabilidad tambien se cumplen para la
    probabilidad condicional.

    \begin{ej}
        \begin{itemize}
            \item $A_1,A_2$ eventos: $A_1\leq A_2\to P\left(A_1|B\right)\leq
                P\left(A_2|B\right)$

            \item $P\left(A_1\cup A_2|B\right) = P\left(A_1|B\right) +
                P\left(A_1|B\right) - P\left(A_1\cap A_2\right)$
        \end{itemize}

    \end{ej}

















































    \end{chapter}

    % ====| E N D    O F    W O R K    S P A C E |==== %

    \printbibliography
\end{document}
