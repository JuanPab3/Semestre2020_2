\documentclass[a4paper,dvipsnames]{book}

% ====| P A C K A G E S |==== %
\usepackage{import}
\usepackage{nicematrix}
%\usepackage[backend=biber, style=authoryear-icomp]{biblatex}    %Bibliography stuff
%\input{./../../commands.tex}
\usepackage[margin=1in,includefoot]{geometry}   %Margins
\usepackage[utf8]{inputenc} %Language stuff
%\usepackage [latin1]{inputenc}  %Spanish symbols
\usepackage[spanish]{babel} %Sets document language to spanish
\usepackage{tcolorbox}  %Frame boxes
\usepackage{enumerate}  %Lists options
\usepackage{mfirstuc}   %I use it to capitalize words
\usepackage{graphicx}   %Use Images
\usepackage{listings}   %For displaying code
\usepackage{titlesec}   %Costume titles/sections/...
\usepackage{hyperref}   %Linking options
\usepackage{multicol}	%Use column
\usepackage{amsmath}    %Display equations options
\usepackage{amssymb}    %More symbols
\usepackage{titling}    %Use title variables in other places
\usepackage{xcolor}     %To manage colors
\usepackage{transparent}%For figures
\usepackage{pdfpages}   %For figures
% =========================== %

% ====| P A C K A G E S    S E T T I N G S |==== %
% \addbibresource{/media/jpi/Data/01_Education/04_bibiografia/bibliography.bib} %<--- Bibliography path
% ============================================== %

% ====| P E R S O N A L    C O M M A N D S    &    E N V I R O N M E N T S|==== %

% - New figure
\newcommand{\incfig}[2][1]{%
    \def\svgwidth{#1\columnwidth}
    \import{./figures/}{#2.pdf_tex}
}

\pdfsuppresswarningpagegroup=1

% - New page
\newcommand{\np}{\null\newpage}

% - new homework
\newenvironment{tarea}[3]
    {
        \null\newpage
        \begin{tcolorbox}
            \textbf{\asignatura\ -\ \autor}
            \subsection{\capitalisewords{#1}}
            \label{ssec:#1}
            \begin{flushright}
            \textbf{Desde:}  #2 \
            \textbf{Hasta:}  #3 \
            \end{flushright}
        \end{tcolorbox}

    \begin{enumerate}[{Ejercicio} 1.]
    }
    {
        \end{enumerate}
        \np
    }

% - new observation
\newenvironment{obs}
    {
        \begin{flushleft}
       \textbf{Observación}\
        \line(1,0){200} \
        \end{flushleft}
    }
    {
        \begin{flushright}
        \line(1,0){200}
        \end{flushright}
    }
% ============================================================================= %


\begin{document}

\section{17.08.2020 - Independencia}
\label{sec:independencia}

Cuando se introdujo la probabilidad condicional \(P\left(A|B\right)\), se hizo
para para capturar la información parcial que \(B\) entrga con respecto de
\(A\). Una sotiación interesante (e importante) sucede cuando el hecho de que
\(B\) haya sucedido no altera la probabilidad de que \(A\) suceda, tal que:
\[P\left(A|B\right)=P\left(A\right)\]

Cuando esté tipo de situación sucede, se dice que {\color{MidnightBlue} \(A\)
es \textbf{independiente} de \(B\)}. En esté caso por definición de \(P\left(A|B\right)=\frac{P\left(A\cap B\right)}{P\left(B\right)}\) es equivalente a:
\[ P\left(A\cap B\right)=P\left(A\right)P\left(B\right) .\]
Esta ultima relación se adopta como la definición de \textbf{independencia}
pues esta se puede usar en el caso tal donde \(P\left(B\right)=0\), algo que
con la probabilidad condicional no se podría a causa de que
\(P\left(A|B\right)\) quedaría indefinido.

\subsubsection{Simetría}

La simetría de está relación tambien indica que la \textbf{independencia} es una propiedad simetrica; es decir, si \(A\) es independiente de \(B\), de igual manera \(B\) es independiente de \(A\). Por lo tanto cuando hay una relación de independencia se describe como un par de eventos independientes,ej: {\color{Salmon} \(A\) y \(B\) son eventos \textbf{independientes}}.

\begin{obs}
    Aunque en los entornos abituales del anánlisis y del pensamiento la
    \textbf{independencia} es bastante inintuitiva, en la visualización de
    \(\Omega \) suele ser algo confusa; ya que lo mas común es que se vea a una
    relación independiento como un par de conjuntos \textit{disyuntos}, pero en
    realidad es todo lo contrario. Es decir, si se tienen dos conjuntos
    disyuntos \(A\) y \(B\) donde \(P\left(A\right)>0 \text{ y
    }P\left(B\right)>0\) nunca serán independientes ya que su intersección
    \(A\cap B=0\). {\color{BrickRed} El ejemplo más obvio es el de una
    probabilidad \(A\) con respecto a su complemento \(A^{c}\), estó pues si lo
    que sucede esta en uno de los dos conjuntos, es obvio que no está en el otro}.
    Lo que contradice la definición de \textbf{independencia}.
\end{obs}

\subsection{Ejemplo}

Condiferando un experimento donde se lancen {\color{orange} dos dados de cuatro
caras} en forma simultania, se tendrian 16 posibles resultados con la misma
probabilidad de \(\frac{1}{16}\).
\begin{enumerate}[{(a) }]
    \item Son los eventos \(A_{i}\) y \(B_{j}\) independientes? tal que:
        \[
        A_{i}=\left\{\text{Primer lanzamiento resulata en }i\right\},
        B_{j}=\left\{\text{Segundo lanzamiento resulata en }j\right\}
        .\]
       Se tiene que:
       \[
       P\left(A_{i}\cap B_{j}\right)= P\left(\text{el resultado de los dos lanzamientos es }\left(i,j\right)\right)=\frac{1}{16}
       ,\]
       \[
       P\left(A_{i}\right)= \frac{\text{número de elementos en }A_{i}}{\text{cantidad total de elementos}}=\frac{4}{16}
       ,\]
       \[
       P\left(B_{j}\right)= \frac{\text{número de elementos en }B_{j}}{\text{cantidad total de elementos}}=\frac{4}{16}
       .\]
Con lo anterior podemos observar que \(P\left(A_{i}\cap A_{j}\right)=P\left(A_{i}\right)P\left(B_{j}\right)\), confirmando que \(A_{i}\) y \(B_{j}\) son independietes.

    \item Siendo \(A\text{ y }B\) los eventos definidos de forma:
        \[ A=\left\{\text{el primer tiro es 1}\right\},B=\left\{ \text{la suma
        de los dos tiros es 5}\right\} \]

    son \(A \text{ y }B\) independientes? Se tiene:
    \[ P\left(A\cap B\right)=P\left(\text{el resultado de los dos lanzamientos
    está entre 1 y 4}\right)=\frac{1}{16} ,\]
    tambien:
    \[
    P\left(A\right)=\frac{\text{número de elementos en \(A\)}}{\text{cantidad total de elementos}}=\frac{4}{16},\]
    El evento \(B\) se puede formar con los resultaddos (1,4), (2,3), (3,2) y
    (4,5). Ahora:
    \[ P\left(B\right)=\frac{\text{numero de elmentos en \(B\)}}{\text{cantidad
    total de elementos}}=\frac{4}{16} .\]
    Teniendo ahora que  \(P\left(A\cap
    B\right)=P\left(A\right)P\left(B\right)\), se puede concluir de que \(A
    \text{ y  }B\) son eventos independientes.

    \item Los eventos:
        \[ A=\left\{\text{el resultado máximo de dos lanzamientos es 2}\right\},\]
        \[B=\left\{\text{el resultado mínimo de los dos lanzamientos es 2}\right\}.\]

    Intuitivamente se diría que estos dos eventos no son independientes, pues
    el valor mínimo entre dos lanzamientos comparte información con el máximo.
    Por ejemplo, si el {\color{Plum} mínimo es 2}, {\color{BlueViolet} el
    máximo no puede ser 1}. Pero para salír de las dudas lo mejor es demostrar
    la relacción. Por lo tanto se tiene:
    \[ P\left(A\cap B\right)=P\left(\text{el resultado de dos lanzamientos es
    }\left(2,2\right)\right)=\frac{1}{16} ,\]
    además:
    \[ P\left(A\right)=\frac{\text{número de elementos en }A}{\text{cantidad
    total de elementos}}=\frac{3}{16},\]
    \[ P\left(B\right)=\frac{\text{número de elementos en }B}{\text{cantidad
    total de elementos}}=\frac{5}{16} ,\]
    Pero en esté coso  \(P\left(A\right)P\left(B\right)=\frac{15}{256}\),
    confirmando la sospecha de que \(A \text{ y }B\) no son eventos
    independientes.
\end{enumerate}
\begin{obs}
    Se puede verificar que si \(A \text{ y }B\) son independientes, la mísma
    propiedad se cumple entre \(A \text{ y }B^{c}\).\footnote{Revisar los
    problemas al final del capitulo.}
\end{obs}

\subsection{Independencia Condicional}
\label{ssec:independencia_condicional}

Apartir de la conclusión previa donde se extrajo el concepto de la
\textbf{probabilidad condicional}  desde el razonamiento de leyes legitimas de
la probabilidad, que indicaba que esté tipo de probabilidad también sería
tratada como una ley legitima de la probabilidad. {\color{SeaGreen} Se puede
hacer mencón de la independencia de varios eventos con respecto a esta ley de
la probabilidad condicional}. En particular, teniendo un evento \(C\), los
eventos \(A \text{ y }B\) son llamados {\color{teal} \textbf{condicionalmente
independientes}} si:

\[ P\left(A\cap B|C\right)=P\left(A|C\right)P\left(B|C\right) \]

\subsubsection{Propiedad anexa}
Par deribar un caracterización alternativa de la condiciónal independiente, se utiliza la definición de \textit{probabilidad condicional} y la regla de la multiplicación para obtener:

\begin{align*}
    P\left(A\cap B|C\right)&=\frac{P\left(A\cap B\cap C\right)}{P\left(C\right)}\\
             &=\frac{P\left(A|B\cap C\right)P\left(B|C\right)P\left(C\right)}{P\left(C\right)}\\
             &=P\left(A|B\cap C\right)P\left(B|C\right),
\end{align*}
Teniendo en cuenta la igualdad principal de la independencia condicional:
\[ P\left(A|C\right)P\left(B|C\right)=P\left(A|B\cap C\right)P\left(B|C\right)\]
Y si se divide por \(P\left(B|C\right)\) en ambos lados queda:
\[ P\left(A|C\right)=P\left(A|B\cap C\right) .\]
    Es decir que si \(P\left(B\cap C\right)>0,\) donde \(A,B\) son eventos
    condicionalmente independientes, no importa si despues de que ocurra \(C\)
    ocurra \(B\) con respecto de \(A\).

\begin{obs}
    ¿Independencia implica independencia condicional?
    \begin{center}
        {\color{OrangeRed} \textbf{N O P}}
    \end{center}
\end{obs}

\subsection{Ejemplo}

Considere dos lanzamientos de ponedas (justos), en donde los cuatro distintos posibles resultados son igualmente posibles; Ahora:
\begin{align*}
    H_1&=\left\{\text{la primera lanzada cae cara}\right\},\\
    H_2&=\left\{\text{la segunda lanzada cae cara}\right\},\\
    D &=\left\{\text{ambos lanzamientos entregan resultados distintos}\right\}.
\end{align*}
Los eventos \(H_1\) y \(H_2\) son independientes, pero:
\[P\left(H_1|D\right)=\frac{1}{2},\hspace{1cm}
P\left(H_2|D\right)=\frac{1}{2},\hspace{1cm}
P\left(H_1\cap H_2|D\right)=0\]

en esté caso se tiene que \(P\left(H_1\cap H_2|D\right)\neq
P\left(H_1|D\right)P\left(H_2|D\right)\), por lo tanto \(H_1\) y \(H_2\)
{\color{RedViolet} no son condicionalmente independientes}.

\begin{obs}
Con esté ejemplo se puede generalizar el hecho de que si teniendo a los eventos
independientes \(A\) y \(B\) y además al evento \(C\) tal que:
\(P\left(C\right)>0\), \(P\left(A|C\right)>0\) y \(P\left(B|C\right)>0\), pero
que \(A\cap B\cap C=\emptyset\). Entonces \(A\) y \(B\) no pueden ser
condicionalmente independientes dado \(C\), pued \(P\left(A\cap B|C\right)=0\)
mientras que \(P\left(A|C\right)P\left(B|C\right)>0\).
\end{obs}

\subsection{Independencia de varios eventos}
\label{ssec:independencia_de_varios_eventos}

Se dice que los eventos \(A_1,A_2,\ldots,A_{n}\) son independientes si:
\[P\left(\bigcap_{i \in S} A_{i}\right)=\prod_{i\in S}P\left(A_{i}\right),
\text{para todo subset }S \text{ en }\left\{1,2,\ldots,n\right\} .\]








































\end{document}
