\documentclass[a4paper,dvipsnames]{book}

% ====| P A C K A G E S |==== %
\usepackage{import}
\usepackage{nicematrix}
%\usepackage[backend=biber, style=authoryear-icomp]{biblatex}    %Bibliography stuff
%\ProvidesPackage{cmands}
% ====| P A C K A G E S |==== %
\usepackage{import}
%\usepackage[backend=biber, style=authoryear-icomp]{biblatex}    %Bibliography stuff
%\ProvidesPackage{cmands}
% ====| P A C K A G E S |==== %
\usepackage{import}
%\usepackage[backend=biber, style=authoryear-icomp]{biblatex}    %Bibliography stuff
%\ProvidesPackage{cmands}
% ====| P A C K A G E S |==== %
\usepackage{import}
%\usepackage[backend=biber, style=authoryear-icomp]{biblatex}    %Bibliography stuff
%\input{./../../commands.tex}
\usepackage[margin=1in,includefoot]{geometry}   %Margins
\usepackage[utf8]{inputenc} %Language stuff
\usepackage [latin1]{inputenc}  %Spanish symbols
\usepackage[spanish]{babel} %Sets document language to spanish
\usepackage{tcolorbox}  %Frame boxes
\usepackage{enumerate}  %Lists options
\usepackage{mfirstuc}   %I use it to capitalize words
\usepackage{graphicx}   %Use Images
\usepackage{listings}   %For displaying code
\usepackage{titlesec}   %Costume titles/sections/...
\usepackage{hyperref}   %Linking options
\usepackage{multicol}	%Use column
\usepackage{amsmath}    %Display equations options
\usepackage{amssymb}    %More symbols
\usepackage{titling}    %Use title variables in other places
\usepackage{xcolor}     %To manage colors
\usepackage{transparent}%For figures
\usepackage{pdfpages}   %For figures
% =========================== %

% ====| P A C K A G E S    S E T T I N G S |==== %
% \addbibresource{/media/jpi/Data/01_Education/04_bibiografia/bibliography.bib} %<--- Bibliography path
\setlength{\columnsep}{1cm}
\hypersetup{
    colorlinks,
    citecolor=black,
    filecolor=black,
    linkcolor=black,
    urlcolor=black
}
% ============================================== %

% ====| P E R S O N A L    C O M M A N D S    &    E N V I R O N M E N T S|==== %

% - New figure
\newcommand{\incfig}[2][1]{%
    \def\svgwidth{#1\columnwidth}
    \import{./figures/}{#2.pdf_tex}
}

\pdfsuppresswarningpagegroup=1

% - New page
\newcommand{\np}{\null\newpage}

% - vertical thic line
\newcommand\mybar{\kern1pt\rule[-\dp\strutbox]{.8pt}{\baselineskip}\kern1pt}

% - new homework
\newenvironment{tarea}[3]
    {
        \null\newpage
        \begin{tcolorbox}
            \textbf{\asignatura\ -\ \autor}
            \subsection{\capitalisewords{#1}}
            \label{ssec:#1}
            \begin{flushright}
            \textbf{Desde:}  #2 \\
            \textbf{Hasta:}  #3 \\
            \end{flushright}
        \end{tcolorbox}

    \begin{enumerate}[{Ejercicio} 1.]
    }
    {
        \end{enumerate}
        \np
    }



% ============================================================================= %


% ====| H E A T H E R S    S E T T I N G S |==== %

\titleformat{\chapter}[display]
    {\normalfont\huge\bfseries\raggedleft}{\chaptertitlename\ \thechapter}
    {20pt}{\Huge}
\titleformat{\section}[display]
    {\Large\bfseries}{}
    {0em}{}[\titlerule]


\newcommand{\titPag}{
    \begin{titlepage}
        \begin{flushright}
            \textsc{\large {\semestre\ Semestre}}\\
            \line(1,0){450} \\
            [0.635cm]
            \huge{\bfseries \asignatura} \\
            [0.2cm]
            \line(1,0){350} \\
            \LARGE{\bfseries \autor} \\
            [16.25cm]
        \end{flushright}
        \begin{flushright}
        \textsc{
            \universidad \\
            [0.1cm]
            \escuela \\
            [0.1cm]
            \carrera
        }
        \end{flushright}
    \end{titlepage}
}
% ============================================== %

% ====| C O D E    I N    F I L E S    S E T T I N G S |==== %
\definecolor{codegreen}{rgb}{0,0.6,0}
\definecolor{codegray}{rgb}{0.5,0.5,0.5}
\definecolor{codepurple}{rgb}{0.58,0,0.82}
\definecolor{backcolour}{rgb}{0.95,0.95,0.92}

\lstdefinestyle{mystyle}{
    backgroundcolor=\color{backcolour},
    commentstyle=\color{codegreen},
    keywordstyle=\color{magenta},
    numberstyle=\tiny\color{codegray},
    stringstyle=\color{codepurple},
    basicstyle=\ttfamily\footnotesize,
    breakatwhitespace=false,
    breaklines=true,
    captionpos=b,
    keepspaces=true,
    numbers=left,
    numbersep=5pt,
    showspaces=false,
    showstringspaces=false,
    showtabs=false,
    tabsize=2
}

\lstset{style=mystyle}
% ========================================================== %

\usepackage[margin=1in,includefoot]{geometry}   %Margins
\usepackage[utf8]{inputenc} %Language stuff
\usepackage [latin1]{inputenc}  %Spanish symbols
\usepackage[spanish]{babel} %Sets document language to spanish
\usepackage{tcolorbox}  %Frame boxes
\usepackage{enumerate}  %Lists options
\usepackage{mfirstuc}   %I use it to capitalize words
\usepackage{graphicx}   %Use Images
\usepackage{listings}   %For displaying code
\usepackage{titlesec}   %Costume titles/sections/...
\usepackage{hyperref}   %Linking options
\usepackage{multicol}	%Use column
\usepackage{amsmath}    %Display equations options
\usepackage{amssymb}    %More symbols
\usepackage{titling}    %Use title variables in other places
\usepackage{xcolor}     %To manage colors
\usepackage{transparent}%For figures
\usepackage{pdfpages}   %For figures
% =========================== %

% ====| P A C K A G E S    S E T T I N G S |==== %
% \addbibresource{/media/jpi/Data/01_Education/04_bibiografia/bibliography.bib} %<--- Bibliography path
\setlength{\columnsep}{1cm}
\hypersetup{
    colorlinks,
    citecolor=black,
    filecolor=black,
    linkcolor=black,
    urlcolor=black
}
% ============================================== %

% ====| P E R S O N A L    C O M M A N D S    &    E N V I R O N M E N T S|==== %

% - New figure
\newcommand{\incfig}[2][1]{%
    \def\svgwidth{#1\columnwidth}
    \import{./figures/}{#2.pdf_tex}
}

\pdfsuppresswarningpagegroup=1

% - New page
\newcommand{\np}{\null\newpage}

% - vertical thic line
\newcommand\mybar{\kern1pt\rule[-\dp\strutbox]{.8pt}{\baselineskip}\kern1pt}

% - new homework
\newenvironment{tarea}[3]
    {
        \null\newpage
        \begin{tcolorbox}
            \textbf{\asignatura\ -\ \autor}
            \subsection{\capitalisewords{#1}}
            \label{ssec:#1}
            \begin{flushright}
            \textbf{Desde:}  #2 \\
            \textbf{Hasta:}  #3 \\
            \end{flushright}
        \end{tcolorbox}

    \begin{enumerate}[{Ejercicio} 1.]
    }
    {
        \end{enumerate}
        \np
    }



% ============================================================================= %


% ====| H E A T H E R S    S E T T I N G S |==== %

\titleformat{\chapter}[display]
    {\normalfont\huge\bfseries\raggedleft}{\chaptertitlename\ \thechapter}
    {20pt}{\Huge}
\titleformat{\section}[display]
    {\Large\bfseries}{}
    {0em}{}[\titlerule]


\newcommand{\titPag}{
    \begin{titlepage}
        \begin{flushright}
            \textsc{\large {\semestre\ Semestre}}\\
            \line(1,0){450} \\
            [0.635cm]
            \huge{\bfseries \asignatura} \\
            [0.2cm]
            \line(1,0){350} \\
            \LARGE{\bfseries \autor} \\
            [16.25cm]
        \end{flushright}
        \begin{flushright}
        \textsc{
            \universidad \\
            [0.1cm]
            \escuela \\
            [0.1cm]
            \carrera
        }
        \end{flushright}
    \end{titlepage}
}
% ============================================== %

% ====| C O D E    I N    F I L E S    S E T T I N G S |==== %
\definecolor{codegreen}{rgb}{0,0.6,0}
\definecolor{codegray}{rgb}{0.5,0.5,0.5}
\definecolor{codepurple}{rgb}{0.58,0,0.82}
\definecolor{backcolour}{rgb}{0.95,0.95,0.92}

\lstdefinestyle{mystyle}{
    backgroundcolor=\color{backcolour},
    commentstyle=\color{codegreen},
    keywordstyle=\color{magenta},
    numberstyle=\tiny\color{codegray},
    stringstyle=\color{codepurple},
    basicstyle=\ttfamily\footnotesize,
    breakatwhitespace=false,
    breaklines=true,
    captionpos=b,
    keepspaces=true,
    numbers=left,
    numbersep=5pt,
    showspaces=false,
    showstringspaces=false,
    showtabs=false,
    tabsize=2
}

\lstset{style=mystyle}
% ========================================================== %

\usepackage[margin=1in,includefoot]{geometry}   %Margins
\usepackage[utf8]{inputenc} %Language stuff
\usepackage [latin1]{inputenc}  %Spanish symbols
\usepackage[spanish]{babel} %Sets document language to spanish
\usepackage{tcolorbox}  %Frame boxes
\usepackage{enumerate}  %Lists options
\usepackage{mfirstuc}   %I use it to capitalize words
\usepackage{graphicx}   %Use Images
\usepackage{listings}   %For displaying code
\usepackage{titlesec}   %Costume titles/sections/...
\usepackage{hyperref}   %Linking options
\usepackage{multicol}	%Use column
\usepackage{amsmath}    %Display equations options
\usepackage{amssymb}    %More symbols
\usepackage{titling}    %Use title variables in other places
\usepackage{xcolor}     %To manage colors
\usepackage{transparent}%For figures
\usepackage{pdfpages}   %For figures
% =========================== %

% ====| P A C K A G E S    S E T T I N G S |==== %
% \addbibresource{/media/jpi/Data/01_Education/04_bibiografia/bibliography.bib} %<--- Bibliography path
\setlength{\columnsep}{1cm}
\hypersetup{
    colorlinks,
    citecolor=black,
    filecolor=black,
    linkcolor=black,
    urlcolor=black
}
% ============================================== %

% ====| P E R S O N A L    C O M M A N D S    &    E N V I R O N M E N T S|==== %

% - New figure
\newcommand{\incfig}[2][1]{%
    \def\svgwidth{#1\columnwidth}
    \import{./figures/}{#2.pdf_tex}
}

\pdfsuppresswarningpagegroup=1

% - New page
\newcommand{\np}{\null\newpage}

% - vertical thic line
\newcommand\mybar{\kern1pt\rule[-\dp\strutbox]{.8pt}{\baselineskip}\kern1pt}

% - new homework
\newenvironment{tarea}[3]
    {
        \null\newpage
        \begin{tcolorbox}
            \textbf{\asignatura\ -\ \autor}
            \subsection{\capitalisewords{#1}}
            \label{ssec:#1}
            \begin{flushright}
            \textbf{Desde:}  #2 \\
            \textbf{Hasta:}  #3 \\
            \end{flushright}
        \end{tcolorbox}

    \begin{enumerate}[{Ejercicio} 1.]
    }
    {
        \end{enumerate}
        \np
    }



% ============================================================================= %


% ====| H E A T H E R S    S E T T I N G S |==== %

\titleformat{\chapter}[display]
    {\normalfont\huge\bfseries\raggedleft}{\chaptertitlename\ \thechapter}
    {20pt}{\Huge}
\titleformat{\section}[display]
    {\Large\bfseries}{}
    {0em}{}[\titlerule]


\newcommand{\titPag}{
    \begin{titlepage}
        \begin{flushright}
            \textsc{\large {\semestre\ Semestre}}\\
            \line(1,0){450} \\
            [0.635cm]
            \huge{\bfseries \asignatura} \\
            [0.2cm]
            \line(1,0){350} \\
            \LARGE{\bfseries \autor} \\
            [16.25cm]
        \end{flushright}
        \begin{flushright}
        \textsc{
            \universidad \\
            [0.1cm]
            \escuela \\
            [0.1cm]
            \carrera
        }
        \end{flushright}
    \end{titlepage}
}
% ============================================== %

% ====| C O D E    I N    F I L E S    S E T T I N G S |==== %
\definecolor{codegreen}{rgb}{0,0.6,0}
\definecolor{codegray}{rgb}{0.5,0.5,0.5}
\definecolor{codepurple}{rgb}{0.58,0,0.82}
\definecolor{backcolour}{rgb}{0.95,0.95,0.92}

\lstdefinestyle{mystyle}{
    backgroundcolor=\color{backcolour},
    commentstyle=\color{codegreen},
    keywordstyle=\color{magenta},
    numberstyle=\tiny\color{codegray},
    stringstyle=\color{codepurple},
    basicstyle=\ttfamily\footnotesize,
    breakatwhitespace=false,
    breaklines=true,
    captionpos=b,
    keepspaces=true,
    numbers=left,
    numbersep=5pt,
    showspaces=false,
    showstringspaces=false,
    showtabs=false,
    tabsize=2
}

\lstset{style=mystyle}
% ========================================================== %

\usepackage[margin=1in,includefoot]{geometry}   %Margins
\usepackage[utf8]{inputenc} %Language stuff
%\usepackage [latin1]{inputenc}  %Spanish symbols
\usepackage[spanish]{babel} %Sets document language to spanish
\usepackage{tcolorbox}  %Frame boxes
\usepackage{enumerate}  %Lists options
\usepackage{mfirstuc}   %I use it to capitalize words
\usepackage{graphicx}   %Use Images
\usepackage{listings}   %For displaying code
\usepackage{titlesec}   %Costume titles/sections/...
\usepackage{hyperref}   %Linking options
\usepackage{multicol}	%Use column
\usepackage{amsmath}    %Display equations options
\usepackage{amssymb}    %More symbols
\usepackage{titling}    %Use title variables in other places
\usepackage{xcolor}     %To manage colors
\usepackage{transparent}%For figures
\usepackage{pdfpages}   %For figures
% =========================== %

% ====| P A C K A G E S    S E T T I N G S |==== %
% \addbibresource{/media/jpi/Data/01_Education/04_bibiografia/bibliography.bib} %<--- Bibliography path
% ============================================== %

% ====| P E R S O N A L    C O M M A N D S    &    E N V I R O N M E N T S|==== %

% - New figure
\newcommand{\incfig}[2][1]{%
    \def\svgwidth{#1\columnwidth}
    \import{./figures/}{#2.pdf_tex}
}

\pdfsuppresswarningpagegroup=1

% - New page
\newcommand{\np}{\null\newpage}

% - new homework
\newenvironment{tarea}[3]
    {
        \null\newpage
        \begin{tcolorbox}
            \textbf{\asignatura\ -\ \autor}
            \subsection{\capitalisewords{#1}}

            \begin{flushright}
            \textbf{Desde:}  #2 \
            \textbf{Hasta:}  #3 \
            \end{flushright}
        \end{tcolorbox}

    \begin{enumerate}[{Ejercicio} 1.]
    }
    {
        \end{enumerate}
        \np
    }

% - new observation
\newenvironment{obs}
    {
        \begin{flushleft}
       \textbf{Observación}\
        \line(1,0){200} \
        \end{flushleft}
    }
    {
        \begin{flushright}
        \line(1,0){200}
        \end{flushright}
    }
% ============================================================================= %

\begin{document}

    \section{03.08.2020 - Conjuntos, Modelos Probabilísticos y Axiomas}


    \subsection{Repaso Tería de Conjuntos}


    \begin{description}
        \item[Conjunto] Un conjunto es una colección de objetos de cualquier
            tipo (ej: números, personas, colores, sabores, etc...), a estos
            objetos se les  conoce como los elementos del conjunto.\\
            \textbf{Ejemplos:}
            \begin{itemize}
                \item Los números naturales $\mathbb N$.
                \item Los alumnos del curso.
            \end{itemize}
        \item[Objetos] Siendo así, son los objetos aquellos que definen a los
            \textbf{conjuntos} en su totalidad.
    \end{description}
    \subsubsection{Notación de Conjuntos}


        Si $\mathbb S$ es un conjunto y \textbf{x} un elemento de $\mathbb S$,
        escribimos $x\in \mathbb S$. Pero en el caso contrario, donde
        \textbf{x} no es un elemento de $\mathbb S$ escribimos $x\notin \mathbb
        S$.

    \begin{description}
        \item[Conjunto Vacío] Este conjunto se caracteriza por no tener ningún
            elemento dentro de si mismo, y se denota: $\emptyset$.
    \end{description}

    \subsubsection{Notación Conjunto Finito}


        Si $ \mathbb S $ es un conjunto finito co elementos $ x_1,x_2,...,x_n
        $. Podemos denotar a S como:
        \begin{equation*}

            \mathbb S = \{x_1,x_2,...,x_n\}
        \end{equation*}

    \begin{description}
        \item[Ejemplos:] $ \mathbb S $ conjuto de resulatdos de un dado
            \begin{equation*}

                \mathbb S=\{1,2,3,...,6\}
            \end{equation*}
    \end{description}

    \subsubsection{Notación Conjunto Infinito}


    Si $ \mathbb S$ es un conjunto infinito enumerable con elementos $
    x_1,x_2,...$ se puede escribir a $ \mathbb S $ como:

    \begin{equation*}

        \mathbb S=\{x_1,x_2,...\}
    \end{equation*}

    \subsection{Tipos de Notación}


    \begin{description}
        \item[Extención] Como los casos presentados anteriormente, esté tipo de
            notación implica en enumerar elementos demostrando el patron que
            describe el comportamiento del conjunto.

            \begin{equation*}

                \mathbb P=\{2,4,6,...,40\}
            \end{equation*}

            \begin{equation*}

                \mathbb W=\{amarillo,azul,rojo\}
            \end{equation*}

        \item[Comprención] En este caso tipo de notación no se mencionan los
            elemento, sino que se mencionan las caracteristicas que tiene cada
            elemento perteneciete a el dicho conjunto.

            \begin{equation*}

                 \mathbb W = \{x|x\ es\ un\ color\ primario\}
            \end{equation*}

            \begin{equation*}

                \mathbb P=\{x|x\ es\ un\ número\ par\ entre\ el\ 2\ y\ el\ 40\}
            \end{equation*}
    \end{description}

    \subsection{Relaciones Entre Conjuntos}


    Se dice que $ \mathbb S $ es un subconjunto de $ \mathbb T $ ($ \mathbb
    S,\mathbb T $ conjuntos) es decir que $ \mathbb S $ está contenido en $
    \mathbb T $ si todo elemento de $ \mathbb S $ es también un elemento de $
    \mathbb T $.

    \subsubsection{Notación Sub Conjuntos}


    En el caso donde el \textbf{sub conjunto} puede ser el mismo
    \textbf{conjunto} se utiliza la siguiente notación:

    \begin{equation*}

        \mathbb S\subseteq \mathbb T
    \end{equation*}

    Pero si el \textbf{sub conjunto} no puede ser el mismo \textbf{conjunto} se
    utiliza la siguiente notación:

    \begin{equation*}

        \mathbb S\subset \mathbb T
    \end{equation*}

    En el caso de que un \textbf{conjunto} no esté contenido en otro, se
    utiliza la siguiente notación:

    \begin{equation*}

        \mathbb S\not\subseteq \mathbb T
    \end{equation*}

    \subsection{Conjunto Universal}


    Denotamos con $ \Omega $ el \textbf{Conjunto Universal}; un conjunto
    especial que como caracteristica principal tiene a todos los elementos de
    interés en un determinado contexto.

    \begin{description}
        \item[Ejemplo] $\Omega=\mathbb C$ si estudiamos raíces de polinomios
            con coeficientes reales. (\textit{Teorema Findamental del
            Álgebra}).
    \end{description}

    \subsection{Álgebra de Conjuntos}


    \begin{description}
        \item[Complemento] La notación pala el complemento es $\mathbb S^{c}$
            dondo nos referimos al \textbf{complemento} de $\mathbb S$. Y esté
            se puede definir de la siguiente forma:

            \begin{equation*}

                \mathbb S^{c} = \{x|x\not\in \mathbb S\ (x\in \Omega)\}
            \end{equation*}

        \item[Unión] Donde $\mathbb S$ y $\mathbb J$ son conjuntos, la notación
            para la unión entre dos conjuntos es $\mathbb S\cup \mathbb J$ e
            implica:

            \begin{equation*}

                \mathbb S\cup \mathbb J = \{x|x\in \mathbb S\ ó\ x\in \mathbb J\}
            \end{equation*}

        \item[Intersección] Sean $\mathbb S$ y $\mathbb T$ conjuntos, su
            intersección se escribe: $\mathbb S\cap \mathbb T$, y se define de
            la siguiente forma:

            \begin{equation*}

                \mathbb S \cap \mathbb T = \{x|x\in \mathbb S\ y\ x\in \mathbb T\}
            \end{equation*}

        \item[Unión entre varios (o infinitos) conjunto]
            \begin{equation*}

                \bigcup_{i=0}^{n}\mathbb S_{i} = \{x|x\in \mathbb S_{i}\ (0>i>n)\}
            \end{equation*}

        \item[Intersección entre varios (o infinitos) conjunto]
            \begin{equation*}

                \bigcap_{i=0}^{n}\mathbb S_{i} = \{x|x\in \mathbb S_{i}\ (0>i>n)\}
            \end{equation*}

        \item[Conjuntos Disyuntos] Dos conjuntos $\mathbb S\ y\ \mathbb T$ se
            dicen disyuntos o disjuntos si $\mathbb S\cap \mathbb T=\emptyset$
            lo que se generaliza al decir que $\bigcap_{i=0}^{n}\mathbb
            S_{i}=\emptyset$.

        \item[Disyunción 2 a 2] Varios conjuntos $\mathbb S_i$ se dicen
            conjuntos disyuntos 2 a 2 si $\mathbb S_i \cap \mathbb
            S_j=\emptyset$.
    \end{description}

    \begin{obs}
        El par ordenado de dos objetos $x,y$ se denota por $(x,y)$ donde
        $(x,y)\ne(y,x)$. Lo que se diferencia de conjuntos donde
        $\{x,y\}=\{y,x\}$.
    \end{obs}

    \begin{obs}
        Los diagramas de venn (representaciones graficas de conjuntos) resultan
        útiles al realizar problemas que involucran conjuntos.
    \end{obs}

        \section{05.08.2020 - Tarea 1}



        \item Demostrar los siguientes lemas:

            \begin{enumerate}

                \item $\mathbb S\cup \mathbb T= \mathbb T\cup \mathbb S$
                    (conmutatividad)

                    \textbf{Demostración}

                    Sean $\mathbb S,\mathbb T$ conjuntos, para demostrar que
                    $\mathbb S\cup \mathbb T= \mathbb T\cup \mathbb S$ es
                    necesario demostrar las dos siguientes condiciones:

                    \begin{itemize}
                        \item Si $x\in(\mathbb S\cup \mathbb T)$ es decir $x\in
                            \mathbb S$ o $x\in \mathbb S$ o $x\in \mathbb T$,
                            luego se pude decir que $x\in(\mathbb T\cup \mathbb
                            S)$.

                        \item Si $x\in(\mathbb T\cup \mathbb S)$ es decir $x\in
                            \mathbb T$ o $x\in \mathbb S$ o $x\in \mathbb S$,
                            luego se pude decir que $x\in(\mathbb S\cup \mathbb
                            T)$.
                    \end{itemize}

                    Como ambas situaciones son verdaderas, se puede concluir
                    que la proposición es verdadera. $\blacksquare$


                \item $\mathbb S\cap(\mathbb T\cup \mathbb U)=(\mathbb S\cap
                    \mathbb T)\cup(\mathbb S\cap \mathbb U)$ (distributividad)

                    \textbf{Demostración}

                    Sean $\mathbb S, \mathbb T\ y\ \mathbb U$ conjuntos, para
                    demostrar que la proposición es verdadera hay que demostrar
                    las dos siguientes condiciones:

                    \begin{itemize}
                        \item Si $x\in\left( \mathbb S\cap\left( \mathbb
                            T\cup\mathbb U \right) \right)$ significa que
                            $x\in\mathbb S$ y también que $x\in\left( \mathbb
                            T\cup\mathbb U \right)$ es decir, que
                            incondicionalmente $x\in\mathbb S$ pero también
                            $x\in\mathbb T$ ó $x\in\mathbb U$. Por lo tanto se
                            puede decir que $x\in\left( \left( \mathbb
                            S\cap\mathbb T \right)\cup\left( \mathbb
                            S\cap\mathbb U\right) \right)$.
                        \item Por otro lado hay que asumir que $x\in\left(
                            \left( \mathbb S\cap\mathbb T\right)\cup\left(
                            \mathbb S\cap\mathbb U\right) \right)$, lo que
                            significa que ó $x\in\left( \mathbb S\cap\mathbb T
                            \right)$ ó $x\in\left( \mathbb S\cap\mathbb U
                            \right)$. A partir de lo anterior se puede asegurar
                            que $x\in\mathbb S$ y que $x\in\mathbb T$ ó
                            $x\in\mathbb U$, que es lo mismo que decir
                            $x\in\left( \mathbb S\cap\left( \mathbb
                            T\cup\mathbb U \right) \right)$
                    \end{itemize}

                    Ya que se cumplen ambas condiciones, se puede concluir que
                    la proposición es verdadera. $\blacksquare$

                \item $(\mathbb S^{c})^{c} = \mathbb S$

                \textbf{Demostración}

                Sea $\mathbb S$ un conjunto, es necesario demostrar dos
                situaciones para demostrar verdadera a la proposición:

                \begin{itemize}
                    \item El hecho de que $x\in(S^{c})^{c}$ quiere decir que
                        $x\notin\mathbb S^{c}$ y por la definición de
                        complemento, se puede asegurar que $x\in\mathbb S$.
                    \item Asumiendo que $x\in\mathbb S$, por definición se
                        puede decir que $x\notin\mathbb S^{c}$, implicando que
                        x pertenece al complemento de $\mathbb S$.
                \end{itemize}

                Ya que ambas situaciones son verdaderas se a demostrado
                verdadera a la proposición. $\blacksquare$

                \item $\mathbb S\cup \Omega = \Omega$

                \textbf{Demostración}

                Sea $\mathbb S$ un \textbf{sub conjunto} de $\Omega$, se tiene:

                \begin{itemize}
                    \item Al decir $x\in\left( \mathbb S\cup\Omega \right)$,
                        por definición de \textbf{sub conjunto} se puede
                        asegurar que es lo mismo que decir que $x\in\Omega$ ya
                        que todo x que esté en $\mathbb S$ va a estar en
                        $\Omega$.
                    \item Por otro lado al decir que $x\in\Omega$ se asegura
                        que x pertenece a la unión entre $\Omega$ y cualquiera
                        de sus \textbf{sub conjuntos}. Por lo tanto se puede
                        igualar con $\mathbb S\cup\Omega$.
                \end{itemize}

            Teniendo en cuenta de que ambas condiciones se cumplen, se ha
            demostrando que la proposición es verdadera. $\blacksquare$
            \np
                \item $\mathbb S\cup(\mathbb T\cup  \mathbb U) = (\mathbb S\cup
                    \mathbb T)\cup \mathbb U$ (asociatividad)

                \textbf{Demostración}

                Sean $\mathbb S,\mathbb T,\mathbb U$ conjuntos luego:
                \begin{equation*}

                \begin{split}
                    x\in\left( \mathbb S\cup\left( \mathbb T\cup\mathbb U
                    \right) \right) & \iff x\in\mathbb S\lor x\in\left( \mathbb
                    T\cup\mathbb U\right)\\
                    & \iff x\in\mathbb S\lor \left( x\in\mathbb T\lor x
                    \in\mathbb U \right)\\
                    & \iff x\in\mathbb S\lor x\in\mathbb T\lor x \in\mathbb U\\
                    & \iff x\in\left( \left( \mathbb S\cup \mathbb T
                    \right)\cup\mathbb U \right)\\
                \end{split}
                \end{equation*}

                Así demostrando que la proposición es verdadera. $\blacksquare$

                \item $\mathbb S\cup(\mathbb T\cap \mathbb U)=(\mathbb S\cup
                    \mathbb T)\cap(\mathbb S\cup \mathbb U)$

                \textbf{Demostración}

                Sean $\mathbb S\cup\mathbb T\cup\mathbb U$ conjuntos, entonces:
                \begin{equation*}

                \begin{split}
                    x\in\left(\mathbb S\cup\left(\mathbb T\cap \mathbb
                    U\right)\right)&\iff x\in\mathbb S\lor\left(\mathbb
                    T\cap\mathbb U\right)\\
                    &\iff x\in\mathbb S\lor\left(x\in\mathbb T\land x\in\mathbb
                    U\right)\\
                    &\iff \left(x\in\mathbb S\lor x\in\mathbb
                    T\right)\land\left(x\in\mathbb S\lor x\in\mathbb U\right)\\
                    &\iff x \in\left(\left(\mathbb S\cup\mathbb
                    T\right)\cap\left(\mathbb S\cup\mathbb U\right)\right)
                \end{split}
                \end{equation*}

                De está forma se acaba de comprobar que la proposición es
                verdadera. $\blacksquare$

                \item $\mathbb S\cap \mathbb S^{c}=\emptyset$

                \textbf{Demostración}

                Sea $\mathbb S$ un conjunto, y por contradicción suponiendo que
                $\mathbb S\cap\mathbb S^{c}\ne\emptyset$, por lo tanto $\exists
                x$ tal que $x\in\left(\mathbb S\cap\mathbb S^{c}\right)$ pero
                por definición del complemento de $\mathbb S$ ($\mathbb S^{c}$
                es todo lo que no está dentro de $\mathbb S$) significa que
                $\nexists x\ \left(\to\gets\right)$. Asi demostrando que la
                proposición es verdadera. $\blacksquare$

                \item $\mathbb S\cap \Omega=\mathbb S$

                \textbf{Demostración}

                Sea $\mathbb S$ un conjunto, entonces es necesario evealuar las
                siguientes situaciones:
                \begin{itemize}
                    \item Suponiendo que $x\in\left(\mathbb S\cap\Omega\right)$
                        como $\mathbb S\subseteq\Omega$ entonces de cualquier
                        forma $\forall x\in\mathbb S$ tambien $x\in\Omega$.
                        Pero $\nexist x$ talque $x\in\mathbb S^{c}$.

                    \item Suponiendo que $x\in\mathbb S$, y como $\mathbb
                        S\subseteq\Omega$ entonces $x\in\mathbb S\land
                        x\in\Omeg$ es decir $x\in\left(\mathbb
                        S\cap\Omega\right)$.
                \end{itemize}

                Ya que en ambos casos son verdaderos se ha demostrado que la
                proposición es verdadera. $\blacksquare$
            \end{enumerate}

        \item Demostrar las \textbf{Leyes de De Morgan}:
            \begin{enumerate}
                \item \[\Big(\bigcup_{i=1}^{n}\mathbb S_{i}\Big)^{c} = \bigcap_{i=1}^{n}\mathbb S_{i}^{c}\]

                \textbf{Demostración}

                Sea la union de varios \textbf{conjuntos} tal que
                    $\bigcup_{i=1}^{n}\mathbb S_{i}=\mathbb S_{1}\cup\mathbb
                    S_{2}\cup\mathbb S_{3}\cup...\cup\mathbb S_{n}$ ahora por
                    induccion matematica se van a revisar los siguientes casos:

                    \begin{description}
                        \item[Caso base (n=1)]\\
                        \begin{equation*}

                        \Big(\bigcup_{i=1}^{1}\mathbb S_{i}\Big)^{c} =
                            \Big(\mathbb S_{i}\Big)^{c} = \mathbb
                            S_{1}^{c}=\bigcap_{i=1}^{1}\mathbb S_{1}^{c}
                        \end{equation*}
                        \np
                        \item[Caso inductivo]\\
                            Suponiendo que $\bigcup_{i=1}^{n}\mathbb S_{i}$ la propiedad tal que:
                            \begin{equation*}

                            \begin{split}
                            &\Big(\bigcup_{i=1}^{n}\mathbb S_{i}\Big)^{c} =
                                \mathbb S_{1}^{c}=\bigcap_{i=1}^{n}\mathbb
                                S_{i}^{c}\\
                            &Ahora:\\
                            &\left(\mathbb S_{1}\cup\mathbb
                                S_{2}\cup...\cup\mathbb
                                S_{n}\right)^{c}\cap\mathbb S_{n+1}^{c}
                                =\left(\mathbb S_{1}^{c}\cap\mathbb
                                S_{2}^{c}\cap...\cap\mathbb
                                S_{n}^{c}\right)\cap\mathbb S_{n+1}^{c}\\
                            &\left(\mathbb S_{1}\cup\mathbb
                                S_{2}\cup...\cup\mathbb S_{n}\cup\mathbb
                                S_{n+1}\right)^{c}=\left(\mathbb
                                S_{1}^{c}\cap\mathbb
                                S_{2}^{c}\cap...\cap\mathbb
                                S_{n}^{c}\cap\mathbb S_{n+1}^{c}\right)\\
                            &\Big(\bigcup_{i=1}^{n+1}\mathbb
                                S_{i}\Big)^{c}=\bigcap_{i=1}^{n+1}\mathbb
                                S_{i}^{c}\\
                            \end{split}
                            \end{equation*}
                            Así demostrando por inducción matemática que la
                            proposición es verdadera para todo $n>1$.
                            $\blacksquare$

                    \end{description}

                \item \[\Big(\bigcap_{i=0}^{n}\mathbb S_i\Big)^{c} =
                    \bigcup_{i=0}^{n}\mathbb S_{i}^{c}\]
            \end{enumerate}



    \subsection{Repaso Tería de Conjuntos}


    \begin{description}
        \item[Conjunto] Un conjunto es una colección de objetos de cualquier
            tipo (ej: números, personas, colores, sabores, etc...), a estos
            objetos se les  conoce como los elementos del conjunto.\\
            \textbf{Ejemplos:}
            \begin{itemize}
                \item Los números naturales $\mathbb N$.
                \item Los alumnos del curso.
            \end{itemize}
        \item[Objetos] Siendo así, son los objetos aquellos que definen a los
            \textbf{conjuntos} en su totalidad.
    \end{description}
    \subsubsection{Notación de Conjuntos}


        Si $\mathbb S$ es un conjunto y \textbf{x} un elemento de $\mathbb S$,
        escribimos $x\in \mathbb S$. Pero en el caso contrario, donde
        \textbf{x} no es un elemento de $\mathbb S$ escribimos $x\notin \mathbb
        S$.

    \begin{description}
        \item[Conjunto Vacío] Este conjunto se caracteriza por no tener ningún
            elemento dentro de si mismo, y se denota: $\emptyset$.
    \end{description}

    \subsubsection{Notación Conjunto Finito}


        Si $ \mathbb S $ es un conjunto finito co elementos $ x_1,x_2,...,x_n
        $. Podemos denotar a S como:
        \begin{equation*}

            \mathbb S = \{x_1,x_2,...,x_n\}
        \end{equation*}

    \begin{description}
        \item[Ejemplos:] $ \mathbb S $ conjuto de resulatdos de un dado
            \begin{equation*}

                \mathbb S=\{1,2,3,...,6\}
            \end{equation*}
    \end{description}

    \subsubsection{Notación Conjunto Infinito}


    Si $ \mathbb S$ es un conjunto infinito enumerable con elementos $
    x_1,x_2,...$ se puede escribir a $ \mathbb S $ como:

    \begin{equation*}

        \mathbb S=\{x_1,x_2,...\}
    \end{equation*}

    \subsection{Tipos de Notación}


    \begin{description}
        \item[Extención] Como los casos presentados anteriormente, esté tipo de
            notación implica en enumerar elementos demostrando el patron que
            describe el comportamiento del conjunto.

            \begin{equation*}

                \mathbb P=\{2,4,6,...,40\}
            \end{equation*}

            \begin{equation*}

                \mathbb W=\{amarillo,azul,rojo\}
            \end{equation*}

        \item[Comprención] En este caso tipo de notación no se mencionan los
            elemento, sino que se mencionan las caracteristicas que tiene cada
            elemento perteneciete a el dicho conjunto.

            \begin{equation*}

                 \mathbb W = \{x|x\ es\ un\ color\ primario\}
            \end{equation*}

            \begin{equation*}

                \mathbb P=\{x|x\ es\ un\ número\ par\ entre\ el\ 2\ y\ el\ 40\}
            \end{equation*}
    \end{description}

    \subsection{Relaciones Entre Conjuntos}


    Se dice que $ \mathbb S $ es un subconjunto de $ \mathbb T $ ($ \mathbb
    S,\mathbb T $ conjuntos) es decir que $ \mathbb S $ está contenido en $
    \mathbb T $ si todo elemento de $ \mathbb S $ es también un elemento de $
    \mathbb T $.

    \subsubsection{Notación Sub Conjuntos}


    En el caso donde el \textbf{sub conjunto} puede ser el mismo
    \textbf{conjunto} se utiliza la siguiente notación:

    \begin{equation*}

        \mathbb S\subseteq \mathbb T
    \end{equation*}

    Pero si el \textbf{sub conjunto} no puede ser el mismo \textbf{conjunto} se
    utiliza la siguiente notación:

    \begin{equation*}

        \mathbb S\subset \mathbb T
    \end{equation*}

    En el caso de que un \textbf{conjunto} no esté contenido en otro, se
    utiliza la siguiente notación:

    \begin{equation*}

        \mathbb S\not\subseteq \mathbb T
    \end{equation*}

    \subsection{Conjunto Universal}


    Denotamos con $ \Omega $ el \textbf{Conjunto Universal}; un conjunto
    especial que como caracteristica principal tiene a todos los elementos de
    interés en un determinado contexto.

    \begin{description}
        \item[Ejemplo] $\Omega=\mathbb C$ si estudiamos raíces de polinomios
            con coeficientes reales. (\textit{Teorema Findamental del
            Álgebra}).
    \end{description}

    \subsection{Álgebra de Conjuntos}


    \begin{description}
        \item[Complemento] La notación pala el complemento es $\mathbb S^{c}$
            dondo nos referimos al \textbf{complemento} de $\mathbb S$. Y esté
            se puede definir de la siguiente forma:

            \begin{equation*}

                \mathbb S^{c} = \{x|x\not\in \mathbb S\ (x\in \Omega)\}
            \end{equation*}

        \item[Unión] Donde $\mathbb S$ y $\mathbb J$ son conjuntos, la notación
            para la unión entre dos conjuntos es $\mathbb S\cup \mathbb J$ e
            implica:

            \begin{equation*}

                \mathbb S\cup \mathbb J = \{x|x\in \mathbb S\ ó\ x\in \mathbb J\}
            \end{equation*}

        \item[Intersección] Sean $\mathbb S$ y $\mathbb T$ conjuntos, su
            intersección se escribe: $\mathbb S\cap \mathbb T$, y se define de
            la siguiente forma:

            \begin{equation*}

                \mathbb S \cap \mathbb T = \{x|x\in \mathbb S\ y\ x\in \mathbb T\}
            \end{equation*}

        \item[Unión entre varios (o infinitos) conjunto]
            \begin{equation*}

                \bigcup_{i=0}^{n}\mathbb S_{i} = \{x|x\in \mathbb S_{i}\ (0>i>n)\}
            \end{equation*}

        \item[Intersección entre varios (o infinitos) conjunto]
            \begin{equation*}

                \bigcap_{i=0}^{n}\mathbb S_{i} = \{x|x\in \mathbb S_{i}\ (0>i>n)\}
            \end{equation*}

        \item[Conjuntos Disyuntos] Dos conjuntos $\mathbb S\ y\ \mathbb T$ se
            dicen disyuntos o disjuntos si $\mathbb S\cap \mathbb T=\emptyset$
            lo que se generaliza al decir que $\bigcap_{i=0}^{n}\mathbb
            S_{i}=\emptyset$.

        \item[Disyunción 2 a 2] Varios conjuntos $\mathbb S_i$ se dicen
            conjuntos disyuntos 2 a 2 si $\mathbb S_i \cap \mathbb
            S_j=\emptyset$.
    \end{description}

    \begin{obs}
        El par ordenado de dos objetos $x,y$ se denota por $(x,y)$ donde
        $(x,y)\ne(y,x)$. Lo que se diferencia de conjuntos donde
        $\{x,y\}=\{y,x\}$.
    \end{obs}

    \begin{obs}
        Los diagramas de venn (representaciones graficas de conjuntos) resultan
        útiles al realizar problemas que involucran conjuntos.
    \end{obs}

    \begin{tarea}{Tarea Demostración de Lemas}{03.08.2020}{05.08.2020}
        \item Demostrar los siguientes lemas:

            \begin{enumerate}

                \item $\mathbb S\cup \mathbb T= \mathbb T\cup \mathbb S$
                    (conmutatividad)

                    \textbf{Demostración}

                    Sean $\mathbb S,\mathbb T$ conjuntos, para demostrar que
                    $\mathbb S\cup \mathbb T= \mathbb T\cup \mathbb S$ es
                    necesario demostrar las dos siguientes condiciones:

                    \begin{itemize}
                        \item Si $x\in(\mathbb S\cup \mathbb T)$ es decir $x\in
                            \mathbb S$ o $x\in \mathbb S$ o $x\in \mathbb T$,
                            luego se pude decir que $x\in(\mathbb T\cup \mathbb
                            S)$.

                        \item Si $x\in(\mathbb T\cup \mathbb S)$ es decir $x\in
                            \mathbb T$ o $x\in \mathbb S$ o $x\in \mathbb S$,
                            luego se pude decir que $x\in(\mathbb S\cup \mathbb
                            T)$.
                    \end{itemize}

                    Como ambas situaciones son verdaderas, se puede concluir
                    que la proposición es verdadera. $\blacksquare$


                \item $\mathbb S\cap(\mathbb T\cup \mathbb U)=(\mathbb S\cap
                    \mathbb T)\cup(\mathbb S\cap \mathbb U)$ (distributividad)

                    \textbf{Demostración}

                    Sean $\mathbb S, \mathbb T\ y\ \mathbb U$ conjuntos, para
                    demostrar que la proposición es verdadera hay que demostrar
                    las dos siguientes condiciones:

                    \begin{itemize}
                        \item Si $x\in\left( \mathbb S\cap\left( \mathbb
                            T\cup\mathbb U \right) \right)$ significa que
                            $x\in\mathbb S$ y también que $x\in\left( \mathbb
                            T\cup\mathbb U \right)$ es decir, que
                            incondicionalmente $x\in\mathbb S$ pero también
                            $x\in\mathbb T$ ó $x\in\mathbb U$. Por lo tanto se
                            puede decir que $x\in\left( \left( \mathbb
                            S\cap\mathbb T \right)\cup\left( \mathbb
                            S\cap\mathbb U\right) \right)$.
                        \item Por otro lado hay que asumir que $x\in\left(
                            \left( \mathbb S\cap\mathbb T\right)\cup\left(
                            \mathbb S\cap\mathbb U\right) \right)$, lo que
                            significa que ó $x\in\left( \mathbb S\cap\mathbb T
                            \right)$ ó $x\in\left( \mathbb S\cap\mathbb U
                            \right)$. A partir de lo anterior se puede asegurar
                            que $x\in\mathbb S$ y que $x\in\mathbb T$ ó
                            $x\in\mathbb U$, que es lo mismo que decir
                            $x\in\left( \mathbb S\cap\left( \mathbb
                            T\cup\mathbb U \right) \right)$
                    \end{itemize}

                    Ya que se cumplen ambas condiciones, se puede concluir que
                    la proposición es verdadera. $\blacksquare$

                \item $(\mathbb S^{c})^{c} = \mathbb S$

                \textbf{Demostración}

                Sea $\mathbb S$ un conjunto, es necesario demostrar dos
                situaciones para demostrar verdadera a la proposición:

                \begin{itemize}
                    \item El hecho de que $x\in(S^{c})^{c}$ quiere decir que
                        $x\notin\mathbb S^{c}$ y por la definición de
                        complemento, se puede asegurar que $x\in\mathbb S$.
                    \item Asumiendo que $x\in\mathbb S$, por definición se
                        puede decir que $x\notin\mathbb S^{c}$, implicando que
                        x pertenece al complemento de $\mathbb S$.
                \end{itemize}

                Ya que ambas situaciones son verdaderas se a demostrado
                verdadera a la proposición. $\blacksquare$

                \item $\mathbb S\cup \Omega = \Omega$

                \textbf{Demostración}

                Sea $\mathbb S$ un \textbf{sub conjunto} de $\Omega$, se tiene:

                \begin{itemize}
                    \item Al decir $x\in\left( \mathbb S\cup\Omega \right)$,
                        por definición de \textbf{sub conjunto} se puede
                        asegurar que es lo mismo que decir que $x\in\Omega$ ya
                        que todo x que esté en $\mathbb S$ va a estar en
                        $\Omega$.
                    \item Por otro lado al decir que $x\in\Omega$ se asegura
                        que x pertenece a la unión entre $\Omega$ y cualquiera
                        de sus \textbf{sub conjuntos}. Por lo tanto se puede
                        igualar con $\mathbb S\cup\Omega$.
                \end{itemize}

            Teniendo en cuenta de que ambas condiciones se cumplen, se ha
            demostrando que la proposición es verdadera. $\blacksquare$
            \np
                \item $\mathbb S\cup(\mathbb T\cup  \mathbb U) = (\mathbb S\cup
                    \mathbb T)\cup \mathbb U$ (asociatividad)

                \textbf{Demostración}

                Sean $\mathbb S,\mathbb T,\mathbb U$ conjuntos luego:
                \begin{equation*}

                \begin{split}
                    x\in\left( \mathbb S\cup\left( \mathbb T\cup\mathbb U
                    \right) \right) & \iff x\in\mathbb S\lor x\in\left( \mathbb
                    T\cup\mathbb U\right)\\
                    & \iff x\in\mathbb S\lor \left( x\in\mathbb T\lor x
                    \in\mathbb U \right)\\
                    & \iff x\in\mathbb S\lor x\in\mathbb T\lor x \in\mathbb U\\
                    & \iff x\in\left( \left( \mathbb S\cup \mathbb T
                    \right)\cup\mathbb U \right)\\
                \end{split}
                \end{equation*}

                Así demostrando que la proposición es verdadera. $\blacksquare$

                \item $\mathbb S\cup(\mathbb T\cap \mathbb U)=(\mathbb S\cup
                    \mathbb T)\cap(\mathbb S\cup \mathbb U)$

                \textbf{Demostración}

                Sean $\mathbb S\cup\mathbb T\cup\mathbb U$ conjuntos, entonces:
                \begin{equation*}

                \begin{split}
                    x\in\left(\mathbb S\cup\left(\mathbb T\cap \mathbb
                    U\right)\right)&\iff x\in\mathbb S\lor\left(\mathbb
                    T\cap\mathbb U\right)\\
                    &\iff x\in\mathbb S\lor\left(x\in\mathbb T\land x\in\mathbb
                    U\right)\\
                    &\iff \left(x\in\mathbb S\lor x\in\mathbb
                    T\right)\land\left(x\in\mathbb S\lor x\in\mathbb U\right)\\
                    &\iff x \in\left(\left(\mathbb S\cup\mathbb
                    T\right)\cap\left(\mathbb S\cup\mathbb U\right)\right)
                \end{split}
                \end{equation*}

                De está forma se acaba de comprobar que la proposición es
                verdadera. $\blacksquare$

                \item $\mathbb S\cap \mathbb S^{c}=\emptyset$

                \textbf{Demostración}

                Sea $\mathbb S$ un conjunto, y por contradicción suponiendo que
                $\mathbb S\cap\mathbb S^{c}\ne\emptyset$, por lo tanto $\exists
                x$ tal que $x\in\left(\mathbb S\cap\mathbb S^{c}\right)$ pero
                por definición del complemento de $\mathbb S$ ($\mathbb S^{c}$
                es todo lo que no está dentro de $\mathbb S$) significa que
                $\nexists x\ \left(\to\gets\right)$. Asi demostrando que la
                proposición es verdadera. $\blacksquare$

                \item $\mathbb S\cap \Omega=\mathbb S$

                \textbf{Demostración}

                Sea $\mathbb S$ un conjunto, entonces es necesario evealuar las
                siguientes situaciones:
                \begin{itemize}
                    \item Suponiendo que $x\in\left(\mathbb S\cap\Omega\right)$
                        como $\mathbb S\subseteq\Omega$ entonces de cualquier
                        forma $\forall x\in\mathbb S$ tambien $x\in\Omega$.
                        Pero $\nexist x$ talque $x\in\mathbb S^{c}$.

                    \item Suponiendo que $x\in\mathbb S$, y como $\mathbb
                        S\subseteq\Omega$ entonces $x\in\mathbb S\land
                        x\in\Omeg$ es decir $x\in\left(\mathbb
                        S\cap\Omega\right)$.
                \end{itemize}

                Ya que en ambos casos son verdaderos se ha demostrado que la
                proposición es verdadera. $\blacksquare$
            \end{enumerate}

        \item Demostrar las \textbf{Leyes de De Morgan}:
            \begin{enumerate}
                \item \[\Big(\bigcup_{i=1}^{n}\mathbb S_{i}\Big)^{c} = \bigcap_{i=1}^{n}\mathbb S_{i}^{c}\]

                \textbf{Demostración}

                Sea la union de varios \textbf{conjuntos} tal que
                    $\bigcup_{i=1}^{n}\mathbb S_{i}=\mathbb S_{1}\cup\mathbb
                    S_{2}\cup\mathbb S_{3}\cup...\cup\mathbb S_{n}$ ahora por
                    induccion matematica se van a revisar los siguientes casos:

                    \begin{description}
                        \item[Caso base (n=1)]\\
                        \begin{equation*}

                        \Big(\bigcup_{i=1}^{1}\mathbb S_{i}\Big)^{c} =
                            \Big(\mathbb S_{i}\Big)^{c} = \mathbb
                            S_{1}^{c}=\bigcap_{i=1}^{1}\mathbb S_{1}^{c}
                        \end{equation*}
                        \np
                        \item[Caso inductivo]\\
                            Suponiendo que $\bigcup_{i=1}^{n}\mathbb S_{i}$ la propiedad tal que:
                            \begin{equation*}

                            \begin{split}
                            &\Big(\bigcup_{i=1}^{n}\mathbb S_{i}\Big)^{c} =
                                \mathbb S_{1}^{c}=\bigcap_{i=1}^{n}\mathbb
                                S_{i}^{c}\\
                            &Ahora:\\
                            &\left(\mathbb S_{1}\cup\mathbb
                                S_{2}\cup...\cup\mathbb
                                S_{n}\right)^{c}\cap\mathbb S_{n+1}^{c}
                                =\left(\mathbb S_{1}^{c}\cap\mathbb
                                S_{2}^{c}\cap...\cap\mathbb
                                S_{n}^{c}\right)\cap\mathbb S_{n+1}^{c}\\
                            &\left(\mathbb S_{1}\cup\mathbb
                                S_{2}\cup...\cup\mathbb S_{n}\cup\mathbb
                                S_{n+1}\right)^{c}=\left(\mathbb
                                S_{1}^{c}\cap\mathbb
                                S_{2}^{c}\cap...\cap\mathbb
                                S_{n}^{c}\cap\mathbb S_{n+1}^{c}\right)\\
                            &\Big(\bigcup_{i=1}^{n+1}\mathbb
                                S_{i}\Big)^{c}=\bigcap_{i=1}^{n+1}\mathbb
                                S_{i}^{c}\\
                            \end{split}
                            \end{equation*}
                            Así demostrando por inducción matemática que la
                            proposición es verdadera para todo $n>1$.
                            $\blacksquare$

                    \end{description}

                \item \[\Big(\bigcap_{i=0}^{n}\mathbb S_i\Big)^{c} =
                    \bigcup_{i=0}^{n}\mathbb S_{i}^{c}\]
            \end{enumerate}

\end{document}
