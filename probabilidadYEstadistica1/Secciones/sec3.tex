\documentclass[a4paper]{book}

% ====| P A C K A G E S |==== %
\usepackage{import}
%\usepackage[backend=biber, style=authoryear-icomp]{biblatex}    %Bibliography stuff
%\ProvidesPackage{cmands}
% ====| P A C K A G E S |==== %
\usepackage{import}
%\usepackage[backend=biber, style=authoryear-icomp]{biblatex}    %Bibliography stuff
%\ProvidesPackage{cmands}
% ====| P A C K A G E S |==== %
\usepackage{import}
%\usepackage[backend=biber, style=authoryear-icomp]{biblatex}    %Bibliography stuff
%\ProvidesPackage{cmands}
% ====| P A C K A G E S |==== %
\usepackage{import}
%\usepackage[backend=biber, style=authoryear-icomp]{biblatex}    %Bibliography stuff
%\input{./../../commands.tex}
\usepackage[margin=1in,includefoot]{geometry}   %Margins
\usepackage[utf8]{inputenc} %Language stuff
\usepackage [latin1]{inputenc}  %Spanish symbols
\usepackage[spanish]{babel} %Sets document language to spanish
\usepackage{tcolorbox}  %Frame boxes
\usepackage{enumerate}  %Lists options
\usepackage{mfirstuc}   %I use it to capitalize words
\usepackage{graphicx}   %Use Images
\usepackage{listings}   %For displaying code
\usepackage{titlesec}   %Costume titles/sections/...
\usepackage{hyperref}   %Linking options
\usepackage{multicol}	%Use column
\usepackage{amsmath}    %Display equations options
\usepackage{amssymb}    %More symbols
\usepackage{titling}    %Use title variables in other places
\usepackage{xcolor}     %To manage colors
\usepackage{transparent}%For figures
\usepackage{pdfpages}   %For figures
% =========================== %

% ====| P A C K A G E S    S E T T I N G S |==== %
% \addbibresource{/media/jpi/Data/01_Education/04_bibiografia/bibliography.bib} %<--- Bibliography path
\setlength{\columnsep}{1cm}
\hypersetup{
    colorlinks,
    citecolor=black,
    filecolor=black,
    linkcolor=black,
    urlcolor=black
}
% ============================================== %

% ====| P E R S O N A L    C O M M A N D S    &    E N V I R O N M E N T S|==== %

% - New figure
\newcommand{\incfig}[2][1]{%
    \def\svgwidth{#1\columnwidth}
    \import{./figures/}{#2.pdf_tex}
}

\pdfsuppresswarningpagegroup=1

% - New page
\newcommand{\np}{\null\newpage}

% - vertical thic line
\newcommand\mybar{\kern1pt\rule[-\dp\strutbox]{.8pt}{\baselineskip}\kern1pt}

% - new homework
\newenvironment{tarea}[3]
    {
        \null\newpage
        \begin{tcolorbox}
            \textbf{\asignatura\ -\ \autor}
            \subsection{\capitalisewords{#1}}
            \label{ssec:#1}
            \begin{flushright}
            \textbf{Desde:}  #2 \\
            \textbf{Hasta:}  #3 \\
            \end{flushright}
        \end{tcolorbox}

    \begin{enumerate}[{Ejercicio} 1.]
    }
    {
        \end{enumerate}
        \np
    }



% ============================================================================= %


% ====| H E A T H E R S    S E T T I N G S |==== %

\titleformat{\chapter}[display]
    {\normalfont\huge\bfseries\raggedleft}{\chaptertitlename\ \thechapter}
    {20pt}{\Huge}
\titleformat{\section}[display]
    {\Large\bfseries}{}
    {0em}{}[\titlerule]


\newcommand{\titPag}{
    \begin{titlepage}
        \begin{flushright}
            \textsc{\large {\semestre\ Semestre}}\\
            \line(1,0){450} \\
            [0.635cm]
            \huge{\bfseries \asignatura} \\
            [0.2cm]
            \line(1,0){350} \\
            \LARGE{\bfseries \autor} \\
            [16.25cm]
        \end{flushright}
        \begin{flushright}
        \textsc{
            \universidad \\
            [0.1cm]
            \escuela \\
            [0.1cm]
            \carrera
        }
        \end{flushright}
    \end{titlepage}
}
% ============================================== %

% ====| C O D E    I N    F I L E S    S E T T I N G S |==== %
\definecolor{codegreen}{rgb}{0,0.6,0}
\definecolor{codegray}{rgb}{0.5,0.5,0.5}
\definecolor{codepurple}{rgb}{0.58,0,0.82}
\definecolor{backcolour}{rgb}{0.95,0.95,0.92}

\lstdefinestyle{mystyle}{
    backgroundcolor=\color{backcolour},
    commentstyle=\color{codegreen},
    keywordstyle=\color{magenta},
    numberstyle=\tiny\color{codegray},
    stringstyle=\color{codepurple},
    basicstyle=\ttfamily\footnotesize,
    breakatwhitespace=false,
    breaklines=true,
    captionpos=b,
    keepspaces=true,
    numbers=left,
    numbersep=5pt,
    showspaces=false,
    showstringspaces=false,
    showtabs=false,
    tabsize=2
}

\lstset{style=mystyle}
% ========================================================== %

\usepackage[margin=1in,includefoot]{geometry}   %Margins
\usepackage[utf8]{inputenc} %Language stuff
\usepackage [latin1]{inputenc}  %Spanish symbols
\usepackage[spanish]{babel} %Sets document language to spanish
\usepackage{tcolorbox}  %Frame boxes
\usepackage{enumerate}  %Lists options
\usepackage{mfirstuc}   %I use it to capitalize words
\usepackage{graphicx}   %Use Images
\usepackage{listings}   %For displaying code
\usepackage{titlesec}   %Costume titles/sections/...
\usepackage{hyperref}   %Linking options
\usepackage{multicol}	%Use column
\usepackage{amsmath}    %Display equations options
\usepackage{amssymb}    %More symbols
\usepackage{titling}    %Use title variables in other places
\usepackage{xcolor}     %To manage colors
\usepackage{transparent}%For figures
\usepackage{pdfpages}   %For figures
% =========================== %

% ====| P A C K A G E S    S E T T I N G S |==== %
% \addbibresource{/media/jpi/Data/01_Education/04_bibiografia/bibliography.bib} %<--- Bibliography path
\setlength{\columnsep}{1cm}
\hypersetup{
    colorlinks,
    citecolor=black,
    filecolor=black,
    linkcolor=black,
    urlcolor=black
}
% ============================================== %

% ====| P E R S O N A L    C O M M A N D S    &    E N V I R O N M E N T S|==== %

% - New figure
\newcommand{\incfig}[2][1]{%
    \def\svgwidth{#1\columnwidth}
    \import{./figures/}{#2.pdf_tex}
}

\pdfsuppresswarningpagegroup=1

% - New page
\newcommand{\np}{\null\newpage}

% - vertical thic line
\newcommand\mybar{\kern1pt\rule[-\dp\strutbox]{.8pt}{\baselineskip}\kern1pt}

% - new homework
\newenvironment{tarea}[3]
    {
        \null\newpage
        \begin{tcolorbox}
            \textbf{\asignatura\ -\ \autor}
            \subsection{\capitalisewords{#1}}
            \label{ssec:#1}
            \begin{flushright}
            \textbf{Desde:}  #2 \\
            \textbf{Hasta:}  #3 \\
            \end{flushright}
        \end{tcolorbox}

    \begin{enumerate}[{Ejercicio} 1.]
    }
    {
        \end{enumerate}
        \np
    }



% ============================================================================= %


% ====| H E A T H E R S    S E T T I N G S |==== %

\titleformat{\chapter}[display]
    {\normalfont\huge\bfseries\raggedleft}{\chaptertitlename\ \thechapter}
    {20pt}{\Huge}
\titleformat{\section}[display]
    {\Large\bfseries}{}
    {0em}{}[\titlerule]


\newcommand{\titPag}{
    \begin{titlepage}
        \begin{flushright}
            \textsc{\large {\semestre\ Semestre}}\\
            \line(1,0){450} \\
            [0.635cm]
            \huge{\bfseries \asignatura} \\
            [0.2cm]
            \line(1,0){350} \\
            \LARGE{\bfseries \autor} \\
            [16.25cm]
        \end{flushright}
        \begin{flushright}
        \textsc{
            \universidad \\
            [0.1cm]
            \escuela \\
            [0.1cm]
            \carrera
        }
        \end{flushright}
    \end{titlepage}
}
% ============================================== %

% ====| C O D E    I N    F I L E S    S E T T I N G S |==== %
\definecolor{codegreen}{rgb}{0,0.6,0}
\definecolor{codegray}{rgb}{0.5,0.5,0.5}
\definecolor{codepurple}{rgb}{0.58,0,0.82}
\definecolor{backcolour}{rgb}{0.95,0.95,0.92}

\lstdefinestyle{mystyle}{
    backgroundcolor=\color{backcolour},
    commentstyle=\color{codegreen},
    keywordstyle=\color{magenta},
    numberstyle=\tiny\color{codegray},
    stringstyle=\color{codepurple},
    basicstyle=\ttfamily\footnotesize,
    breakatwhitespace=false,
    breaklines=true,
    captionpos=b,
    keepspaces=true,
    numbers=left,
    numbersep=5pt,
    showspaces=false,
    showstringspaces=false,
    showtabs=false,
    tabsize=2
}

\lstset{style=mystyle}
% ========================================================== %

\usepackage[margin=1in,includefoot]{geometry}   %Margins
\usepackage[utf8]{inputenc} %Language stuff
\usepackage [latin1]{inputenc}  %Spanish symbols
\usepackage[spanish]{babel} %Sets document language to spanish
\usepackage{tcolorbox}  %Frame boxes
\usepackage{enumerate}  %Lists options
\usepackage{mfirstuc}   %I use it to capitalize words
\usepackage{graphicx}   %Use Images
\usepackage{listings}   %For displaying code
\usepackage{titlesec}   %Costume titles/sections/...
\usepackage{hyperref}   %Linking options
\usepackage{multicol}	%Use column
\usepackage{amsmath}    %Display equations options
\usepackage{amssymb}    %More symbols
\usepackage{titling}    %Use title variables in other places
\usepackage{xcolor}     %To manage colors
\usepackage{transparent}%For figures
\usepackage{pdfpages}   %For figures
% =========================== %

% ====| P A C K A G E S    S E T T I N G S |==== %
% \addbibresource{/media/jpi/Data/01_Education/04_bibiografia/bibliography.bib} %<--- Bibliography path
\setlength{\columnsep}{1cm}
\hypersetup{
    colorlinks,
    citecolor=black,
    filecolor=black,
    linkcolor=black,
    urlcolor=black
}
% ============================================== %

% ====| P E R S O N A L    C O M M A N D S    &    E N V I R O N M E N T S|==== %

% - New figure
\newcommand{\incfig}[2][1]{%
    \def\svgwidth{#1\columnwidth}
    \import{./figures/}{#2.pdf_tex}
}

\pdfsuppresswarningpagegroup=1

% - New page
\newcommand{\np}{\null\newpage}

% - vertical thic line
\newcommand\mybar{\kern1pt\rule[-\dp\strutbox]{.8pt}{\baselineskip}\kern1pt}

% - new homework
\newenvironment{tarea}[3]
    {
        \null\newpage
        \begin{tcolorbox}
            \textbf{\asignatura\ -\ \autor}
            \subsection{\capitalisewords{#1}}
            \label{ssec:#1}
            \begin{flushright}
            \textbf{Desde:}  #2 \\
            \textbf{Hasta:}  #3 \\
            \end{flushright}
        \end{tcolorbox}

    \begin{enumerate}[{Ejercicio} 1.]
    }
    {
        \end{enumerate}
        \np
    }



% ============================================================================= %


% ====| H E A T H E R S    S E T T I N G S |==== %

\titleformat{\chapter}[display]
    {\normalfont\huge\bfseries\raggedleft}{\chaptertitlename\ \thechapter}
    {20pt}{\Huge}
\titleformat{\section}[display]
    {\Large\bfseries}{}
    {0em}{}[\titlerule]


\newcommand{\titPag}{
    \begin{titlepage}
        \begin{flushright}
            \textsc{\large {\semestre\ Semestre}}\\
            \line(1,0){450} \\
            [0.635cm]
            \huge{\bfseries \asignatura} \\
            [0.2cm]
            \line(1,0){350} \\
            \LARGE{\bfseries \autor} \\
            [16.25cm]
        \end{flushright}
        \begin{flushright}
        \textsc{
            \universidad \\
            [0.1cm]
            \escuela \\
            [0.1cm]
            \carrera
        }
        \end{flushright}
    \end{titlepage}
}
% ============================================== %

% ====| C O D E    I N    F I L E S    S E T T I N G S |==== %
\definecolor{codegreen}{rgb}{0,0.6,0}
\definecolor{codegray}{rgb}{0.5,0.5,0.5}
\definecolor{codepurple}{rgb}{0.58,0,0.82}
\definecolor{backcolour}{rgb}{0.95,0.95,0.92}

\lstdefinestyle{mystyle}{
    backgroundcolor=\color{backcolour},
    commentstyle=\color{codegreen},
    keywordstyle=\color{magenta},
    numberstyle=\tiny\color{codegray},
    stringstyle=\color{codepurple},
    basicstyle=\ttfamily\footnotesize,
    breakatwhitespace=false,
    breaklines=true,
    captionpos=b,
    keepspaces=true,
    numbers=left,
    numbersep=5pt,
    showspaces=false,
    showstringspaces=false,
    showtabs=false,
    tabsize=2
}

\lstset{style=mystyle}
% ========================================================== %

\usepackage[margin=1in,includefoot]{geometry}   %Margins
\usepackage[utf8]{inputenc} %Language stuff
%\usepackage [latin1]{inputenc}  %Spanish symbols
\usepackage[spanish]{babel} %Sets document language to spanish
\usepackage{tcolorbox}  %Frame boxes
\usepackage{enumerate}  %Lists options
\usepackage{mfirstuc}   %I use it to capitalize words
\usepackage{graphicx}   %Use Images
\usepackage{listings}   %For displaying code
\usepackage{titlesec}   %Costume titles/sections/...
\usepackage{hyperref}   %Linking options
\usepackage{multicol}	%Use column
\usepackage{amsmath}    %Display equations options
\usepackage{amssymb}    %More symbols
\usepackage{titling}    %Use title variables in other places
\usepackage{xcolor}     %To manage colors
\usepackage{transparent}%For figures
\usepackage{pdfpages}   %For figures
% =========================== %

% ====| P A C K A G E S    S E T T I N G S |==== %
% \addbibresource{/media/jpi/Data/01_Education/04_bibiografia/bibliography.bib} %<--- Bibliography path
% ============================================== %

% ====| P E R S O N A L    C O M M A N D S    &    E N V I R O N M E N T S|==== %

% - New figure
\newcommand{\incfig}[2][0.7]{%
    \def\svgwidth{#1\columnwidth}
    \import{./figures/}{#2.pdf_tex}
}

\pdfsuppresswarningpagegroup=1

% - New page
\newcommand{\np}{\null\newpage}

% - new homework
\newenvironment{tarea}[3]
    {
        \null\newpage
        \begin{tcolorbox}
            \textbf{\asignatura\ -\ \autor}
            \subsection{\capitalisewords{#1}}
            \label{ssec:#1}
            \begin{flushright}
            \textbf{Desde:}  #2 \
            \textbf{Hasta:}  #3 \
            \end{flushright}
        \end{tcolorbox}

    \begin{enumerate}[{Ejercicio} 1.]
    }
    {
        \end{enumerate}
        \np
    }

% - new observation
\newenvironment{obs}
    {
        \begin{flushleft}
       \textbf{Observación}\
        \line(1,0){200} \
        \end{flushleft}
    }
    {
        \begin{flushright}
        \line(1,0){200}
        \end{flushright}
    }
% ============================================================================= %

\begin{document}

\section{11.08.2020 - Teorema de Probabilidad Total y Teorema de Bayes}
\label{sec:teorema_de_probabilidad_total_y_teorema_de_bayes}

\subsubsection{Definic Partición}
\label{ssec:definicion_de_particion}

Sean \(A_{1},\ldots,A_{n}\) eventos. Decimos que \(A_{1},\ldots,A_{n}\) forman
una partición de \(\Omega\) si \(\bigcup_{i=1}^{\infty}A_{i}=\Omega\) y
\(A_{i}\cap A_{j}=\emptyset \forall i\neq j\) donde \(i,j=1,\ldots,n\)

\subsection{Teorema de Probabilidad Total}
\label{ssec:teorema_de_probabilidad_total}

Sean \(A_{1},\ldots,A_{n}\) eventos disyuntos 2 a 2 que forman una partición de
\(\Omega \). Suponiendo además que \(P\left( A_{i}>0\right) \), \(i=1,\ldots,n\). Entonces \(\forall B\) eventos \(\left( B\in F \right) \).
\[
    P\left( B \right) =\sum_{i=1}^{n} P\left(A_{i}\cap B\right)=\sum_{i=1}^{n} P\left( A_{i} \right) P\left( B|A_{i} \right)
\]
\subsubsection{Demostración}

\begin{figure}[ht]
    \centering
    \incfig{repgraficaprobtotal}
    \caption{Representación Gráfica Probabilidad Total}
    \label{fig:repgraficaprobtotal}
\end{figure}

\[
    B=\bigcup_{i=1}^{n}\left( A_{1}\cap B \right) \to P\left(B\right)=\sum_{i=1}^{n} P\left(A_{i}\cap B \right)=\sum_{i=1}^{n} P\left( A_{i} \right) P\left(B|A_{i}\right)
.\]

\begin{obs}
    \begin{center}
    \textbf{Promedio Ponderado}\\
    P E N D I E N T E
    \end{center}
\end{obs}


\subsubsection{Interpretación}
\label{ssec:interpretacion}

\begin{itemize}
    \item Se puede dividir a \(B\) en los pedazos que caen en casa evento de la
        partición.
    \item \(P\left(B \right) \) es un \textbf{promedio ponderaso} de las
        probabilidades condicionales \(\left( P\left( B|A_{i} \right)  \right)
        \) donde el peso de ponderación está dado por la probabilidad se cada
        evento de la partición \(P\left( A_{i} \right) \).
    \item También se puede interpretar el \textbf{teorema de probabilidad
        total} secuencialmente.
\end{itemize}

\subsection{Ejemplos:}
\label{ssec:ejemplos}

\begin{enumerate}[{Ej.1) }]
    \item Usted está en un torneo de ajedrez. Donde se enfrenta a un oponente al azar.
        \begin{itemize}
            \item La probabilidad de ganarle a la mitad \(G_1\) es de \(0.3\).
            \item La probabilidad de ganarle a un cuarto \(G_2\) es de \(0.4\).
            \item La probabilidad de ganarle a un cuarto \(G_3\) es de \(0.5\).
        \end{itemize}

        ¿Cuál es la probrobabilidad de ganar la partida?

        \begin{itemize}
            \item \(B:\) ganar partida
            \item \(A_1:\) Oponente está en el  \(G_1\)
            \item \(A_2:\) Oponente está en el  \(G_2\)
            \item \(A_2:\) Oponente está en el  \(G_3\)
        \end{itemize}
        \begin{itemize}
            \item \(P\left(A_1\right)=0.5\)
            \item \(P\left(A_2\right)=0.25\)
            \item \(P\left(A_3\right)=0.25\)
        \end{itemize}

        Por lo tanto:

        \[
            P\left(B|A_1\right) = 0.3,P\left(B|A_2\right) =0.4,P\left(B|A_3 \right) =0.5
        ,\]
        \[
            P\left(B\right) = P\left(A_1\cap B\right)+P\left(A_2\cap B\right)+P\left(A_3\cap B\right)
        \]
        \[
            =P\left(A_1\right)P\left(B|A_1\right)+P\left(A_2\right)\left(B|A_2\right)+P\left(A_3\right)P\left(B|A_3\right)
        ,\]
        \[
        =0.375
        .\]

    \item Se tiene un dado just de 4 caras. Si el resultado es \(1\) ó \(2\) se
        tira una segunda vez.

        ¿Cúal es la probabilidad de que la suma total sea mayor o igual a 4?

        \begin{itemize}
            \item \(B:\) suma al menos 4.
            \item \(A_1:\) El primer tiro es 1.
            \item \(A_2:\) El primer tiro es 2.
            \item \(A_3:\) El primer tiro es 3 ó 4.
        \end{itemize}

        \begin{itemize}
            \item \(P\left(A_1\right)=\frac{1}{4}\)
            \item \(P\left(A_2\right)=\frac{1}{4}\)
            \item \(P\left(A_3\right)=\frac{2}{4}=\frac{1}{2}\)
        \end{itemize}

        \begin{itemize}
            \item \(P\left(B|A_1\right)=\frac{1}{2}\)
            \item \(P\left(B|A_2\right)=\frac{3}{4}\)
            \item \(P\left(B|A_3\right)=\frac{1}{2}\)
        \end{itemize}

        \[
            P\left(B\right) = \sum_{i=1}^{3} P\left(B\cap A_{i}\right) = \sum_{i=1}^{3} P\left(A_{i}\right)P\left(B|A_{1}\right) = \frac{1}{4}\left(\frac{1}{2}\right)+\frac{1}{4}\left(\frac{3}{4}\right)+\frac{1}{2}\left( \frac{1}{2} \right)=\frac{9}{16}
        .\]

\end{enumerate}

\subsection{Teorema de Bayes}
\label{ssec:teorema_de_bayes}

Sean \(A_1,\ldots,A_{n}\) eventos que forman una partición de \(\Omega \), talque \(P\left(A_{i}\right)>0\), \(i=1,\ldots,n\). Entonces \(\forall B\in F\) \(\left(B\text{ evento}\right) \) tal que \(P\left(B\right)>0\), se tiene que:

\[
P\left(A_{i}|B\right)=\frac{P\left(B|A_{i}\right)P\left(A_{i}\right)}{P\left(B\right)} = \frac{P\left(A_{i}\right)P\left(B|A_{i}\right)}{\sum_{i=0}^{n} P\left(A_{i}\right)P\left(B|A_{i}\right)}
.\]

\subsubsection{Demostración}

\[
    P\left(A_{i}|B\right) = \frac{P\left(A_{i}\cap B\right)}{P\left(B\right)} = \frac{P\left(A_{i}\right)P\left(B|A_{i}\right)}{P\left(B\right)}
.\]

Que por el \textbf{teorema de probabilidad total}:

\[
= \frac{P\left(A_{i}\right)P\left(B|A_{i}\right)}{\sum_{i=1}^{n}P\left(B|A_{i}\right)P\left(A_{i}\right) }
.\]

\subsubsection{Interpretación}

\begin{figure}[pbth!]
    \centering
    \incfig{bayes}
\end{figure}


Utilizamos el \textbf{teorema de Bayes} cuando le tiene distintaas causs
(excluyentes)) que pueden causar un efecto, y visto el efecto se busca
determinar la probabilidad de las causas.

\subsubsection{Ejercicios:}

\begin{enumerate}[{Ej 1: }]
    \item \( \)
\begin{figure}[ht]
    \centering
    \incfig{recesion-economica}
    \caption{recesion economica}
    \label{fig:recesion-economica}
\end{figure}

    \item Se observa una mancha en la radiografía de un paciente, por lo tanto se tiene:
        \begin{itemize}
            \item \(A_1:\) La mancha es un tumor maligno.
            \item \(A_2:\) La mancha es un tumor benigno.
            \item \(A_3:\) La mancha es otra cosa.
            \item \(B:\) Hay una mancha en la radiografía.
        \end{itemize}

        Tal que:

        \[
        P\left( A_{i}|B \right) , i=1,2,3
        .\]
        \begin{obs}
            Se hacer referencia a \(P\left(A_{i}|B\right)\) como las
            \textbf{probabilidades posteriores} y a \(P\left(A_{i}\right)\)
            como las \textbf{probabilidades previas}.
        \end{obs}

    \item Recordando al ejemplo del radar y el avión:
        \begin{itemize}
            \item \(A:\) Hay avión.
            \item \(B:\) Suena la alarma.
        \end{itemize}

        \begin{itemize}
            \item \(P\left(A\right)=0.05\to P\left(A^{c}\right)=0.95\)
            \item \(P\left(B|A\right)=0.99\)
            \item \(P\left(B|A^{c}\right)=0.1\)
        \end{itemize}

        Luego:

        \[
        P\left(A|B\right) = \frac{P\left(A\right)P\left(B|A\right)}{P\left(B\right)}
    ,\]
    \[
    = \frac{P\left(A\right)P\left(B|A \right)}{P\left(A \right)P\left(B|A\right)+P\left(A^{c}\right)P\left(B|A^{c}\right)}
    ,\]
    \[
        =\frac{0.05\left(0.99\right)}{0.5\left(0.99\right)+0.95\left(0.1\right)}=0.3426
    .\]

    \item Con base en el ejercicio de ajedrez:
        \begin{itemize}
           \item \(P\left(A_1\right)=0.5\)
           \item \(P\left(A_2\right)=0.25\)
           \item \(P\left(A_3\right)=0.25\)
        \end{itemize}

        \begin{itemize}
            \item \(P\left(B|A_1\right)=0.3\)
            \item \(P\left(B|A_2\right)=0.4\)
            \item \(P\left(B|A_3\right)=0.5\)
        \end{itemize}

        Luego:

        \[
            P\left(A_1|B\right) = \frac{P\left(B|A_1\right)P\left(A_1\right)}{\sum_{i=1}^{n} P\left(B|A_{i}\right)P\left(A_{i}\right)}
        ,\]

        \[
        = \frac{0.5\left(0.3\right)}{0.5\left(0.3\right)+0.25\left(0.4\right)+0.25\left(0.5\right)} = 0.4
        .\]

    \item Probabilidad fasos positivos; suponiendo que existe una enfermedad extraña entonces:
        \begin{itemize}
            \item Si el paciente tiene la enfermedad, la prueba sale positiva
                con probabilidad de 0.95.
            \item Si el paciente no tiene la enfermedad, la prueba sale
                negativa con probabilidad de 0.95
        \end{itemize}
        \begin{itemize}
            \item \(A:\) Paciente tiene la enfermedad.
            \item \(B:\) Prueba positiva.
        \end{itemize}
        \begin{itemize}
            \item \(P\left(A\right)=0.001\)
            \item \(P\left(A^{c}\right)=0.999\)
        \end{itemize}
        \begin{itemize}
            \item \(P\left(B|A\right)=0.95 \to P\left(B^{c}|A\right)=0.05\)
            \item \(P\left(B^{c}|A^{c}\right)=0.95\to P\left(B|A^{c}\right)=0.05\)
        \end{itemize}
        \[
        P\left(A|B\right)=\frac{P\left(B|A\right)P\left(A\right) }{P\left(B|A\right)P\left(A\right)+P\left(A^{c}\right)P\left(B|A^{c}\right)}
        ,\]
        \[
        =\frac{0.001\left(0.95\right)}{0.001\left(0.95\right)+0.999\left(0.05\right)}=0.0187
        .\]

\end{enumerate}













\end{document}










