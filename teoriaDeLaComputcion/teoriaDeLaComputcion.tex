\documentclass[a4paper]{book}

\usepackage{import}
\input{./../../commands.tex}
\usepackage{cmands}

\newcommand{\asignatura}{Teoría de la Computacíon}
\newcommand{\autor}{Juan Pablo Sierra Useche}
\newcommand{\semestre}{Tercer}
\newcommand{\universidad}{Universidad Del Rosario}
\newcommand{\escuela}{Escuela de Ingeniería, Ciencia y tecnología}
\newcommand{\carrera}{Matemáticas Aplicadas y Ciencias de la Computación}


\begin{document}

    \titPag
    \tableofcontents

    % ====| B E G I N     O F    W O R K    S P A C E |==== %

    \begin{chapter}{Automatas Finitos Deterministas y no Deterministas}
    \label{chap:automatas_finitos_deterministas_y_no_deterministas}

    \section{04.08.2020 $\mybar$ ¿Que es un Problema?}
    \label{sec:_que_es_un_problema_}

    \subsubsection{Terminología}
    \label{ssec:terminologia}

    En Teoría de la Computación, un problema es una función de un conjunto
    $\mathbb A$ en un conjunto $\mathbb B$.

    Se dice que $f:\mathbb A\to\{0,1\}$ es un problema de decisión.

    \begin{obs}
        Sea $\mathbb A$ un conjunto y $\mathbb B\subseteq\mathbb A$. Resolver el
        problema de decir si x es un elemento de $\mathbb B$ es equivalente a
        counstruir la función.
        \begin{equation*}
        \label{eq:1.1}
            F_{\mathbb B}:\mathbb A\to\{0,1\}
        \end{equation*}
    \end{obs}

    \begin{theorem}{Teorema}
    \label{teo:1}
        Sea $\mathbb A$ un conjunto. Se tiene que existe una función biyectiva
        $F:\varphi\left(\mathbb A\right)\to\{0,1\}^{\mathbb A}$

        \textbf{Demostración}

        \begin{itemize}
                \item Sea $\mathbb B\subseteq\mathbb A$. Definimos $F\left(\mathbb B\right)=f_{\mathbb B}$

                \begin{equation*}
                \label{eq:1.2}
                \begin{split}
                    f_{\mathbb B}(x)=
                    \left\{
                    \begin{array}{11}
                        1&\ si\ x\in\mathbb B\\
                        0&\ si\ x\notin\mathbb B
                    \end{array}
                \end{split}
                \end{equation*}

                \item Ahora resta demostrar que F es \textbf{biyectiva}:

                    \begin{enumerate}[{1) }]
                        \item Sean $\mathbb B,\mathbb C\subseteq\mathbb A$
                            suponga que $f_{\mathbb B}=f_{\mathbb C}$. Observe
                            que:
                            \begin{equation*}
                            \label{eq:1.3}
                                x\in\mathbb B \iff f_{\mathbb
                                B}\left(x\right)=1=f_{\mathbb C}\left(x\right)
                                \iff x\in\mathbb C
                            \end{equation*}

                            Lo que indica que $\mathbb B=\mathbb C$.
                            Demostrando que la función es inyectiva.

                        \item Sea $f\in\{0,1\}^{\mathbb A}$ y existe $\mathbb
                            B\subseteq\mathbb A$ teniendo en cuenta la primera
                            parte de la demostración si $f_{\mathbb B}(x) = 1\
                            $ entonces $f_{\mathbb B^{c}}(x)=0$ y como $\mathbb
                            B^{c}\subseteq\mathbb A$ se puede concluir que para
                            todo $f\in\{0,1\}^{\mathbb A}$ existe un conjunto
                            $\mathbb I$ tal que $f=f_{\mathbb I}$.
                    \end{enumerate}
                Por todo lo anterior ase acaba de demostrar la proposición. $\blacksquare$
        \end{itemize}
    \end{theorem}

    <++>






















    \end{chapter}





    % ====| E N D    O F    W O R K    S P A C E |==== %

    \printbibliography
\end{document}
