\documentclass[a4paper]{book}

% ====| P A C K A G E S |==== %
\usepackage{import}
%\usepackage[backend=biber, style=authoryear-icomp]{biblatex}    %Bibliography stuff
%\input{./../../commands.tex}
\usepackage[margin=1in,includefoot]{geometry}   %Margins
\usepackage[utf8]{inputenc} %Language stuff
%\usepackage [latin1]{inputenc}  %Spanish symbols
\usepackage[spanish]{babel} %Sets document language to spanish
\usepackage{tcolorbox}  %Frame boxes
\usepackage{enumerate}  %Lists options
\usepackage{mfirstuc}   %I use it to capitalize words
\usepackage{graphicx}   %Use Images
\usepackage{listings}   %For displaying code
\usepackage{titlesec}   %Costume titles/sections/...
\usepackage{hyperref}   %Linking options
\usepackage{multicol}	%Use column
\usepackage{amsmath}    %Display equations options
\usepackage{amssymb}    %More symbols
\usepackage{titling}    %Use title variables in other places
\usepackage{xcolor}     %To manage colors
\usepackage{transparent}%For figures
\usepackage{pdfpages}   %For figures
% =========================== %

% ====| P A C K A G E S    S E T T I N G S |==== %
% \addbibresource{/media/jpi/Data/01_Education/04_bibiografia/bibliography.bib} %<--- Bibliography path
% ============================================== %

% ====| P E R S O N A L    C O M M A N D S    &    E N V I R O N M E N T S|==== %

% - New figure
\newcommand{\incfig}[2][1]{%
    \def\svgwidth{#1\columnwidth}
    \import{./figures/}{#2.pdf_tex}
}

\pdfsuppresswarningpagegroup=1

% - New page
\newcommand{\np}{\null\newpage}

% - new homework
\newenvironment{tarea}[3]
    {
        \null\newpage
        \begin{tcolorbox}
            \textbf{\asignatura\ -\ \autor}
            \subsection{\capitalisewords{#1}}
            \label{ssec:#1}
            \begin{flushright}
            \textbf{Desde:}  #2 \
            \textbf{Hasta:}  #3 \
            \end{flushright}
        \end{tcolorbox}

    \begin{enumerate}[{Ejercicio} 1.]
    }
    {
        \end{enumerate}
        \np
    }

% - new observation
\newenvironment{obs}
    {
        \begin{flushleft}
       \textbf{Observación}\
        \line(1,0){200} \
        \end{flushleft}
    }
    {
        \begin{flushright}
        \line(1,0){200}
        \end{flushright}
    }
% ============================================================================= %

\begin{document}
\section{13.08.2020 - Caminos y Ciclos}
\label{sec:caminos_y_ciclos}

\subsection{Caminata}
\label{ssec:caminata}

Una \textbf{caminata (walk)} en un grafo \(G\) es una lista:
\[
v_0e_1v_{1}\ldots e_{k}v_{k}
.\]
 de vértices y aristas tal que para todo \(1\le 1 \le k\) la aarista \(e_{i}\)
 tiene extremos \(v_{i-1}\) y \(v_{i}\).

\subsection{Sendero}
\label{ssec:sendero}

Un \textbf{sendero (trail)} es una caminata sin aristas repetidas.

\subsection{u,v-caminata}
\label{ssec:u_v_caminata}

Una u,v-caminata tiene primer vértice u u último vértice v. estos dos son sus
\textbf{extremos}. Los otros vértices son \textbf{vértices internos}.
Análogamente se define un u,v-\textbf{sendero}.

\subsection{Camino}
\label{ssec:camino}

Un \textbf{camino} es un sendero sin vértices repetidos. Análogamente se define
un u,v-\textbf{camino}.

\subsection{Circuito}
\label{ssec:circuito}

Una caminata es \textbf{cerrada} si sus extremos son iguales. Un \textbf{circuito} es un sendero cerrado.

\subsection{Ciclo}
\label{ssec:ciclo}

Un \textbf{ciclo} es un camino cerrado.

\subsection{Longitud}
\label{ssec:longitud}

La \textbf{longitud} de una caminata, sendero, camino o ciclo es el número de aristas
que la conforman.

\begin{obs}
    \begin{itemize}
        \item Un bucle es un ciclo de longitud 1.
        \item Un ciclo de longitud 2 genera aristas paralelas.
        \item Se \(G\) es un grafo simple la cominata, sendero, circuito,
            camino o ciclo únicamente elista los vértices.
    \end{itemize}
\end{obs}

\begin{obs}
    ¿Si se sigue un u,v-camino y un v,w-camino, el resultado es un u,w-camino?
\begin{figure}[ht]
    \centering
    \incfig[0.5]{grafo}
\end{figure}
\begin{itemize}
    \item \(a-x-v\) y \(v-y-u-x-b\)
\end{itemize}
\end{obs}

\subsection{Lema}

Cada u,v-caminata contiene un u,v-camino.

\subsubsection{Demostración (Inducción fuerte sobre longitud de la caminata)}
\label{ssec:demostracion_induccion_fuerte_sobre_longitud_de_la_caminata_}

\begin{enumerate}[{P}]
    \item Base: u,v caminata de longitud 1, entonces es u,v-camino.
    \item Inductivo: Cualquier ccaminata con lingitud menor o igual a \(n\) contiene un camino. Sea u,v-caminata de longitud \(n+1\). Si ya es un camino perfecto. Pero si no lo es, existe al menos un vertice interno n¿\(w\) que se repite, luego considere la u,v-caminata excluyendo las aristas entre las dos apariciones u,vcaminata excluyendo las arista entre lasss dos apariciones de w. Está nueva caminata tiene longitud menor o igual que \(\left(n+1\right)=n\) \ldots \textbf{S C R E E N S H O T}
\end{enumerate}

\subsection{Grafo Conexo}
\label{ssec:grafo_conexo}

\begin{itemize}
    \item Un grafo \(G\) es \textbf{conexo} si existe un u,v-camino entre cada
        par \(uv\in V\left( G \right)\). En otro caso es \textbf{disconexo}.
    \item  Si \(G\) tiene un u,v-camino entonces u está \textbf{conectado} con
        v.
\end{itemize}

\subsection{Relación de conexión}
\label{ssec:relacion_de_conexion}

La \textbf{relación de conexión} en \(V\left(G\right)\) consiste en todos los pares ordenados \(\left(u,v \right)\) tales que \(u\) está conectado con \(v\):
\[
uRv\text{ sii existe un, u,v-camino.}
.\]




























\end{document}
