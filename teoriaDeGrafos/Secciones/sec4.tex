\documentclass[a4paper,dvipsnames]{book}

% ====| P A C K A G E S |==== %
\usepackage{import}
\usepackage{nicematrix}
%\usepackage[backend=biber, style=authoryear-icomp]{biblatex}    %Bibliography stuff
%\input{./../../commands.tex}
\usepackage[margin=1in,includefoot]{geometry}   %Margins
\usepackage[utf8]{inputenc} %Language stuff
%\usepackage [latin1]{inputenc}  %Spanish symbols
\usepackage[spanish]{babel} %Sets document language to spanish
\usepackage{tcolorbox}  %Frame boxes
\usepackage{enumerate}  %Lists options
\usepackage{mfirstuc}   %I use it to capitalize words
\usepackage{graphicx}   %Use Images
\usepackage{listings}   %For displaying code
\usepackage{titlesec}   %Costume titles/sections/...
\usepackage{hyperref}   %Linking options
\usepackage{multicol}	%Use column
\usepackage{amsmath}    %Display equations options
\usepackage{amssymb}    %More symbols
\usepackage{titling}    %Use title variables in other places
\usepackage{xcolor}     %To manage colors
\usepackage{transparent}%For figures
\usepackage{pdfpages}   %For figures
% =========================== %

% ====| P A C K A G E S    S E T T I N G S |==== %
% \addbibresource{/media/jpi/Data/01_Education/04_bibiografia/bibliography.bib} %<--- Bibliography path
% ============================================== %

% ====| P E R S O N A L    C O M M A N D S    &    E N V I R O N M E N T S|==== %

% - New figure
\newcommand{\incfig}[2][1]{%
    \def\svgwidth{#1\columnwidth}
    \import{./figures/}{#2.pdf_tex}
}

\pdfsuppresswarningpagegroup=1

% - New page
\newcommand{\np}{\null\newpage}

% - new homework
\newenvironment{tarea}[3]
    {
        \null\newpage
        \begin{tcolorbox}
            \textbf{\asignatura\ -\ \autor}
            \subsection{\capitalisewords{#1}}
            \label{ssec:#1}
            \begin{flushright}
            \textbf{Desde:}  #2 \
            \textbf{Hasta:}  #3 \
            \end{flushright}
        \end{tcolorbox}

    \begin{enumerate}[{Ejercicio} 1.]
    }
    {
        \end{enumerate}
        \np
    }

% - new observation
\newenvironment{obs}
    {
        \begin{flushleft}
       \textbf{Observación}\
        \line(1,0){200} \
        \end{flushleft}
    }
    {
        \begin{flushright}
        \line(1,0){200}
        \end{flushright}
    }
% ============================================================================= %

\begin{document}

\section{18.08.2020 - Continuación de lo anterior}
\label{sec:continuacion_de_lo_anterior}

\begin{obs}
    Las clases de equivalencia de la relación de conexión en
    \(V\left(G\right)\) son los conjuntos de vértices de las componentes de
    \(G\).
\end{obs}

\subsection{Proposición}

Todo grafo con \(n\) vértices y \(k\) aristas tiene al menos \(n-k\) componentes

\begin{obs}
    \begin{itemize}
        \item Las componentes de un grafo son disyuntas y no comparten
            vértices. Si se agrega una arista con extremos en distintas
            componentes, estas se combinan en una nueva componente.
        \item Agregar una arista a \(G\) disminuye el número de componentes en 1 ó 0.
        \item Quitar una arista a \(G\) aumenta el número de componentes en 1 ó 0.
    \end{itemize}
\end{obs}

\subsubsection{Demostración}
\begin{center}
    \textbf{M I N    2 5    D E    L A    C L A S E}
\end{center}

\subsection{Arista de corte-vértice de corte}
\label{ssec:arista_de_corte_vertice_de_corte}

Una \textbf{arista de corte} o un \textbf{vértice de corte} es unna arista o
vértice cuya eliminación incrementa el número de componentes.

\begin{obs}
    \begin{itemize}
        \item  Al eliminaar un vértice se  deben eliminar todas las aristas incidentes.
        \item  El número de componentes podría aumetar en más de una. Como el caso de \(K_{1,m}\)
    \end{itemize}
\end{obs}
\begin{figure}[ht]
    \centering
    \incfig[0.5]{k1m}
    \caption{k1m}
    \label{fig:k1m}
\end{figure}

\subsection{Definición}

\begin{itemize}
    \item \(G-e\): Subgrafo que se obtiene al eliminar la arista \(e\).
    \item \(G-v\): Subgrafo que se obtiene al eliminar la arista \(e\).
    \item \(G-M\): Subgrafo que se obtiene al eliminar el conjunto de aristas \(M\).
    \item \(G-S\): Subgrafo que se obtiene al eliminar el conjunto de vértices \(S\).
\end{itemize}

\subsection{Grafo Inducido}
\label{ssec:grafo_inducido}

Un \textbf{subgrafo} es un subgrafo que se obtiene al eliminar un conjunto de vértices. Se escribe \(G\left[T\right]\) para \(G-\overline{T}\) donde \(\overline{T}=V\left(G\right)-T\), este es el subgrafo iducido por \(T\).

\subsection{Ejercicio}

Un conjunto \(S\) de vértices es independiente sii \(G\left[S\right]\) no tiene
aristas.

\subsection{Teorema}

Una arista \(e\) es una arista de corte sii no pertenece a ningún ciclo.


\subsection{Lema}

Toda caminata cerrada impar contiene un ciclo impar.

\subsection{Ejercicio}

¿Toda caminata cerrada par contiene un ciclo par?  {\color{Violet} \textbf{N \ \ O}}

\subsection{Teorema de (König)}
\label{ssec:teorema_de_konig_}

Un grafo \(G\) es bibpartito sii no tiene ciclos impares.

\subsection{Unión}
\label{ssec:union}

La unión de grafos \(G_1,G_2,\ldots,G_{k}\) notada \(G_1\cup G_2\cup \ldots\cup
G_{k}\) es el grafo  \(G\) con conjuntos de vértices
\[ V\left(G\right) =\bigcup_{i=1}^{k}V\left(G_{i}\right) ,\]
y conjunto de aristas
\[E\left(G\right)=\bigcup_{i=1}^{k}E\left(G_{i}\right).\]

\subsubsection{Ejemplo}
\(K_4\) es la unión de dos 4-ciclos bipartitos.

\subsection{Sendero Euleriano - Circuito Euleriano}
\label{ssec:sendero_euleriano_circuito_euleriano}
\begin{itemize}
    \item Un \textbf{Sendero Euleriano} en un grafo \(G\) es un sendero que
        contiene todas las aristas de  \(G\).
    \item  Un \textbf{Circuito Euleriano} en un grfo \(G\) es un circuito que
        contiene todas las aristas de \(G\).
\end{itemize}

\subsection{Grafo Euleriano}
\label{ssec:grafo_euleriano}
Un \textbf{grafo} \(G\) es \textbf{Euleriano} si tiene un circuito Euleriano.

































\end{document}
