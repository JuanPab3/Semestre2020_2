\documentclass[a4paper]{book}

% ====| P A C K A G E S |==== %
\usepackage{import}
%\usepackage[backend=biber, style=authoryear-icomp]{biblatex}    %Bibliography stuff
%\input{./../../commands.tex}
\usepackage[margin=1in,includefoot]{geometry}   %Margins
\usepackage[utf8]{inputenc} %Language stuff
%\usepackage [latin1]{inputenc}  %Spanish symbols
\usepackage[spanish]{babel} %Sets document language to spanish
\usepackage{tcolorbox}  %Frame boxes
\usepackage{enumerate}  %Lists options
\usepackage{mfirstuc}   %I use it to capitalize words
\usepackage{graphicx}   %Use Images
\usepackage{listings}   %For displaying code
\usepackage{titlesec}   %Costume titles/sections/...
\usepackage{hyperref}   %Linking options
\usepackage{multicol}	%Use column
\usepackage{amsmath}    %Display equations options
\usepackage{amssymb}    %More symbols
\usepackage{titling}    %Use title variables in other places
\usepackage{xcolor}     %To manage colors
\usepackage{transparent}%For figures
\usepackage{pdfpages}   %For figures
% =========================== %

% ====| P A C K A G E S    S E T T I N G S |==== %
% \addbibresource{/media/jpi/Data/01_Education/04_bibiografia/bibliography.bib} %<--- Bibliography path
% ============================================== %

% ====| P E R S O N A L    C O M M A N D S    &    E N V I R O N M E N T S|==== %

% - New figure
\newcommand{\incfig}[2][1]{%
    \def\svgwidth{#1\columnwidth}
    \import{./figures/}{#2.pdf_tex}
}

\pdfsuppresswarningpagegroup=1

% - New page
\newcommand{\np}{\null\newpage}

% - new homework
\newenvironment{tarea}[3]
    {
        \null\newpage
        \begin{tcolorbox}
            \textbf{\asignatura\ -\ \autor}
            \subsection{\capitalisewords{#1}}
            \label{ssec:#1}
            \begin{flushright}
            \textbf{Desde:}  #2 \
            \textbf{Hasta:}  #3 \
            \end{flushright}
        \end{tcolorbox}

    \begin{enumerate}[{Ejercicio} 1.]
    }
    {
        \end{enumerate}
        \np
    }

% - new observation
\newenvironment{obs}
    {
        \begin{flushleft}
       \textbf{Observación}\
        \line(1,0){200} \
        \end{flushleft}
    }
    {
        \begin{flushright}
        \line(1,0){200}
        \end{flushright}
    }
% ============================================================================= %

\begin{document}

\section{06.08.2020 - Conceptos Fundamentales}
\label{sec:conceptos_fundamentales}

\subsection{Grafo}
Un \textbf{grafo} \(G\) es un terna que consiste en un conjunto de vértices
\(V\left(G\right)\), un conjunto de aristas \(E\left(G\right)\) y una relación
que asocia a cada arista un par de vértices no necesariamente distinos.

\subsection{Relación de adyacencia}
\begin{itemize}
    \item Dos vértices \(y\text{ y }v\) son \textbf{adyacentes} (vecinos) si \(u\text{ y }v\) son los extremos de una aeista \(e\).
    \item  Si \(u\) es adyacente a \(v\) se nota: \(u\leftrightarrow v\).
\end{itemize}

\begin{obs}
    \begin{itemize}
        \item Un \textbf{bucle} o \textbf{lazo} es una arista cuyos extremos son iguales.
        \item Dos o más aristas son \textbf{múltiples} o \textbf{paralelas} si
            tiene el mismo par de extremos.
    \end{itemize}
\end{obs}

\subsection{Grafo simple}
Un grafo simple \(G=\left( V,E \right) \) es un grafo sin bucles ni aristas
múltiples, donde \(E\) es un conjunto de pares no ordenados de vértices.

\subsection{Grafo finito}
 Un grafo es  \textbf{finito} si \(V\left(G\right)\text{ y } E
 \left(G\right) \) son conjuntos finitos.

\subsection{Grafo nulo}
El \textbf{grafo nul} es el grafo \(G\), tal que \(V\left( G \right)
=\emptyset \text{ y } E\left(G\right)=\emptyset\).

\subsection{Grafos complemetarios}
El complemeto \(\overline{G}\) de un grafo \(G\), es el grafo simple con
conjunto de vértices \(V\left(G\right)\) definido por: \(uv\in
E\left(\overline{G}\right) \text{ sii } \not\in E\left(G\right)\).

\subsection{Clique}
Un \textbf{clique} es un conjunto de vértices \textit{2 a 2}\footnote{Todo par
de vértices en el conjunto es adyacente.}

\subsection{Conjunto independiente}
Un \textbf{conjunto independiente} es un conjunto de vértices no adyacentes \textit{2 a 2}

\begin{obs}
    \(W\) es un clique en \(G\) sii \(W\) es un conjunto independiente en
    \(\overline{G}\).
\end{obs}

\subsection{Grafo bipartito}
Un grafo \(G\) es \textbf{bipartito} si \(V\left( G \right) \) es la unión de
dos conjuntos disyuntos independientes denominados conjuntos partitos de \(G\).
\begin{figure}[ht]
    \centering
    \incfig{bipartito}
    \caption{bipartito}
    \label{fig:bipartito}
\end{figure}

\subsection{Grafo k-partito}
Un grafo es \textbf{\textit{k}-partito} si \(V\left(G\right)\) es la unión de
\(k\) conjuntos disyuntos independientes.

\subsection{Número cromático}
El \textbf{número cromático} de un grafo \(G\), \(X\left(G \right)\), es el
mínimo número de colores necesarios para etiquetar los vértices de \(G\) de tal
manera que vértices adyacentes reciban colores distintos.

\subsubsection{Teorema}
Un grafo es \(G\) es k-partito sii  \(X\left(G\right) \le k\).

\subsection{Subgrafo}
Un \textbf{subgrafo} de un grafo \(G\) es un grafo \(H\) tal que:
\begin{enumerate}
    \item \(V\left(H\right)\subseteq V\left(G\right) \)
    \item \(E\left(H\right)\subseteq E\left(G\right)\)
\end{enumerate}
Y la asignación de extremos a las aristas en \(H\) es la misma que en \(G\). Se
nota \(H\subseteq G\).

\subsection{Camino}
Un \textbf{camino} es un grafo simple cuyos vértices pueden ordenarse en una
lista de tal manera que dos vértices son adyacentes sii son consecutivos en la
lista.

\subsection{Ciclo}
Un \textbf{ciclo} es un grafo simple con el mismo númer de vértices y aristas
cuyos vpertices pueden ubicarse alrededor de un cirdulo de tal manera que dos
vértices son adyacentes sii aparecen de manera consecutiva sobre el circulo.

\subsection{Conexidad}
Un grafo \(G\) es \textbf{conexo} si cada par de vértices en \(G\) pertenece a
un camino, de lo contrario, \(G\) es disconexo.

\subsection{Matriz de adyacencia y Matriz di incidencia}
Sea \(G\) un grafo sin bucles con \(V\left(G\right)=\left\{v_1,v_2,\ldots,v_{n}
\right\}\) y  \(E\left(G\right)=\left\{e_1,e_2,\ldots,e_{n}\right\}\).
\begin{itemize}
    \item La \textbf{matriz de adyacencia} de \(G\) es la matriz \(n\times n,
        A\left(G\right),\) definida por:
        \[
        a_{ij}:= \text{ número de aristas  en }G \text{ con extremos }
        \left\{v_{i},v_{j}\right\}
        \]
    \item La \textbf{matriz de incidencia} de \(G\) es la matriz \(n\times m,\
        M\left(G\right),\) definida por:
        \[m_{ij} := \left\{
            \begin{array}{ll}
                1  & \text{si }v_{i} \text{ es extremo de } e_{j}. \\
                0  & \text{en cualquier otro caso.}
            \end{array}
        \right.
        \]
\end{itemize}

\begin{obs}
    \begin{enumerate}
        \item \(A\left( G \right)\) depende del orden de los vértices.
        \item Toda matriz de adyacencia es simpetrica.
        \item Si \(G\) es simple, la matriz de adyacencia tiene entradas \(1\)
            o \(0\) con \(0's\) en la diagonal.
    \end{enumerate}
\end{obs}

\subsection{Grado de un vértice (1)}
El \textbf{grado} de un vértice \(v\) es la suma de las entradas en la fila \(v\) en \(A\left( G \right)\) o \(M\left( G \right) \). {\color{teal} Se nota \(d\left( v \right) \) }.

\begin{obs}
    \begin{enumerate}
        \item La matriz de adyacencia tambié se usa para representar grafos con
            bucles. Un bucle en el vértice \(v_{i}\) es representado por un
            \(1\) en la posición \(\left( i,i \right) \) de la matriz de
            adyacencia.
        \item {\color{orange} En esté caso no se cumple la propiedad del grado de un vértice.}
    \end{enumerate}
\end{obs}

\subsection{Grado de un vértice (2)}
El \textbf{grado} de un vértice \(v\) es el número de aristas iniciales en
\(v\). {\color{teal}Un bucle en \(v\) aporta dos unidades al grado de  \(v\)}.
\begin{figure}[ht]
    \centering
    \incfig[0.7]{otro-grafo}
\end{figure}

\end{document}
