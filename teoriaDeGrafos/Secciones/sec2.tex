\documentclass[a4paper]{book}

% ====| P A C K A G E S |==== %
\usepackage{import}
%\usepackage[backend=biber, style=authoryear-icomp]{biblatex}    %Bibliography stuff
%\ProvidesPackage{cmands}
% ====| P A C K A G E S |==== %
\usepackage{import}
%\usepackage[backend=biber, style=authoryear-icomp]{biblatex}    %Bibliography stuff
%\ProvidesPackage{cmands}
% ====| P A C K A G E S |==== %
\usepackage{import}
%\usepackage[backend=biber, style=authoryear-icomp]{biblatex}    %Bibliography stuff
%\ProvidesPackage{cmands}
% ====| P A C K A G E S |==== %
\usepackage{import}
%\usepackage[backend=biber, style=authoryear-icomp]{biblatex}    %Bibliography stuff
%\input{./../../commands.tex}
\usepackage[margin=1in,includefoot]{geometry}   %Margins
\usepackage[utf8]{inputenc} %Language stuff
\usepackage [latin1]{inputenc}  %Spanish symbols
\usepackage[spanish]{babel} %Sets document language to spanish
\usepackage{tcolorbox}  %Frame boxes
\usepackage{enumerate}  %Lists options
\usepackage{mfirstuc}   %I use it to capitalize words
\usepackage{graphicx}   %Use Images
\usepackage{listings}   %For displaying code
\usepackage{titlesec}   %Costume titles/sections/...
\usepackage{hyperref}   %Linking options
\usepackage{multicol}	%Use column
\usepackage{amsmath}    %Display equations options
\usepackage{amssymb}    %More symbols
\usepackage{titling}    %Use title variables in other places
\usepackage{xcolor}     %To manage colors
\usepackage{transparent}%For figures
\usepackage{pdfpages}   %For figures
% =========================== %

% ====| P A C K A G E S    S E T T I N G S |==== %
% \addbibresource{/media/jpi/Data/01_Education/04_bibiografia/bibliography.bib} %<--- Bibliography path
\setlength{\columnsep}{1cm}
\hypersetup{
    colorlinks,
    citecolor=black,
    filecolor=black,
    linkcolor=black,
    urlcolor=black
}
% ============================================== %

% ====| P E R S O N A L    C O M M A N D S    &    E N V I R O N M E N T S|==== %

% - New figure
\newcommand{\incfig}[2][1]{%
    \def\svgwidth{#1\columnwidth}
    \import{./figures/}{#2.pdf_tex}
}

\pdfsuppresswarningpagegroup=1

% - New page
\newcommand{\np}{\null\newpage}

% - vertical thic line
\newcommand\mybar{\kern1pt\rule[-\dp\strutbox]{.8pt}{\baselineskip}\kern1pt}

% - new homework
\newenvironment{tarea}[3]
    {
        \null\newpage
        \begin{tcolorbox}
            \textbf{\asignatura\ -\ \autor}
            \subsection{\capitalisewords{#1}}
            \label{ssec:#1}
            \begin{flushright}
            \textbf{Desde:}  #2 \\
            \textbf{Hasta:}  #3 \\
            \end{flushright}
        \end{tcolorbox}

    \begin{enumerate}[{Ejercicio} 1.]
    }
    {
        \end{enumerate}
        \np
    }



% ============================================================================= %


% ====| H E A T H E R S    S E T T I N G S |==== %

\titleformat{\chapter}[display]
    {\normalfont\huge\bfseries\raggedleft}{\chaptertitlename\ \thechapter}
    {20pt}{\Huge}
\titleformat{\section}[display]
    {\Large\bfseries}{}
    {0em}{}[\titlerule]


\newcommand{\titPag}{
    \begin{titlepage}
        \begin{flushright}
            \textsc{\large {\semestre\ Semestre}}\\
            \line(1,0){450} \\
            [0.635cm]
            \huge{\bfseries \asignatura} \\
            [0.2cm]
            \line(1,0){350} \\
            \LARGE{\bfseries \autor} \\
            [16.25cm]
        \end{flushright}
        \begin{flushright}
        \textsc{
            \universidad \\
            [0.1cm]
            \escuela \\
            [0.1cm]
            \carrera
        }
        \end{flushright}
    \end{titlepage}
}
% ============================================== %

% ====| C O D E    I N    F I L E S    S E T T I N G S |==== %
\definecolor{codegreen}{rgb}{0,0.6,0}
\definecolor{codegray}{rgb}{0.5,0.5,0.5}
\definecolor{codepurple}{rgb}{0.58,0,0.82}
\definecolor{backcolour}{rgb}{0.95,0.95,0.92}

\lstdefinestyle{mystyle}{
    backgroundcolor=\color{backcolour},
    commentstyle=\color{codegreen},
    keywordstyle=\color{magenta},
    numberstyle=\tiny\color{codegray},
    stringstyle=\color{codepurple},
    basicstyle=\ttfamily\footnotesize,
    breakatwhitespace=false,
    breaklines=true,
    captionpos=b,
    keepspaces=true,
    numbers=left,
    numbersep=5pt,
    showspaces=false,
    showstringspaces=false,
    showtabs=false,
    tabsize=2
}

\lstset{style=mystyle}
% ========================================================== %

\usepackage[margin=1in,includefoot]{geometry}   %Margins
\usepackage[utf8]{inputenc} %Language stuff
\usepackage [latin1]{inputenc}  %Spanish symbols
\usepackage[spanish]{babel} %Sets document language to spanish
\usepackage{tcolorbox}  %Frame boxes
\usepackage{enumerate}  %Lists options
\usepackage{mfirstuc}   %I use it to capitalize words
\usepackage{graphicx}   %Use Images
\usepackage{listings}   %For displaying code
\usepackage{titlesec}   %Costume titles/sections/...
\usepackage{hyperref}   %Linking options
\usepackage{multicol}	%Use column
\usepackage{amsmath}    %Display equations options
\usepackage{amssymb}    %More symbols
\usepackage{titling}    %Use title variables in other places
\usepackage{xcolor}     %To manage colors
\usepackage{transparent}%For figures
\usepackage{pdfpages}   %For figures
% =========================== %

% ====| P A C K A G E S    S E T T I N G S |==== %
% \addbibresource{/media/jpi/Data/01_Education/04_bibiografia/bibliography.bib} %<--- Bibliography path
\setlength{\columnsep}{1cm}
\hypersetup{
    colorlinks,
    citecolor=black,
    filecolor=black,
    linkcolor=black,
    urlcolor=black
}
% ============================================== %

% ====| P E R S O N A L    C O M M A N D S    &    E N V I R O N M E N T S|==== %

% - New figure
\newcommand{\incfig}[2][1]{%
    \def\svgwidth{#1\columnwidth}
    \import{./figures/}{#2.pdf_tex}
}

\pdfsuppresswarningpagegroup=1

% - New page
\newcommand{\np}{\null\newpage}

% - vertical thic line
\newcommand\mybar{\kern1pt\rule[-\dp\strutbox]{.8pt}{\baselineskip}\kern1pt}

% - new homework
\newenvironment{tarea}[3]
    {
        \null\newpage
        \begin{tcolorbox}
            \textbf{\asignatura\ -\ \autor}
            \subsection{\capitalisewords{#1}}
            \label{ssec:#1}
            \begin{flushright}
            \textbf{Desde:}  #2 \\
            \textbf{Hasta:}  #3 \\
            \end{flushright}
        \end{tcolorbox}

    \begin{enumerate}[{Ejercicio} 1.]
    }
    {
        \end{enumerate}
        \np
    }



% ============================================================================= %


% ====| H E A T H E R S    S E T T I N G S |==== %

\titleformat{\chapter}[display]
    {\normalfont\huge\bfseries\raggedleft}{\chaptertitlename\ \thechapter}
    {20pt}{\Huge}
\titleformat{\section}[display]
    {\Large\bfseries}{}
    {0em}{}[\titlerule]


\newcommand{\titPag}{
    \begin{titlepage}
        \begin{flushright}
            \textsc{\large {\semestre\ Semestre}}\\
            \line(1,0){450} \\
            [0.635cm]
            \huge{\bfseries \asignatura} \\
            [0.2cm]
            \line(1,0){350} \\
            \LARGE{\bfseries \autor} \\
            [16.25cm]
        \end{flushright}
        \begin{flushright}
        \textsc{
            \universidad \\
            [0.1cm]
            \escuela \\
            [0.1cm]
            \carrera
        }
        \end{flushright}
    \end{titlepage}
}
% ============================================== %

% ====| C O D E    I N    F I L E S    S E T T I N G S |==== %
\definecolor{codegreen}{rgb}{0,0.6,0}
\definecolor{codegray}{rgb}{0.5,0.5,0.5}
\definecolor{codepurple}{rgb}{0.58,0,0.82}
\definecolor{backcolour}{rgb}{0.95,0.95,0.92}

\lstdefinestyle{mystyle}{
    backgroundcolor=\color{backcolour},
    commentstyle=\color{codegreen},
    keywordstyle=\color{magenta},
    numberstyle=\tiny\color{codegray},
    stringstyle=\color{codepurple},
    basicstyle=\ttfamily\footnotesize,
    breakatwhitespace=false,
    breaklines=true,
    captionpos=b,
    keepspaces=true,
    numbers=left,
    numbersep=5pt,
    showspaces=false,
    showstringspaces=false,
    showtabs=false,
    tabsize=2
}

\lstset{style=mystyle}
% ========================================================== %

\usepackage[margin=1in,includefoot]{geometry}   %Margins
\usepackage[utf8]{inputenc} %Language stuff
\usepackage [latin1]{inputenc}  %Spanish symbols
\usepackage[spanish]{babel} %Sets document language to spanish
\usepackage{tcolorbox}  %Frame boxes
\usepackage{enumerate}  %Lists options
\usepackage{mfirstuc}   %I use it to capitalize words
\usepackage{graphicx}   %Use Images
\usepackage{listings}   %For displaying code
\usepackage{titlesec}   %Costume titles/sections/...
\usepackage{hyperref}   %Linking options
\usepackage{multicol}	%Use column
\usepackage{amsmath}    %Display equations options
\usepackage{amssymb}    %More symbols
\usepackage{titling}    %Use title variables in other places
\usepackage{xcolor}     %To manage colors
\usepackage{transparent}%For figures
\usepackage{pdfpages}   %For figures
% =========================== %

% ====| P A C K A G E S    S E T T I N G S |==== %
% \addbibresource{/media/jpi/Data/01_Education/04_bibiografia/bibliography.bib} %<--- Bibliography path
\setlength{\columnsep}{1cm}
\hypersetup{
    colorlinks,
    citecolor=black,
    filecolor=black,
    linkcolor=black,
    urlcolor=black
}
% ============================================== %

% ====| P E R S O N A L    C O M M A N D S    &    E N V I R O N M E N T S|==== %

% - New figure
\newcommand{\incfig}[2][1]{%
    \def\svgwidth{#1\columnwidth}
    \import{./figures/}{#2.pdf_tex}
}

\pdfsuppresswarningpagegroup=1

% - New page
\newcommand{\np}{\null\newpage}

% - vertical thic line
\newcommand\mybar{\kern1pt\rule[-\dp\strutbox]{.8pt}{\baselineskip}\kern1pt}

% - new homework
\newenvironment{tarea}[3]
    {
        \null\newpage
        \begin{tcolorbox}
            \textbf{\asignatura\ -\ \autor}
            \subsection{\capitalisewords{#1}}
            \label{ssec:#1}
            \begin{flushright}
            \textbf{Desde:}  #2 \\
            \textbf{Hasta:}  #3 \\
            \end{flushright}
        \end{tcolorbox}

    \begin{enumerate}[{Ejercicio} 1.]
    }
    {
        \end{enumerate}
        \np
    }



% ============================================================================= %


% ====| H E A T H E R S    S E T T I N G S |==== %

\titleformat{\chapter}[display]
    {\normalfont\huge\bfseries\raggedleft}{\chaptertitlename\ \thechapter}
    {20pt}{\Huge}
\titleformat{\section}[display]
    {\Large\bfseries}{}
    {0em}{}[\titlerule]


\newcommand{\titPag}{
    \begin{titlepage}
        \begin{flushright}
            \textsc{\large {\semestre\ Semestre}}\\
            \line(1,0){450} \\
            [0.635cm]
            \huge{\bfseries \asignatura} \\
            [0.2cm]
            \line(1,0){350} \\
            \LARGE{\bfseries \autor} \\
            [16.25cm]
        \end{flushright}
        \begin{flushright}
        \textsc{
            \universidad \\
            [0.1cm]
            \escuela \\
            [0.1cm]
            \carrera
        }
        \end{flushright}
    \end{titlepage}
}
% ============================================== %

% ====| C O D E    I N    F I L E S    S E T T I N G S |==== %
\definecolor{codegreen}{rgb}{0,0.6,0}
\definecolor{codegray}{rgb}{0.5,0.5,0.5}
\definecolor{codepurple}{rgb}{0.58,0,0.82}
\definecolor{backcolour}{rgb}{0.95,0.95,0.92}

\lstdefinestyle{mystyle}{
    backgroundcolor=\color{backcolour},
    commentstyle=\color{codegreen},
    keywordstyle=\color{magenta},
    numberstyle=\tiny\color{codegray},
    stringstyle=\color{codepurple},
    basicstyle=\ttfamily\footnotesize,
    breakatwhitespace=false,
    breaklines=true,
    captionpos=b,
    keepspaces=true,
    numbers=left,
    numbersep=5pt,
    showspaces=false,
    showstringspaces=false,
    showtabs=false,
    tabsize=2
}

\lstset{style=mystyle}
% ========================================================== %

\usepackage{nicematrix}
\usepackage[margin=1in,includefoot]{geometry}   %Margins
\usepackage[utf8]{inputenc} %Language stuff
%\usepackage [latin1]{inputenc}  %Spanish symbols
\usepackage[spanish]{babel} %Sets document language to spanish
\usepackage{tcolorbox}  %Frame boxes
\usepackage{enumerate}  %Lists options
\usepackage{mfirstuc}   %I use it to capitalize words
\usepackage{graphicx}   %Use Images
\usepackage{listings}   %For displaying code
\usepackage{titlesec}   %Costume titles/sections/...
\usepackage{hyperref}   %Linking options
\usepackage{multicol}	%Use column
\usepackage{amsmath}    %Display equations options
\usepackage{amssymb}    %More symbols
\usepackage{titling}    %Use title variables in other places
\usepackage{xcolor}     %To manage colors
\usepackage{transparent}%For figures
\usepackage{pdfpages}   %For figures
% =========================== %

% ====| P A C K A G E S    S E T T I N G S |==== %
% \addbibresource{/media/jpi/Data/01_Education/04_bibiografia/bibliography.bib} %<--- Bibliography path
% ============================================== %

% ====| P E R S O N A L    C O M M A N D S    &    E N V I R O N M E N T S|==== %

% - New figure
\newcommand{\incfig}[2][1]{%
    \def\svgwidth{#1\columnwidth}
    \import{./figures/}{#2.pdf_tex}
}

\pdfsuppresswarningpagegroup=1

% - New page
\newcommand{\np}{\null\newpage}

% - new homework
\newenvironment{tarea}[3]
    {
        \null\newpage
        \begin{tcolorbox}
            \textbf{\asignatura\ -\ \autor}
            \subsection{\capitalisewords{#1}}
            \label{ssec:#1}
            \begin{flushright}
            \textbf{Desde:}  #2 \\
            \textbf{Hasta:}  #3 \\
            \end{flushright}
        \end{tcolorbox}

    \begin{enumerate}[{Ejercicio} 1.]
    }
    {
        \end{enumerate}
        \np
    }

% - new observation
\newenvironment{obs}
    {
        \begin{flushleft}
       \textbf{Observación}\\
        \line(1,0){200} \\
        \end{flushleft}
    }
    {
        \begin{flushright}
        \line(1,0){200}
        \end{flushright}
    }


% ============================================================================= %



\begin{document}

\section{11.08.2020 - Isomorfismos de Grafos}
\label{sec:isomorfismos_de_grafos}

\subsection{Definición Isomorfismos}
\label{ssec:definicion_isomorfismos}

Sean \(G\ y\ H\) grafos simples. Un \textbf{isomorfismo} de \(G\) a \(H\) es
una función biyectiva \(f:V\left(G\right)\to V\left(H\right) \) tal que \(uv\in
E\left(G\right)\iff f\left(u\right) f\left(v\right) \in E\left(H\right) \)

\subsection{Grafos Isomorfos}
\label{ssec:grafos_isomorfos}

\(G\) es \textbf{isomorfo} a \(H\) si existe un isomorfismo de \(G\ a\ H\). Se nota \(G\cong H\).

\begin{figure}[ht]
    \centering
    \incfig{dos-grafos-isomorfos}
    \caption{dos grafos isomorfos}
    \label{fig:dos-grafos-isomorfos}
\end{figure}
\begin{obs}
    \textit{Encontrar isomorfismos} se puede demostrar presentando la \textbf{matriz de adyacencia} con los vertices ordenados de tal forma que se pueda distinguir la igualdad entre las dos matrices a evaluar.
\end{obs}

\subsection{Invariante}
\label{ssec:invariante}

Un \textbf{invariante} de un grafo es una propiedad \(P\) preservada por
isomorfismos. De forma más precisa, \(P\) es una invariante si siempre que
\(G\cong H\):

\begin{center}
    Si \(G\) satisfacea \(P\) entonces \(H\) satisface \(P\).
\end{center}

\begin{figure}[ht]
    \centering
    \incfig[0.5]{ejemplo2-isomorfismo}
    \caption{ejemplo2 isomorfismo}
    \label{fig:ejemplo2-isomorfismo}
\end{figure}

\begin{obs}
    Sea \(P\) es una propiedad invariante, si \(G\) satisface \(P\) y \(H\) no
    satisface \(P\) entonces \(G\cong H\).
\end{obs}
\subsubsection{Listado de invariantes:}

\begin{itemize}
    \item \(G\) tiene \(n\) vértices.
    \item \(G\) tiene \(m\) aristas.
    \item \(G\) tiene \(n\) vértices de grado \(k\).
    \item \(G\) tiene \(n\) ciclos de longitud \(k\).
    \item \(G\) tiene \(n\) vértices adyacentes a \(m\) vértices de grado
        \(k\).
    \item \(X\left(G\right)=k\).
    \item  \(G\) es conexo.
    \item \(G\) es \(k-\text{partito}\).
    \item \(G\) es plano.
    \item \(G\) es euleriano.
    \item \(G\) es hamiltoniano.
\end{itemize}


\subsection{Teorema}
\label{ssec:teorema}

La relación de isomorfidmo es una relación de equivalencia en el conjunto de
grafos simples.

\begin{itemize}
    \item Reflexiva: \(G\cong G\).
    \item Simétrica: Si \(G\cong H\), entonces \(H\cong G\).
    \item Transitiva: Si \(G\cong H\) y \(H\cong J\), entonces \(G\cong J\).
\end{itemize}

\subsection{Clases de isomorfismos}
\label{ssec:clases_de_isomorfismos}

Una \textbf{clase de isomorfismo} de un grafo es una clase de equivalencia de
grafos bajo la relación de isomofismos.

\subsubsection{Ejemplos}
\label{ssec:ejemplos}

\begin{itemize}
    \item Todos los caminos de \(n\) vértices son insomorfismos.
    \item El conjunto de todos los caminos de \(n\) vértices forma una clase de
        isomorfismo.
\end{itemize}

\subsection{Grafos sin etiquetas}
\label{ssec:grafos_sin_etiquetas}

Un \textbf{grafo sin etiquetas} es una  clase de isomorfismo.

\subsection{\(P_{n}\)}
\label{ssec:pn}

\textbf{Camino} \(P_{n}\) :comino con \(n\) vértices.

\subsection{\(C_{n}\)}
\label{ssec:cn}

\textbf{Ciclo} \(C_{n}\) :Ciclo con \(n\) vértices (n-ciclo).

\subsection{\(W_{n}\)}
\label{ssec:wn}

\textbf{Rueda} \(W_{n}\) : Ciclo \(C_{n}\) con vértice adicional adyacente a
todos los vertices del ciclo.

\subsection{\(K_{n}\)}
\label{ssec:kn}

\textbf{Grafo completo} \(K_{n}\) : Grafo simple con n vértices que contiene
exactamente una arista entre cada par de vértices. (Vértices adyacentes 2 a 2)

\subsection{\(K_{m,n}\)}
\label{ssec:kmn}

\textbf{Grafo bipartito completo} \(K_{m,n}\) : Grafo bipartito simple tal que dos vértices son adyacentes sii están en conjuntos partitos diferentes de tamaño \(m\) y \(n\) respectivamente. (\textbf{biclique}).

\begin{obs}
    Cuando se hace mención a \textbf{grafos sin nombrar los vértices} ,
    normalmente nos estaremos refiriendo a su \textbf{clase de isomorfismo}.
    Tecnicamente, <<\(H\) es un subgrafo de \(G\)>> que algún subgrafo  de \(G\)
    es isomorfo a \(H\) (se podría decir <<\(G\) contiene una copia de
    \(H\)>>). Sea \(C_3\)  un subgrafo de \(K_5\) (todo \textbf{grafo completo}
    de 5 vertices tiene 10 subgrafos isomorfos a \(C_3\) ) pero no significa
    que lo sea de \(K_{2,3}\).

    Sucedo lo mismo al preguntar si \(G\) <<es>> \(C_{n}\), pues se está
    preguntando si \(G\) es isomorfo con un ciclo de \(n\) vértices.
\end{obs}


\subsection{Ejemplo - Grafos de n vértices}
\label{ssec:ejemplo_grafos_de_n_vertices}

\begin{itemize}
    \item Si \(|V\left( G \right)|=n \) entonce se puede seleccionar
        \({n}\choose{2}\) parejas de vértices.
    \item El par podría formar una arista o no.
    \item Entonces ha \(2^{{n}\choose{2}}\) grafos simples de  \(n\) vértices. (Subconjuntos del conjunto de pares de vértices).
    \item Si \(n=4\), hay 64 grafos simples de 4 vértices.
        \item Hay 11 clases de isomorfismos.
\end{itemize}

\subsection{Grafo Autocomplementario}
\label{ssec:grafo_autocomplementario}

\begin{figure}[ht]
    \centering
    \incfig[0.5]{grafos-autocomplementarios}
    \caption{grafos autocomplementarios}
    \label{fig:grafos-autocomplementarios}
\end{figure}

Un grafo \(G\) es autocomplementario si es isomorfo a su complemento.

\subsection{Descomposicón}
\label{ssec:descomposicon}

Una \textbf{descomposició} de un grafo \(G\) es una lista de subgrafos
\(H_{i}\subset G\) tal que cada arista \(e\in E\left(G\right)\) pertenece
exactamente a un subgrafo de la lista.

\subsection{Ejercicio}
Un grafo \(K_{i,n-1} \text{ y } K_{n-1}\) forma una descomposición de \(K_{n}\).

\subsubsection{Demostración}

\(\ldots\)

\subsection{Teorema}

Un grafo \(G\) de \(n\) vértices es autocomplementario sii \(K_{n}\) tiene una
descomposición que cnsiste en dos copias de \(G\). (Grafos isomorfos con \(G\)
).

\begin{figure}[ht]
    \centering
    \incfig[0.5]{descomposicion-de-k5}
    \caption{descomposición de k5}
    \label{fig:descomposición-de-k5}
\end{figure}


\subsection{Grafo de Petersen}
El \textbf{grafo de Petersen} es el grafo simple cuyo conjunto de vértices son
subconjuntos de 2 elementos de un conjunto de 5 elementos y sus aristas son
pares disyuntos.
\begin{figure}[ht]
    \centering
    \incfig[0.5]{grafo-de-petersen}
    \caption{grafo de Petersen}
    \label{fig:grafo-de-petersen}
\end{figure}

\subsection{Cintura}
La cintura (\textit{girth}) de un grafo con ciclos es la longitud de su ciclo
más pequeñoo. Un grafo sin ciclos es de \textit{cintura} infinita.

\subsection{Circunferencia}
La \textit{circunferencia} de un grafo es la longitud de su ciclo más largo.

























\end{document}
