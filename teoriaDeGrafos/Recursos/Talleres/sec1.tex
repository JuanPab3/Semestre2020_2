\documentclass[a4paper]{book}

% ====| P A C K A G E S |==== %
\usepackage{import}
\usepackage{nicematrix}
%\usepackage[backend=biber, style=authoryear-icomp]{biblatex}    %Bibliography stuff
%\input{./../../commands.tex}
\usepackage[margin=1in,includefoot]{geometry}   %Margins
\usepackage[utf8]{inputenc} %Language stuff
%\usepackage [latin1]{inputenc}  %Spanish symbols
\usepackage[spanish]{babel} %Sets document language to spanish
\usepackage{tcolorbox}  %Frame boxes
\usepackage{enumerate}  %Lists options
\usepackage{mfirstuc}   %I use it to capitalize words
\usepackage{graphicx}   %Use Images
\usepackage{listings}   %For displaying code
\usepackage{titlesec}   %Costume titles/sections/...
\usepackage{hyperref}   %Linking options
\usepackage{multicol}	%Use column
\usepackage{amsmath}    %Display equations options
\usepackage{amssymb}    %More symbols
\usepackage{titling}    %Use title variables in other places
\usepackage{xcolor}     %To manage colors
\usepackage{transparent}%For figures
\usepackage{pdfpages}   %For figures
\usepackage{subfig}
% =========================== %

% ====| P A C K A G E S    S E T T I N G S |==== %
% \addbibresource{/media/jpi/Data/01_Education/04_bibiografia/bibliography.bib} %<--- Bibliography path
% ============================================== %

% ====| P E R S O N A L    C O M M A N D S    &    E N V I R O N M E N T S|==== %

% - New figure
\newcommand{\incfig}[2][1]{%
    \def\svgwidth{#1\columnwidth}
    \import{./figures/}{#2.pdf_tex}
}

\pdfsuppresswarningpagegroup=1

% - New page
\newcommand{\np}{\null\newpage}

% - new homework
\newenvironment{tarea}[3]
    {
        \null\newpage
        \begin{tcolorbox}
            \textbf{\asignatura\ -\ \autor}
            \subsection{\capitalisewords{#1}}
            \label{ssec:#1}
            \begin{flushright}
            \textbf{Desde:}  #2 \
            \textbf{Hasta:}  #3 \
            \end{flushright}
        \end{tcolorbox}

    \begin{enumerate}[{Ejercicio} 1.]
    }
    {
        \end{enumerate}
        \np
    }

% - new observation
\newenvironment{obs}
    {
        \begin{flushleft}
       \textbf{Observación}\
        \line(1,0){200} \
        \end{flushleft}
    }
    {
        \begin{flushright}
        \line(1,0){200}
        \end{flushright}
    }
% ============================================================================= %
\begin{document}

\section{15.08.2020 - Tarea 1}
\label{sec:tarea_1}

\begin{enumerate}[{Ej 1: }]
    \item Calcule el complemento de los siguientes grafos:
    \begin{figure}[!ht]
        \centering
        \incfig[0.4]{t1ej1}
        \incfig[0.4]{t1ej1a}
    \end{figure}
    \[ A_{G}= \begin{bNiceMatrix}[first-row,first-col]
         &a&b&c&d&e\\
        a&0&1&0&0&1\\
        b&1&0&1&1&1\\
        c&0&1&0&0&0\\
        d&0&1&0&0&1\\
        e&1&1&0&1&0\\
        \end{bNiceMatrix}
        I_{H}= \begin{bNiceMatrix}[first-row,first-col]
             &e_1&e_2&e_3&e_4&e_5&e_6&e_7&e_8\\
            a&1&0&0&1&1&0&0&0\\
            b&1&1&0&0&0&1&1&1\\
            c&0&0&1&1&0&0&0&1\\
            d&0&1&1&0&0&0&1&0\\
            e&0&0&0&0&1&1&1&1\\
        \end{bNiceMatrix} \]


    \item Encuentre un conjunto independiente de tamaño máximo en el grafo
        \(G\) y un clique de tamaño máximo en el grafo \(H\).

        \begin{itemize}
            \item En el caso de \(G\) se tieme al conjunto
                \(\left\{a,d,c\right\}\) como el conjunto independiente de
                tamaño máximo con un tamaño de 3 elémentos.
            \item Por otro lado, en \(H\) se pueden encontrar más de un clique
                de máximo tamaño los cuales son:
                \(\left\{a,b,e\right\}, \left\{e,b,d\right\},
                \left\{d,e,c\right\},\left\{e,c,a\right\}\).
        \end{itemize}


    \item Escriba la matriz de adyacencia e incidencia del siguiente grafo:
\begin{figure}[ht]
    \centering
    \incfig[1]{t1ej3}
\end{figure}
\[
    A_{M}=\begin{bNiceMatrix}[first-row,first-col]
         &a&b&c&d\\
        a&0&3&0&2\\
        b&3&0&1&1\\
        c&0&1&1&2\\
        d&2&1&2&0\\
    \end{bNiceMatrix}
    ,I_{M}=\begin{bNiceMatrix}[first-row,first-col]
         &e_1&e_2&e_3&e_4&e_5&e_6&e_7&e_8&e_9&e_{10}\\
        a&1&1&1&1&1&0&0&0&0&0\\
        b&1&1&1&0&0&1&0&0&0&1\\
        c&0&0&0&0&0&0&1&1&1&1\\
        d&0&0&0&1&1&1&1&1&0&0\\
    \end{bNiceMatrix}\]


    \item Dubuje los grafos correspondintes a las siguientes matrices:
        \[A_{A}=\begin{bNiceMatrix}[first-row,first-col]
             &a&b&c&d\\
            a&0&3&0&2\\
            b&3&0&1&1\\
            c&0&1&1&2\\
            d&2&1&2&0\\
        \end{bNiceMatrix}
        , I_{B}=\begin{bNiceMatrix}[first-row,first-col]
             &e_1&e_2&e_3&e_4&e_5&e_6&e_7&e_8\\
            a&1&1&1&1&0&0&0&0\\
            b&0&1&1&1&0&1&1&0\\
            c&0&0&0&1&1&0&0&0\\
            d&0&0&0&0&0&0&1&1\\
            e&0&0&0&0&1&1&0&0\\
        \end{bNiceMatrix}\]
\begin{figure}[ht]
    \centering
    \incfig[0.4]{tiej4}
\end{figure}

    \item Escriba la forma general de la matriz de adyacencia del gafo
        \(K_{n}\) y \(K_{m,n}\).
        p{\color{teal}
        \[A_{k_{n}}=\begin{bNiceMatrix}
            0&1&1&\Cdots&1&\Cdots&1\\
            1&0&1&\Cdots&1&\Cdots&1\\
            1&1&0&&1&\Cdots&1\\
            \Vdots&\Vdots&&\Ddots&\Vdots&&\Vdots\\
            1&1&1&\Cdots&0&&1\\
            \Vdots&\Vdots&\Vdots&&&\Ddots&\Vdots\\
            1&1&1&\Cdots&1&\Cdots&0
        \end{bNiceMatrix}
        .\]

        Un matriz \(n,n\) donde todos los elementos son 1
        con exepción de la diagonal donde son 0.}

    {\color{orange} En el caso \(K_{m,n}\) por el bien de la demostración, se
        va a suponer que el grafo K se va a partir en dos sub-grafos con diseño
        vertical.

\begin{figure}[ht]
    \centering
    \incfig[0.5]{knm}
\end{figure}
    \vspace{-0.5cm}
    Además, como se trata de un grafo sin etiquetas, si el grafo crece por
    vertices <<arriba>> la matriz crecera por la derecha, y si el grafo crece
    por <<abajo>> la matriz crecerá por la izquierda. la matriz de adyacencia
    sería de la forma:
    \[
    A_{K_{n,m}}=\begin{bNiceMatrix}[first-row,first-col]
     &1&2&\Cdots&n&1&2&\Cdots&m\\
    1&0&0&\Cdots&0&1&1&\Cdots&1\\
    2&0&0&\Cdots&0&1&1&\Cdots&1\\ \Vdots&
    \Vdots&\Vdots&\Ddots&\Vdots&\Vdots&\Vdots&\Ddots&\Vdots\\
    n&0&0&\Cdots&0&1&1&\Cdots&1\\
    1&1&1&\Cdots&1&0&0&\Cdots&0\\
    2&1&1&\Cdots&1&0&0&\Cdots&0\\ \Vdots&
    \Vdots&\Vdots&\Ddots&\Vdots&\Vdots&\Vdots&\Ddots&\Vdots\\
    m&1&1&\Cdots&1&0&0&\Cdots&0\\
    \end{bNiceMatrix}
    .\] }

\end{enumerate}

\end{document}
