\documentclass[a4paper]{book}

% ====| P A C K A G E S |==== %
\usepackage{import}
%\usepackage[backend=biber, style=authoryear-icomp]{biblatex}    %Bibliography stuff
%\ProvidesPackage{cmands}
% ====| P A C K A G E S |==== %
\usepackage{import}
%\usepackage[backend=biber, style=authoryear-icomp]{biblatex}    %Bibliography stuff
%\ProvidesPackage{cmands}
% ====| P A C K A G E S |==== %
\usepackage{import}
%\usepackage[backend=biber, style=authoryear-icomp]{biblatex}    %Bibliography stuff
%\ProvidesPackage{cmands}
% ====| P A C K A G E S |==== %
\usepackage{import}
%\usepackage[backend=biber, style=authoryear-icomp]{biblatex}    %Bibliography stuff
%\input{./../../commands.tex}
\usepackage[margin=1in,includefoot]{geometry}   %Margins
\usepackage[utf8]{inputenc} %Language stuff
\usepackage [latin1]{inputenc}  %Spanish symbols
\usepackage[spanish]{babel} %Sets document language to spanish
\usepackage{tcolorbox}  %Frame boxes
\usepackage{enumerate}  %Lists options
\usepackage{mfirstuc}   %I use it to capitalize words
\usepackage{graphicx}   %Use Images
\usepackage{listings}   %For displaying code
\usepackage{titlesec}   %Costume titles/sections/...
\usepackage{hyperref}   %Linking options
\usepackage{multicol}	%Use column
\usepackage{amsmath}    %Display equations options
\usepackage{amssymb}    %More symbols
\usepackage{titling}    %Use title variables in other places
\usepackage{xcolor}     %To manage colors
\usepackage{transparent}%For figures
\usepackage{pdfpages}   %For figures
% =========================== %

% ====| P A C K A G E S    S E T T I N G S |==== %
% \addbibresource{/media/jpi/Data/01_Education/04_bibiografia/bibliography.bib} %<--- Bibliography path
\setlength{\columnsep}{1cm}
\hypersetup{
    colorlinks,
    citecolor=black,
    filecolor=black,
    linkcolor=black,
    urlcolor=black
}
% ============================================== %

% ====| P E R S O N A L    C O M M A N D S    &    E N V I R O N M E N T S|==== %

% - New figure
\newcommand{\incfig}[2][1]{%
    \def\svgwidth{#1\columnwidth}
    \import{./figures/}{#2.pdf_tex}
}

\pdfsuppresswarningpagegroup=1

% - New page
\newcommand{\np}{\null\newpage}

% - vertical thic line
\newcommand\mybar{\kern1pt\rule[-\dp\strutbox]{.8pt}{\baselineskip}\kern1pt}

% - new homework
\newenvironment{tarea}[3]
    {
        \null\newpage
        \begin{tcolorbox}
            \textbf{\asignatura\ -\ \autor}
            \subsection{\capitalisewords{#1}}
            \label{ssec:#1}
            \begin{flushright}
            \textbf{Desde:}  #2 \\
            \textbf{Hasta:}  #3 \\
            \end{flushright}
        \end{tcolorbox}

    \begin{enumerate}[{Ejercicio} 1.]
    }
    {
        \end{enumerate}
        \np
    }



% ============================================================================= %


% ====| H E A T H E R S    S E T T I N G S |==== %

\titleformat{\chapter}[display]
    {\normalfont\huge\bfseries\raggedleft}{\chaptertitlename\ \thechapter}
    {20pt}{\Huge}
\titleformat{\section}[display]
    {\Large\bfseries}{}
    {0em}{}[\titlerule]


\newcommand{\titPag}{
    \begin{titlepage}
        \begin{flushright}
            \textsc{\large {\semestre\ Semestre}}\\
            \line(1,0){450} \\
            [0.635cm]
            \huge{\bfseries \asignatura} \\
            [0.2cm]
            \line(1,0){350} \\
            \LARGE{\bfseries \autor} \\
            [16.25cm]
        \end{flushright}
        \begin{flushright}
        \textsc{
            \universidad \\
            [0.1cm]
            \escuela \\
            [0.1cm]
            \carrera
        }
        \end{flushright}
    \end{titlepage}
}
% ============================================== %

% ====| C O D E    I N    F I L E S    S E T T I N G S |==== %
\definecolor{codegreen}{rgb}{0,0.6,0}
\definecolor{codegray}{rgb}{0.5,0.5,0.5}
\definecolor{codepurple}{rgb}{0.58,0,0.82}
\definecolor{backcolour}{rgb}{0.95,0.95,0.92}

\lstdefinestyle{mystyle}{
    backgroundcolor=\color{backcolour},
    commentstyle=\color{codegreen},
    keywordstyle=\color{magenta},
    numberstyle=\tiny\color{codegray},
    stringstyle=\color{codepurple},
    basicstyle=\ttfamily\footnotesize,
    breakatwhitespace=false,
    breaklines=true,
    captionpos=b,
    keepspaces=true,
    numbers=left,
    numbersep=5pt,
    showspaces=false,
    showstringspaces=false,
    showtabs=false,
    tabsize=2
}

\lstset{style=mystyle}
% ========================================================== %

\usepackage[margin=1in,includefoot]{geometry}   %Margins
\usepackage[utf8]{inputenc} %Language stuff
\usepackage [latin1]{inputenc}  %Spanish symbols
\usepackage[spanish]{babel} %Sets document language to spanish
\usepackage{tcolorbox}  %Frame boxes
\usepackage{enumerate}  %Lists options
\usepackage{mfirstuc}   %I use it to capitalize words
\usepackage{graphicx}   %Use Images
\usepackage{listings}   %For displaying code
\usepackage{titlesec}   %Costume titles/sections/...
\usepackage{hyperref}   %Linking options
\usepackage{multicol}	%Use column
\usepackage{amsmath}    %Display equations options
\usepackage{amssymb}    %More symbols
\usepackage{titling}    %Use title variables in other places
\usepackage{xcolor}     %To manage colors
\usepackage{transparent}%For figures
\usepackage{pdfpages}   %For figures
% =========================== %

% ====| P A C K A G E S    S E T T I N G S |==== %
% \addbibresource{/media/jpi/Data/01_Education/04_bibiografia/bibliography.bib} %<--- Bibliography path
\setlength{\columnsep}{1cm}
\hypersetup{
    colorlinks,
    citecolor=black,
    filecolor=black,
    linkcolor=black,
    urlcolor=black
}
% ============================================== %

% ====| P E R S O N A L    C O M M A N D S    &    E N V I R O N M E N T S|==== %

% - New figure
\newcommand{\incfig}[2][1]{%
    \def\svgwidth{#1\columnwidth}
    \import{./figures/}{#2.pdf_tex}
}

\pdfsuppresswarningpagegroup=1

% - New page
\newcommand{\np}{\null\newpage}

% - vertical thic line
\newcommand\mybar{\kern1pt\rule[-\dp\strutbox]{.8pt}{\baselineskip}\kern1pt}

% - new homework
\newenvironment{tarea}[3]
    {
        \null\newpage
        \begin{tcolorbox}
            \textbf{\asignatura\ -\ \autor}
            \subsection{\capitalisewords{#1}}
            \label{ssec:#1}
            \begin{flushright}
            \textbf{Desde:}  #2 \\
            \textbf{Hasta:}  #3 \\
            \end{flushright}
        \end{tcolorbox}

    \begin{enumerate}[{Ejercicio} 1.]
    }
    {
        \end{enumerate}
        \np
    }



% ============================================================================= %


% ====| H E A T H E R S    S E T T I N G S |==== %

\titleformat{\chapter}[display]
    {\normalfont\huge\bfseries\raggedleft}{\chaptertitlename\ \thechapter}
    {20pt}{\Huge}
\titleformat{\section}[display]
    {\Large\bfseries}{}
    {0em}{}[\titlerule]


\newcommand{\titPag}{
    \begin{titlepage}
        \begin{flushright}
            \textsc{\large {\semestre\ Semestre}}\\
            \line(1,0){450} \\
            [0.635cm]
            \huge{\bfseries \asignatura} \\
            [0.2cm]
            \line(1,0){350} \\
            \LARGE{\bfseries \autor} \\
            [16.25cm]
        \end{flushright}
        \begin{flushright}
        \textsc{
            \universidad \\
            [0.1cm]
            \escuela \\
            [0.1cm]
            \carrera
        }
        \end{flushright}
    \end{titlepage}
}
% ============================================== %

% ====| C O D E    I N    F I L E S    S E T T I N G S |==== %
\definecolor{codegreen}{rgb}{0,0.6,0}
\definecolor{codegray}{rgb}{0.5,0.5,0.5}
\definecolor{codepurple}{rgb}{0.58,0,0.82}
\definecolor{backcolour}{rgb}{0.95,0.95,0.92}

\lstdefinestyle{mystyle}{
    backgroundcolor=\color{backcolour},
    commentstyle=\color{codegreen},
    keywordstyle=\color{magenta},
    numberstyle=\tiny\color{codegray},
    stringstyle=\color{codepurple},
    basicstyle=\ttfamily\footnotesize,
    breakatwhitespace=false,
    breaklines=true,
    captionpos=b,
    keepspaces=true,
    numbers=left,
    numbersep=5pt,
    showspaces=false,
    showstringspaces=false,
    showtabs=false,
    tabsize=2
}

\lstset{style=mystyle}
% ========================================================== %

\usepackage[margin=1in,includefoot]{geometry}   %Margins
\usepackage[utf8]{inputenc} %Language stuff
\usepackage [latin1]{inputenc}  %Spanish symbols
\usepackage[spanish]{babel} %Sets document language to spanish
\usepackage{tcolorbox}  %Frame boxes
\usepackage{enumerate}  %Lists options
\usepackage{mfirstuc}   %I use it to capitalize words
\usepackage{graphicx}   %Use Images
\usepackage{listings}   %For displaying code
\usepackage{titlesec}   %Costume titles/sections/...
\usepackage{hyperref}   %Linking options
\usepackage{multicol}	%Use column
\usepackage{amsmath}    %Display equations options
\usepackage{amssymb}    %More symbols
\usepackage{titling}    %Use title variables in other places
\usepackage{xcolor}     %To manage colors
\usepackage{transparent}%For figures
\usepackage{pdfpages}   %For figures
% =========================== %

% ====| P A C K A G E S    S E T T I N G S |==== %
% \addbibresource{/media/jpi/Data/01_Education/04_bibiografia/bibliography.bib} %<--- Bibliography path
\setlength{\columnsep}{1cm}
\hypersetup{
    colorlinks,
    citecolor=black,
    filecolor=black,
    linkcolor=black,
    urlcolor=black
}
% ============================================== %

% ====| P E R S O N A L    C O M M A N D S    &    E N V I R O N M E N T S|==== %

% - New figure
\newcommand{\incfig}[2][1]{%
    \def\svgwidth{#1\columnwidth}
    \import{./figures/}{#2.pdf_tex}
}

\pdfsuppresswarningpagegroup=1

% - New page
\newcommand{\np}{\null\newpage}

% - vertical thic line
\newcommand\mybar{\kern1pt\rule[-\dp\strutbox]{.8pt}{\baselineskip}\kern1pt}

% - new homework
\newenvironment{tarea}[3]
    {
        \null\newpage
        \begin{tcolorbox}
            \textbf{\asignatura\ -\ \autor}
            \subsection{\capitalisewords{#1}}
            \label{ssec:#1}
            \begin{flushright}
            \textbf{Desde:}  #2 \\
            \textbf{Hasta:}  #3 \\
            \end{flushright}
        \end{tcolorbox}

    \begin{enumerate}[{Ejercicio} 1.]
    }
    {
        \end{enumerate}
        \np
    }



% ============================================================================= %


% ====| H E A T H E R S    S E T T I N G S |==== %

\titleformat{\chapter}[display]
    {\normalfont\huge\bfseries\raggedleft}{\chaptertitlename\ \thechapter}
    {20pt}{\Huge}
\titleformat{\section}[display]
    {\Large\bfseries}{}
    {0em}{}[\titlerule]


\newcommand{\titPag}{
    \begin{titlepage}
        \begin{flushright}
            \textsc{\large {\semestre\ Semestre}}\\
            \line(1,0){450} \\
            [0.635cm]
            \huge{\bfseries \asignatura} \\
            [0.2cm]
            \line(1,0){350} \\
            \LARGE{\bfseries \autor} \\
            [16.25cm]
        \end{flushright}
        \begin{flushright}
        \textsc{
            \universidad \\
            [0.1cm]
            \escuela \\
            [0.1cm]
            \carrera
        }
        \end{flushright}
    \end{titlepage}
}
% ============================================== %

% ====| C O D E    I N    F I L E S    S E T T I N G S |==== %
\definecolor{codegreen}{rgb}{0,0.6,0}
\definecolor{codegray}{rgb}{0.5,0.5,0.5}
\definecolor{codepurple}{rgb}{0.58,0,0.82}
\definecolor{backcolour}{rgb}{0.95,0.95,0.92}

\lstdefinestyle{mystyle}{
    backgroundcolor=\color{backcolour},
    commentstyle=\color{codegreen},
    keywordstyle=\color{magenta},
    numberstyle=\tiny\color{codegray},
    stringstyle=\color{codepurple},
    basicstyle=\ttfamily\footnotesize,
    breakatwhitespace=false,
    breaklines=true,
    captionpos=b,
    keepspaces=true,
    numbers=left,
    numbersep=5pt,
    showspaces=false,
    showstringspaces=false,
    showtabs=false,
    tabsize=2
}

\lstset{style=mystyle}
% ========================================================== %

\usepackage[margin=1in,includefoot]{geometry}   %Margins
\usepackage[utf8]{inputenc} %Language stuff
%\usepackage [latin1]{inputenc}  %Spanish symbols
\usepackage[spanish]{babel} %Sets document language to spanish
\usepackage{tcolorbox}  %Frame boxes
\usepackage{enumerate}  %Lists options
\usepackage{mfirstuc}   %I use it to capitalize words
\usepackage{graphicx}   %Use Images
\usepackage{listings}   %For displaying code
\usepackage{titlesec}   %Costume titles/sections/...
\usepackage{hyperref}   %Linking options
\usepackage{multicol}	%Use column
\usepackage{amsmath}    %Display equations options
\usepackage{amssymb}    %More symbols
\usepackage{titling}    %Use title variables in other places
\usepackage{xcolor}     %To manage colors
\usepackage{transparent}%For figures
\usepackage{pdfpages}   %For figures
% =========================== %

% ====| P A C K A G E S    S E T T I N G S |==== %
% \addbibresource{/media/jpi/Data/01_Education/04_bibiografia/bibliography.bib} %<--- Bibliography path
% ============================================== %

% ====| P E R S O N A L    C O M M A N D S    &    E N V I R O N M E N T S|==== %

% - New figure
\newcommand{\incfig}[2][1]{%
    \def\svgwidth{#1\columnwidth}
    \import{./figures/}{#2.pdf_tex}
}

\pdfsuppresswarningpagegroup=1

% - New page
\newcommand{\np}{\null\newpage}

% - new homework
\newenvironment{tarea}[3]
    {
        \null\newpage
        \begin{tcolorbox}
            \textbf{\asignatura\ -\ \autor}
            \subsection{\capitalisewords{#1}}
            \label{ssec:#1}
            \begin{flushright}
            \textbf{Desde:}  #2 \
            \textbf{Hasta:}  #3 \
            \end{flushright}
        \end{tcolorbox}

    \begin{enumerate}[{Ejercicio} 1.]
    }
    {
        \end{enumerate}
        \np
    }

% - new observation
\newenvironment{obs}
    {
        \begin{flushleft}
       \textbf{Observación}\
        \line(1,0){200} \
        \end{flushleft}
    }
    {
        \begin{flushright}
        \line(1,0){200}
        \end{flushright}
    }
% ============================================================================= %
\begin{document}

\section{12.08.2020 - Repaso Cálculo 2}
\label{sec:repaso_calculo_2}

\subsection{Sumas de Riemann con una variable}
\label{ssec:sumas_de_riemann_con_una_variable}

En matemáticas, la \textbf{suma de Riemann} es un tipo de aproxinación del
valor de una integralmediante una suma infinita. La suma se calcula dividiendo
las región en formas (rectángulos, trapezoies, parábolas o cúbicas) que juntas
forman una región similar a la región que se está midiento, a las cuales se les
calcula el area para así poder sumarlas (las areas).

\subsubsection{Generalizasión}

\begin{align*}
    &\Delta x= \frac{b-a}{n},\\
    \int_{a}^{b}f\left(x\right)dx &= \lim_{n \to \infty}\sum_{k=1}^{n} f\left( a+k\Delta x\right)\Delta x
.\end{align*}


\subsection{Terorema Fundamental del Cálculo}
\label{ssec:terorema_fundamental_del_calculo}

\subsubsection{Parte I}

Sea \(f\) una función integrable en \(\left[a,b\right]\), y definimos una nueva
función \(F\) en \(\left[a,b\right]\) por:

\[
    F\left(x\right) = \int_{a}^{x} f\left(t\right) dt
.\]

Si \(c \text{ pertenece a }\left[a,b\right] \) y  \(f\) es continua en \(c\), entonces \(F\) es diferenciable en \(c\), y:

\[
F'\left(c\right)=f\left(c\right)
.\]

Este \textbf{primera parte del teorema fundamental del cálculo} afirma que
podemos construir una \textbf{primitiva}\footnote{antíderivada} de cualquier
función continua por integración. Cuando se combina esto con el hecho de que
dos primitivas de la misma función son iguales salvo una constante, se obtiene
el la segunda parte de esté teorema. \footnote{Apostol}

\subsubsection{Parte II}
Suponiendo que \(f\) es continua en un intervalo abierto \(I\), y sea \(P\) cualqueir primitiva (ina integral indefinida, \(P'=f\)) de de \(f\) en \(I\). Entonces, para cada \(a\) y cada \( b\) en \(I\), se tiene que:
\[
\int_{a}^{x} f\left(t\right) dt = P\left(b\right) - P\left(a\right)
.\]

\subsubsection{Demostración}

Sea:
\[
F\left( x \right) =\int_{a}^{x} f\left(t\right) dt
.\]
Entonces, por la \textbf{primera parte del teorema}:
\[
F'\left(x\right) = f\left( x \right) = P'\left( x \right)
.\]
Ya que existe \(C\) constante, se tiene:
\[
F\left(x\right)=P\left(x\right)+C
.\]
Para poder calcular \(C\) se utiliza:
\[
F\left(a\right)=\int_{a}^{a} f\left(t\right) dt = 0 P\left(a\right)+C
.\]
\[
C = -P\left(a\right)
.\]
Apartir del caso de \(F\left(x\right)\) se tiene:
\[
P\left(x\right) = P\left(x\right) - P\left(a\right)
.\]
Ahora si se tiene \(x=b \) la todo lo anterior se puede reflejar de la
siguiente forma:
\[
P\left(b\right) = \int_{a}^{b}f\left(t \right) dt = P\left(b\right) - P\left(a\right)
.\]

\subsubsection{Conclusión}

Este teorema indica que se puede calcular el valor de una integral definida
simplemente restando, si conocemos su \textbf{antiderivada}.


\subsection{Coordenadas Polares}
\label{ssec:coordenadas_polares}

Teniendo en cuenta que el plano cartesiano es simplemente una \textbf{útil}
forma de interpretación de puntos. Pero además de esta, existen otros sistemas de coordenadas, como por ejemplo las \textbf{coordenadas polares}. Pensando a ambos sistemas en terminos de vectores, se pueden encontrar las siguientes distinciones:

\begin{description}
    \item[Coordenadas cartesianas] las componentes x,y de un vector (en \(\mathbb{R}^{2}\)).
    \item [Coordenadas polares] la magnitud (longitud) y dirección (ángulo) del vector.
\end{description}

\begin{figure}[ht]
    \centering
    \incfig[0.7]{de-cartesianas-a-polares}
    \caption{de cartesianas a polares}
    \label{fig:de-cartesianas-a-polares}
\end{figure}

Siendo \(\theta\) el ángulo y \(r\) la distancia desde el vertice, la notación
para las coordenadas polares es: \(\left( \theta,r \right)\) (a diferencia de
\(\left(x,y\right)\) utilizado en es sistema cartesiano). Para calcular la
transición de un sistema a otro se tiene:

\[
\sin\left(\theta\right)=\frac{y}{r}\to y=r\times \sin\left(\theta\right)
.\]
\[
\cos\left(\theta\right) = \frac{x}{r} \to x=r \times\cos\left(\theta\right)
.\]





\subsection{Ecuaciones Parametricas}
\label{ssec:ecuaciones_parametricas}

Las ecuaciones parampétricas de cualqueir recta \(r\) se obtienen por medio de
la siguiente expresión:
\[
\lambda \in \mathbb{R} \to
\]
\[
\to\left\{
    \begin{array}{ll}
        x=&a_1+\lambda \cdot v_1 \\
        y=&a_2+\lambda \cdot v_2
    \end{array}
\right
.\]

Donde:
\begin{itemize}
    \item \(x \text{ e } y\) son las coordenadas de cualquier punto
        \(P\left(x,y\right)\) de la recta.
    \item  \(a_1 \text{ y } a_2\) son las coordenadas de un punto conocido de
        la recta \(A\left(a_1,a_2\right)\).
    \item \(v_1 \text{ y }v_2\) son las componentes de un vector directo
        \(\overrightarrow{v}=\left( v_1,v_2\right)\) de \(r\).
    \item \(\lambda\) es un valor real que determina cada coordenada \(P\left(x,y\right)\) dependiendo del valor que se le asigne.

\end{itemize}

\subsubsection{Explicación}

Cualquier \textbf{recta} \(r\) que se pueda dibujar sobre una hoja de papel puede ser determinada analiticamente por medio de \textbf{punto} \(A\) que forme parte de dicha recta y una dirección que se puede expresar mediante un \textbf{vector no nulo} \(\overrightarrow{v}\).

\subsubsection{Definición de una recta por medio de un punto y un vector}

\begin{figure}[ht]
    \centering
    \incfig[0.7]{punto-y-vector}
    \caption{punto y vector}
    \label{fig:punto-y-vector}
\end{figure}

Como se observar en la figura \ref{fig:punto-y-vector}, \(r\) se trata de una recta que pasa por el punto \(A\) y cuya dirección vine dada por el \(\overrightarrow{v}\).

El vector encargado de determinar la dirección de la recta recibe el nombre de
\textbf{vector director} el cual no es único, ya que cualquier vector paralelo
a este sirve de igual forma para determinar la dirección de \(r\). Por lo tanto
si \(\overrightarrow{v}\) es un vector director de la recta r, también lo será
cualquier múltiplo de \(\overrightarrow{v}\left(\lambda \cdot
\overrightarrow{v}|\lambda \in\mathbb{R}1\right) \)

\subsubsection{De parametricas a \(\left(x,y\right)\)}

Teniendo en cuenta la ecuación vectorial de la recta, si \(A\left(a_1,a_2\right)\) es un punto conocido de una recta \(r\) que posee un vector director \(\overrightarrow{v}=\left( v_1,v_2 \right)\) y \(P\left(x,y\right)\) un punto cualquiera se sabe que:
\[
\left(x,y\right) =\left(a_1,a_2\right) + \lambda \left( v_1,v_2\right), \lambda \in\mathbb{R}
.\]
\[
\left(x,y\right) =\left(a_1+ \lambda v_1,a_2 + \lambda v_2\right), \lambda \in\mathbb{R}
.\]

Por lo que se puede concluir la suposición inicial.
\[
\lambda \in \mathbb{R} \to
\]
\[
\to\left\{
    \begin{array}{ll}
        x=&a_1+\lambda \cdot v_1 \\
        y=&a_2+\lambda \cdot v_2
    \end{array}
\right
.\]




\subsection{Límite para dos variables por definición}
\label{ssec:limite_para_dos_variables_por_definicion}

















\end{document}
